\documentclass[12pt]{article}
\usepackage{fullpage}
\usepackage{titlesec}
\usepackage{tikz}
\usepackage{amsfonts,amssymb}
\usepackage{amsmath}
\usepackage{comment}
\usetikzlibrary{automata, positioning}

\input ../libraries/mac.tex
\input ../libraries/mathmac.tex

\begin{document}
\pagestyle{plain}

\section*{Problem B5}
\subsection*{(a)}
Let $L = L_1 \cap L_2$, where $L_1$ and $L_2$ are regular languages. Since
$L_1$ and $L_2$ are regular, there exist equivalence relations $\simeq_{L_1}$
and $\simeq_{L_2}$ that are right-invariant, have a finite index, and have
contain a subset of equivalence classes who's union define the languages
$L_1$ and $L_2$ respectively. Let $\simeq_L$ be an equivalence relation, over
$\Sigma^*$, for the language L. Let the equivalence classes of $\simeq_L$
be defined as the union of the equivalence classes of $\simeq_{L_1}$ and
$\simeq_{L_2}$. It is immediate then, that $\simeq_L$ has a finite index since
the union of two finite sets is finite. We now create the language L from the
union of some of the equivalence classes of $\simeq_L$ while simultaneously
reducing the number of equivalence classes in $\simeq_L$ to make sure it is
well-defined. We start by combining equivalence classes in $\simeq_L$ in the
following way. If there $\exists x \in \Sigma^*$ such that
$x \in [x]_{\simeq_{L_1}}$ (some equivalence class of $\simeq_{L_1}$) and
$x \in [x]_{\simeq_{L_2}}$ (some equivalence class of $\simeq_{L_2}$)
then combine the two equivalence classes by taking their union. This is safe to
do since $x \simeq_{L_1} y,\ \forall y \in [x]_{\simeq_{L_1}}$ and
$x \simeq_{L_2} y',\ \forall y' \in [x]_{\simeq_{L_2}}$ therefore it must be true
that $y \simeq_{L_2} y',\ \forall y \in [x]_{\simeq_{L_1}},
y' \in [x]_{\simeq_{L_2}}$. Now we can define the language L as the union of all
such combined equivalence classes where the two equivalence classes
$[x]_{\simeq_{L_1}}$ and $[x]_{\simeq_{L_2}}$ that were combined where each in
their respective subset of equivalence classes that define their languages
$L_1$ and $L_2$ respectively. In other words, $\forall x \in \Sigma^*
\text{ if } x \in L_1 \text{ and } x \in L_2$ then $x$ is in some equivalence
class of $\simeq_{L_1}$ and $\simeq_{L_2}$ which we combine into one equivalence
class in $\simeq_L$. The union of all of these such classes then defines
$L_1 \cap L_2 = L$ where $\simeq_L$ is still finite since we have only reduced
the number of equivalence classes from a starting finite amount. Now we can
finally show that $\simeq_L$ is right-invariant since it directly follows from
the fact $\simeq_{L_1} \text{ and } \simeq_(L_2)$ are right-invariant:
$$(1)\ x \simeq_L y \implies x \simeq_{L_1} y \text{ and/or } x \simeq_{L_2} y$$
$$(2)\ x \simeq_{L_1} y \implies xz \simeq_{L_1} yz,\ \forall z \in \Sigma^*$$
$$(3)\ x \simeq_{L_2} y \implies xz \simeq_{L_2} yz,\ \forall z \in \Sigma^*$$
$$\text{therefore } x \simeq_L y \implies xz \simeq_L yz \text{ by a combination
of the above three implications}$$
Thus $L_1 \cap L_2$ is the union of some of the equivalence classes of an
equivalence relation $\simeq_L$ on $\Sigma^*$, which is right-invariant and has
a finite index and therefore by Myhill-Nerode is a regular language.
\subsection*{(b)}
Let $L = h^{-1}(L') = \{u \in \Sigma^*\ |\ h(u) \in L'\}$. Let $\simeq_L$ be an
equivalence relation on $\Sigma^*$ for the language L where
$\forall u,v \in \Sigma^*,\ u \simeq_L v \iff h(u) = h(v)$. Since $L'$ is a
regular language there exists an equivalence relation $\simeq_{L'} $
on $\Delta^*$ that has a finite index, is right-invariant, and where
$L'$ is equal to the union of some of the equivalence classes of $\simeq_{L'}$.
Therefore, $\simeq_L$ is finite since $\forall u \in \Sigma^*$ there are only
finitely many equivalence classes $h(u)$ can belong to thus there only
finitely many equivalence classes $u$ can belong to. Let $u \in L \iff h(u) \in
L'$ by the definition of $L$. Since $h(u) \in L'$ is equivalent to a subset of
the equivalence classes of $\simeq_{L'}$, $u \in L$ is also a subset of
equivalence classes of $\simeq_L$, specifically those where $h(u) \in L'$.
Finally, we can prove that $\simeq_L$ is right=invariant:
$$u \simeq_L v \implies uz \simeq_L vz,\ \forall u,v,z \in \Sigma^*$$
$$h(u) = h(v) \implies h(uz) = h(vz)$$
$$\implies h(u)h(z) = h(v)h(z)$$
$$\implies h(u)h(z) = h(u)h(z)$$

Thus, since $\simeq_L$ is an equivalence relation on $\Sigma^*$ that has a finite
index, is right-invariant, and where the union of some of the equivalence classes
of $\simeq_L$ is equal to $L$, by Myhill-Nerode, $L = h^{-1}(L')$ is a regular
language. Now we can prove by contradiction that the number of states of any
minimal DFA for $L = h^{-1}(L')$ is at most the number of states of any minimal
DFA for $L'$. Proof by contradiction so suppose not. Then there exists
$n$ states in $D_L$ and $m$ states in $D_{L'}$ such that $n > m$. Therefore,
there must exist $u,v \in \Sigma^*$ such that $h(u) = h(v)$ and $u \nsim_L v$
otherwise it would have to be the case that $n \le m$. Thus, $\forall w \in
\Sigma^*,\ h(uw) = h(vw)$ therefore $uw \in F \iff vw \in F$ so the equivalence
classes are indistinguisable and since $D_L$ is a minimal DFA unique states
associated with the equivalent classes are state equivalent therefore $D_L$ is
not a minimal DFA. The number of states for $D_L$ also cannot be strictly
smaller because then there would exist the case that $u \simeq_L v$ but
$h(u) \neq h(v)$ which means there exists some $w \in \Sigma^*$ such that
$h(uw) \in L'$ and $h(vw) \notin L'$ which implies $uw \in L$ and $vw \notin L$
which is a contradiction which is a contradiction to the right-invariance that
$u \simeq_L v \implies uw \simeq_L vw $.


\end{document}
