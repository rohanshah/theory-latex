\documentclass[12pt]{article}
\usepackage{fullpage}
\usepackage{titlesec}
\usepackage{tikz}
\usepackage{amsfonts,amssymb}
\usepackage{amsmath}
\usepackage{comment}
\usetikzlibrary{automata, positioning}

\input ../libraries/mac.tex
\input ../libraries/mathmac.tex

\begin{document}
\pagestyle{plain}

\section*{Problem B5}
\subsection*{(a)}
Let $L = L_1 \cap L_2$, where $L_1$ and $L_2$ are regular languages. Since
$L_1$ and $L_2$ are regular, there exist equivalence relations $\simeq_{L_1}$
and $\simeq_{L_2}$ that are right-invariant, have a finite index, and have
contain a subset of equivalence classes who's union define the languages
$L_1$ and $L_2$ respectively. Let $\simeq_L$ be an equivalence relation, over
$\Sigma^*$, for the language L. Let the equivalence classes of $\simeq_L$
be defined as the union of the equivalence classes of $\simeq_{L_1}$ and
$\simeq_{L_2}$. It is immediate then, that $\simeq_L$ has a finite index since
the union of two finite sets is finite. We now create the language L from the
union of some of the equivalence classes of $\simeq_L$ while simultaneously
reducing the number of equivalence classes in $\simeq_L$ to make sure it is
well-defined. We start by combining equivalence classes in $\simeq_L$ in the
following way. If there $\exists x \in \Sigma^*$ such that
$x \in [x]_{\simeq_{L_1}}$ (some equivalence class of $\simeq_{L_1}$) and
$x \in [x]_{\simeq_{L_2}}$ (some equivalence class of $\simeq_{L_2}$)
then combine the two equivalence classes by taking their union. This is safe to
do since $x \simeq_{L_1} y,\ \forall y \in [x]_{\simeq_{L_1}}$ and
$x \simeq_{L_2} y',\ \forall y' \in [x]_{\simeq_{L_2}}$ therefore it must be true
that $y \simeq_{L_2} y',\ \forall y \in [x]_{\simeq_{L_1}},
y' \in [x]_{\simeq_{L_2}}$. Now we can define the language L as the union of all
such combined equivalence classes where the two equivalence classes
$[x]_{\simeq_{L_1}}$ and $[x]_{\simeq_{L_2}}$ that were combined where each in
their respective subset of equivalence classes that define their languages
$L_1$ and $L_2$ respectively. In other words, $\forall x \in \Sigma^*
\text{ if } x \in L_1 \text{ and } x \in L_2$ then $x$ is in some equivalence
class of $\simeq_{L_1}$ and $\simeq_{L_2}$ which we combine into one equivalence
class in $\simeq_L$. The union of all of these such classes then defines
$L_1 \cap L_2 = L$ where $\simeq_L$ is still finite since we have only reduced
the number of equivalence classes from a starting finite amount. Now we can
finally show that $\simeq_L$ is right-invariant since it directly follows from
the fact $\simeq_{L_1} \text{ and } \simeq_(L_2)$ are right-invariant:
$$(1)\ x \simeq_L y \implies x \simeq_{L_1} y \text{ and/or } x \simeq_{L_2} y$$
$$(2)\ x \simeq_{L_1} y \implies xz \simeq_{L_1} yz,\ \forall z \in \Sigma^*$$
$$(3)\ x \simeq_{L_2} y \implies xz \simeq_{L_2} yz,\ \forall z \in \Sigma^*$$
$$\text{therefore } x \simeq_L y \implies xz \simeq_L yz \text{ by a combination
of the above three implications}$$
Thus $L_1 \cap L_2$ is the union of some of the equivalence classes of an
equivalence relation $\simeq_L$ on $\Sigma^*$, which is right-invariant and has
a finite index and therefore by Myhill-Nerode is a regular language.
\subsection*{(b)}
\end{document}
