\documentclass[12pt]{article}
\usepackage{fullpage}
\usepackage{titlesec}
\usepackage{tikz}
\usepackage{amsfonts,amssymb}
\usepackage{amsmath}
\usepackage{comment}
\usetikzlibrary{automata, positioning}

\input ../libraries/mac.tex
\input ../libraries/mathmac.tex

\begin{document}
\pagestyle{plain}
\titleformat{\subsection}[runin]
  {\normalfont\large\bfseries}{\thesubsection}{1em}{}

\title{Homework 3}
\author{Brooke Fugate, Michael O'Connor, Rohan Shah}
\date{}

\maketitle

\section*{Problem B1}
\subsection*{(a)} Show that $\approx$ is a left-invariant equivalence relation
and that $\sim$ is an equivalence relation that is both left and right invariant:
\subsubsection*{$\approx$ is a left-invariant equivalence relation:}
$$x \approx y \implies wx \approx wy ,\ \forall w \in \Sigma^*$$
$$\delta^*(p,x) \in F \iff \delta^*(p,y) \in F,\ \forall p \in Q \implies
\delta^*(p,wx) \in F \iff \delta^*(p,wy) \in F,\ \forall w \in \Sigma^*$$
$$\implies \delta^*(\delta^*(p,w),x) \in F \iff
\delta^*(\delta^*(p,w),y) \in F$$
$$\implies \exists \ p' \in Q \ | \ \delta^*(p,w) = p',
\ \forall p \in Q$$
$$\implies \delta^*(p',x) \in F \iff \delta^*(p',y) \in F$$
\subsubsection*{$\sim$ is an equivalence relation that is both left
and right invariant:}
$$x \sim y \implies uxv \sim uyv,\ \forall u, v \in \Sigma^*$$
$$\delta^*(p,x) = \delta^*(p,y),\ \forall p \in Q \implies
\delta^*(p,uxv) = \delta^*(p,uyv),\ \forall u,v \in \Sigma^*$$
$$\implies \delta^*(\delta^*(p,u), xv) = \delta^*(\delta^*(p,u), yv)$$
$$\implies \exists\ p' \in Q\ |\ \delta^*(p,u) = p'$$
$$\implies \delta^*(p',xv) = \delta^*(p', yv)$$
$$\implies \delta^*(\delta^*(p',x),v) = \delta^*(\delta^*(p',x),v)$$
$$\implies \exists\ p'' \in Q\ |\ \delta^*(p',x) = \delta^*(p',y) = p''
\text{ since } x \sim y \text{ is given.}$$
$$\implies \delta^*(p'',v) = \delta^*(p'',v)$$

\newpage
\subsection*{(b)}Let $n$ be the number of states in $Q$. Show that $\approx$
has at most $2^{n}$ equivalence classes and that $\sim$ has at most $n^{n}$
equivalence classes.
\subsubsection*{Show that $\approx$ has at most $2^n$ equivalence classes:}
The equivalence relation $\approx$ is defined as
$$x \approx y \iff \forall p \in Q,
\ \delta^*(p,x) \in F \iff \delta^*(p,y) \in F$$
which is equivalent to
$$x \approx y \iff \forall p \in Q,
\ \delta^*(p,x) \notin F \iff \delta^*(p,y) \notin F$$
Therefore, each equivalence class $[x_i]_{\approx}$, where $x_i \in \Sigma^*$,
is defined by a binary relation for each of the $n$ states in $Q$
i.e. either $\delta^*(p, x_i) \in F \text{ or } \delta^*(p,x_i) \notin F$
for each state $p \in Q$. Thus there are $2^n$ possible combinations of
transitions from each of the $n$ states to a state either in or not in $F$
for all strings in $\Sigma^*$ and thus $\approx$ has at most $2^n$ possible
equivalence classes.
\vspace{5 mm}
\newline
For example, let $n = 3$ and $Q = \{p_1, p_2, p_3\}.$ Then all the
$2^n = 2^3 = 8$ possible equivalence classes are defined as follows:
$$[x_1]_{\approx} : \delta^*(p_1, x_1) \in F,\ \delta^*(p_2, x_1) \in F,\ \delta^*(p_3, x_1) \in F$$
$$[x_2]_{\approx} : \delta^*(p_1, x_2) \in F,\ \delta^*(p_2, x_2) \in F,\ \delta^*(p_3, x_2) \notin F$$
$$[x_3]_{\approx} : \delta^*(p_1, x_3) \in F,\ \delta^*(p_2, x_3) \notin F,\ \delta^*(p_3, x_3) \in F$$
$$[x_4]_{\approx} : \delta^*(p_1, x_4) \in F,\ \delta^*(p_2, x_4) \notin F,\ \delta^*(p_3, x_4) \notin F$$
$$[x_5]_{\approx} : \delta^*(p_1, x_5) \notin F,\ \delta^*(p_2, x_5) \in F,\ \delta^*(p_3, x_5) \in F$$
$$[x_6]_{\approx} : \delta^*(p_1, x_6) \notin F,\ \delta^*(p_2, x_6) \in F,\ \delta^*(p_3, x_6) \notin F$$
$$[x_7]_{\approx} : \delta^*(p_1, x_7) \notin F,\ \delta^*(p_2, x_7) \notin F,\ \delta^*(p_3, x_7) \in F$$
$$[x_8]_{\approx} : \delta^*(p_1, x_8) \notin F,\ \delta^*(p_2, x_8) \notin F,\ \delta^*(p_3, x_8) \notin F$$

\subsubsection*{Show that $\sim$ has at most $n^n$ equivalence classes:}
The equivalence relation $\sim$ is defined as
$$x \sim y \iff \forall p \in Q,\ \delta^*(p,x) = \delta^*(p,y)$$
Therefore, each equivalence class $[x_i]_{\sim}$, where $x_i \in \Sigma^*$,
is defined as
$$\exists q \in Q\ |\ \delta^*(p,x_i) = q,\ \forall p \in Q$$
Informally, this states that an equivalence class $[x_i]_{\sim}$ is a
unique combination of mappings from each state in $Q$ to another state in $Q$
(including itself) defined by the transition function $\delta^*$
and the strings $x_i \in [x_i]_{\sim}$. Since there are $n$ possible states to
which a state can transition to, given an input $x_i$, and each relation is
defined for each of the $n$ states in $Q$, there are $n^n$ possible
combinations of state to state transitions. Therefore, there are $n^n$ possible
equivalence classes.
\newpage
\noindent For example, let $n = 3$ and $Q = \{p_1, p_2, p_3\}.$ Then all the
$n^n = 3^3 = 27$ possible equivalence classes are defined as follows:
$$[x_1]_{\sim} : \delta^*(p_1, x_1) = p_1,\ \delta^*(p_2, x_1) = p_1,\ \delta^*(p_3, x_1) = p_1$$
$$[x_2]_{\sim} : \delta^*(p_1, x_1) = p_1,\ \delta^*(p_2, x_1) = p_1,\ \delta^*(p_3, x_1) = p_2$$
$$[x_3]_{\sim} : \delta^*(p_1, x_1) = p_1,\ \delta^*(p_2, x_1) = p_1,\ \delta^*(p_3, x_1) = p_3$$
$$[x_4]_{\sim} : \delta^*(p_1, x_1) = p_1,\ \delta^*(p_2, x_1) = p_2,\ \delta^*(p_3, x_1) = p_1$$
$$[x_5]_{\sim} : \delta^*(p_1, x_1) = p_1,\ \delta^*(p_2, x_1) = p_2,\ \delta^*(p_3, x_1) = p_2$$
$$[x_6]_{\sim} : \delta^*(p_1, x_1) = p_1,\ \delta^*(p_2, x_1) = p_2,\ \delta^*(p_3, x_1) = p_3$$
$$[x_7]_{\sim} : \delta^*(p_1, x_1) = p_1,\ \delta^*(p_2, x_1) = p_3,\ \delta^*(p_3, x_1) = p_1$$
$$[x_8]_{\sim} : \delta^*(p_1, x_1) = p_1,\ \delta^*(p_2, x_1) = p_3,\ \delta^*(p_3, x_1) = p_2$$
$$[x_9]_{\sim} : \delta^*(p_1, x_1) = p_1,\ \delta^*(p_2, x_1) = p_3,\ \delta^*(p_3, x_1) = p_3$$
$$[x_{10}]_{\sim} : \delta^*(p_1, x_1) = p_2,\ \delta^*(p_2, x_1) = p_1,\ \delta^*(p_3, x_1) = p_1$$
$$[x_{11}]_{\sim} : \delta^*(p_1, x_1) = p_2,\ \delta^*(p_2, x_1) = p_1,\ \delta^*(p_3, x_1) = p_2$$
$$[x_{12}]_{\sim} : \delta^*(p_1, x_1) = p_2,\ \delta^*(p_2, x_1) = p_1,\ \delta^*(p_3, x_1) = p_3$$
$$[x_{13}]_{\sim} : \delta^*(p_1, x_1) = p_2,\ \delta^*(p_2, x_1) = p_2,\ \delta^*(p_3, x_1) = p_1$$
$$[x_{14}]_{\sim} : \delta^*(p_1, x_1) = p_2,\ \delta^*(p_2, x_1) = p_2,\ \delta^*(p_3, x_1) = p_2$$
$$[x_{15}]_{\sim} : \delta^*(p_1, x_1) = p_2,\ \delta^*(p_2, x_1) = p_2,\ \delta^*(p_3, x_1) = p_3$$
$$[x_{16}]_{\sim} : \delta^*(p_1, x_1) = p_2,\ \delta^*(p_2, x_1) = p_3,\ \delta^*(p_3, x_1) = p_1$$
$$[x_{17}]_{\sim} : \delta^*(p_1, x_1) = p_2,\ \delta^*(p_2, x_1) = p_3,\ \delta^*(p_3, x_1) = p_2$$
$$[x_{18}]_{\sim} : \delta^*(p_1, x_1) = p_2,\ \delta^*(p_2, x_1) = p_3,\ \delta^*(p_3, x_1) = p_3$$
$$[x_{19}]_{\sim} : \delta^*(p_1, x_1) = p_3,\ \delta^*(p_2, x_1) = p_1,\ \delta^*(p_3, x_1) = p_1$$
$$[x_{20}]_{\sim} : \delta^*(p_1, x_1) = p_3,\ \delta^*(p_2, x_1) = p_1,\ \delta^*(p_3, x_1) = p_2$$
$$[x_{21}]_{\sim} : \delta^*(p_1, x_1) = p_3,\ \delta^*(p_2, x_1) = p_1,\ \delta^*(p_3, x_1) = p_3$$
$$[x_{22}]_{\sim} : \delta^*(p_1, x_1) = p_3,\ \delta^*(p_2, x_1) = p_2,\ \delta^*(p_3, x_1) = p_1$$
$$[x_{23}]_{\sim} : \delta^*(p_1, x_1) = p_3,\ \delta^*(p_2, x_1) = p_2,\ \delta^*(p_3, x_1) = p_2$$
$$[x_{24}]_{\sim} : \delta^*(p_1, x_1) = p_3,\ \delta^*(p_2, x_1) = p_2,\ \delta^*(p_3, x_1) = p_3$$
$$[x_{25}]_{\sim} : \delta^*(p_1, x_1) = p_3,\ \delta^*(p_2, x_1) = p_3,\ \delta^*(p_3, x_1) = p_1$$
$$[x_{26}]_{\sim} : \delta^*(p_1, x_1) = p_3,\ \delta^*(p_2, x_1) = p_3,\ \delta^*(p_3, x_1) = p_2$$
$$[x_{27}]_{\sim} : \delta^*(p_1, x_1) = p_3,\ \delta^*(p_2, x_1) = p_3,\ \delta^*(p_3, x_1) = p_3$$

\newpage
\subsection*{(c)} Prove that $\lambda_L$ is left-invariant, $\mu_L$ is
left and right-invariant, and if $L$ is regular, then both $\lambda_L$ and
$\mu_L$ have a finite number of equivalence classes.

\subsubsection*{Prove that $\lambda_L$ is left-invariant:}
$$u\ \lambda_L\ v \implies wu\ \lambda_L\ wv,\ \forall w \in \Sigma^*$$
$$\forall z \in \Sigma^*,\ zu \in L \iff zv \in L \implies
\forall z,w \in \Sigma^*,\ zwu \in L \iff zwv \in L$$
$$\exists z' \in \Sigma^*\ |\ z'=zw,\ \forall z,w \in \Sigma^*$$
$$\forall z \in \Sigma^*,\ zu \in L \iff zv \in L \implies
z'u \in L \iff z'v \in L,\ z' \in \Sigma^*,\ \forall z,w \in \Sigma^*$$

\subsubsection*{Prove that $\mu_L$ is left and right-invariant:}
$$u\ \mu_L\ v \implies wu\ \mu_L\ wv,\ \forall w \in \Sigma^*$$
$$\forall x,y \in \Sigma^*,\ xuy \in L \iff xvy \in L \implies
\forall x,y,w \in \Sigma^*,\ xwuy \in L \iff xwvy \in L$$
$$\exists x' \in \Sigma^*\ |\ x'=xw,\ \forall x,w \in \Sigma^*$$
$$\forall x,y \in \Sigma^*,\ xuy \in L \iff xvy \in L \implies
x'uy \in L \iff x'vy \in L,\ x',y \in \Sigma^*,\ \forall x,y,w \in \Sigma^*$$

\subsubsection*{Prove that if $L$ is regular, then both $\lambda_L$ and $\mu_L$
have a finite number of equivalence classes:}
If $L$ is a regular language then there exists a DFA $D = (Q, \Sigma, \delta,
q_0, F)$ where $Q$ contains a finite number of states $n$.
\vspace{5 mm}
\newline
$\lambda_L$ is defined as
$$u\ \lambda_L\ v \iff\ \forall z \in \Sigma^*,\ zu \in L \iff zv \in L$$
which is equivalent to
$$u\ \lambda_L\ v \iff\ \forall z \in \Sigma^*,\ \delta^*(q_0, zu) \in F
\iff \delta^*(q_0, zv) \in F$$
$$\implies \forall z \in \Sigma^*,\ \delta^*(\delta^*(q_0, z),u) \in F
\iff \delta^*(\delta^*(q_0, z),v) \in F$$
$$\implies \forall z \in \Sigma^*,\ \exists\ p \in Q,\ \delta^*(q_0, z) = p$$
$$\implies \delta^*(p,u) \in F \iff \delta^*(p,v) \in F$$
and since $\{p\ |\ \delta^*(q_0, z) = p,\ \forall z \in \Sigma^*\} \subseteq Q$
then the above proposition is true if the following proposition is true
$$\forall p \in Q,\ \delta^*(p,u) \in F \iff \delta^*(p,v) \in F$$
which is equivalent to the above proposition $\approx$ which we proved has at
most $2^n$ equivalence classes. $\lambda_L$ has at most the number of
equivalence classes as $\approx$ since it is only defined for some
$p \in Q$ where as $\approx$ is for all $p \in Q$ and thus has a finite
number of equivalence classes.
\vspace{5 mm}
\newline
Similarly, $\mu_L$ is defined as
$$u \mu_L v \iff \forall x,y \in \Sigma^*,\ xuy \in L \iff xvy \in L$$
which is equivalent to
$$u \mu_L v \iff\ \forall x,y \in \Sigma^*,\ \delta^*(q_0, xuy) \in F \iff
\delta^*(q_0,xvy) \in F$$
$$\implies \delta^*(\delta^*(q_0,x),uy) \in F \iff
\delta^*(\delta^*(q_0, x),vy) \in F$$
$$\implies \exists\ p \in Q\ |\ \delta^*(q_0,x) = p,\ \forall x \in \Sigma^*$$
$$\implies \delta^*(p,uy) \in F \iff \delta^*(p, vy) \in F$$
$$\implies \exists\ q,q' \in Q\ |\ \delta^*(p,u) = q \text{ and }
\delta^*(p,v) = q'$$
$$\implies \delta^*(q,v) \in F \iff \delta^*(q',v) \in F$$
where the above proposition is restricted to the subset
$\{p\ |\ \delta^*(q_0, x) = p,\ \forall x \in \Sigma^*\} \subseteq Q$,
and thus the following proposition encompasses the above proposition
$$\forall p \in Q,\ \delta^*(p, u) = q,\ \delta^*(p,v) = q'
\ |\ \delta^*(q, y) \in F \iff \delta^*(q', y) \in F$$
which is further encompassed when $q = q'$, or $\delta^*(p,u) = \delta^*(p,v)$
yielding the following more general proposition
$$\forall p \in Q,\ \delta^*(p,u) = \delta^*(p,v)$$
which is equivalent to the above equivalence relation $\sim$ which we proved has
at most $n^n$ equivalence classes. $\mu_L$ has at most the number of equivalence
classes as $\sim$ since it is restricted to a subset of $p \in Q$ and
since the transition to $q$ and $q'$ only cover a binary relation whereas
the explicit $q = q'$ covers an $n$ sized relation. Therefore, $\mu_L$ has at
most $n^n$ equivalence classes and thus a finite number of equivalence classes.


\section*{Problem B2}
\subsection*{(a)}
h is a morphism:
$h(\delta_1(p,a)) = \delta_2(h(p),a), \forall p \in Q, a \in \Sigma$
$h(q_01) = q_02$
prove surjective proper homomorphism:
$\mapdef{\pi}{D}{D/\equiv}$ \newline
$\mapdef{\pi}(p) = [p]$ \newline
\#1 $\mapdef{\pi}(\delta(p, a) = [\delta(p,a) = \delta/\equiv([p], a)
= \delta / \equiv({\pi}(p,a)$ \newline
$\mapdef{\pi}(\delta(p, a) = [\delta(p,a) = \delta/\equiv([p], a)
= \delta / \equiv({\pi}(p,a)$ \newline
\#2 $\mapdef{\pi}(q_0) = [q_0]$ so 2 ok \newline
F-map : ${\pi}(F) \subseteq F$ \newline
B-map: ${\pi}^{-1}(F/\equiv) \subseteq F$ \newline
$\forall [p] \in F/\equiv, {\pi}^{-1}([p]) =\{ q \in Q p = q\}$ \newline
and $p \in F : \{q \in Q \mid p = q\} \subseteq F \text{ iff }
{\pi}^{-1}(F/\equiv) \subseteq F$ using \#2 \newline
if $p=q \text{ and } p\in F \implies q \in F$ from 2 \newline
and its surjective because $\forall \text{ states } q^{'} \in D/\equiv,
\exists q \in Q \text{ such that } {\pi}(p) = p^{'}$ \newline

\subsection*{(b)}
$p \equiv_D q \text{ iff } \forall w \in \Sigma^{*}
(\delta^{*}(p,w) \in F \text{ iff } \delta^{*}(q,w) \in F)$

To Prove: $ \equiv \subseteq \equiv_D$
$\equiv \subseteq \equiv_D \cup_{i \ge 0} \equiv_{Di}$
Claim: $\equiv \subseteq \equiv_i $ for all $i \ge 0$
Proof by induction. \newline
Base Case: $i = 0$.   Show $\equiv \subseteq \equiv_0 $ \newline
0 equivalence ($\equiv_0 $) means either $p,q \in F$ or $p,q \in$ F compliment. 
\newline $p \equiv_{0} q \text{ iff } p,q \in F \text{ or } p,q \notin F $.  
\newline
Assume $p \equiv q$. If $p \in F$, by (2), then $q \in F \implies p \equiv_0 q$. If $p \not \in F$, then by contrapositive $q \not\in F \implies p \equiv_0 q$. 
\newline
Inductive Step:\newline Inductive hypothesis: if $p \equiv q$ then $p \equiv_i q$\newline
By (1), $\delta (p,a) \equiv \delta (q,a)$. By the inductive hypothesis, $\delta (p,a) \equiv_i \delta (q,a)$, for all a $ \in \Sigma$. But, $p \equiv_i q$ and 
$\delta(p,a) \equiv_i \delta(q,a)$, which means $p \equiv_{i+1} q$.

\subsection*{(c)}
PART 1\newline
We have DFAs $D_1$ (trim) and $D_2$. There exists a morphism from $D_1$ to $D_2$
iff $\simeq_{D1} \subseteq \simeq_{D2}$. Suppose there exists a morphism $h:D_1 \rightarrow D_2$. Assume $\simeq_{D1} \subseteq \simeq_{D2}$. Because $D_1$ is
trim, if the morphism exists, then for every $p \in Q \text{ and } u
 \in \Sigma^{*} \text{ such that } p = \delta_1^{*}(q_{0,1}, u),$ $ 
h(p) = \delta^{*}_2(q_{0,2}, u)$.\newline
So, define (for any chosen u) $h(p) = \delta^{*}_2(q_{0,2}, u)$ such
 that $p = \delta^{*}_1(q_{0,1}, u)$.
But, we need to prove for any other $v$ such that $\delta^{*}_1(q_{0,1}, v) = p 
\text{ then } \delta^{*}_2 (q_{0,2} = \delta^{*}_2(q_{0,2}, u) = h(p)$.
We already have $\delta^{*}_1(q_{0,1} , v) = \delta^{*}_1(q_{0,1}, u) = p
\implies u \simeq_{D_1} v \text{ (by definition) and we also have } 
\simeq_{D_1} \subseteq \simeq_{D_2} \implies  u \simeq_{D_2} v$.
Therefore $\delta^{*}_2(q_{0,2} , u) = \delta^{*}_2(q_{0,2}, v)$ because $D_1$ is trim. \newline
Check that h is a morphism by showing $h(q_{0,1})=q_{0,2}$. By definition,
$q_{0,1}=\delta^{*}_1(q_{0,1} , \epsilon )$ and 
$q_{0,2}=\delta^{*}_2(q_{0,2} , \epsilon )$.
Pick a string that takes you to $p=\delta^{*}_1(q_{0,1}, u)$. \newline
$\delta_1(p,a) = \delta_1(\delta^{*}_1(q_{0,1} , u), a) =
\delta^{*}_1(q_{0,1}, ua)$. \newline
$h(q)=h(\delta_1(p,a))=\delta^{*}_2(q_{0,2}, ua) )=
\delta_2(\delta^{*}_2(q_{0,2}, u), u)
= \delta_2(h(p),a)$. 

\medskip

PART 2\newline
Case: $\implies$ \newline
If there is a F-map $h: D_1 \rightarrow D_2$, we know from part (1) that 
for any morphism between DFAs, $\simeq_{D_1} \subseteq \simeq_{D_2}$.  
We also know from Homework 1 that if there is an F-map, then $L(D_1) \subseteq L(D_2)$. \newline
Case: $\Longleftarrow$ \newline
We have $\simeq_{D_1} \subseteq \simeq_{D_2}$ and $L(D_1) \subseteq 
L(D_2)$ . We know from part 1 that $\simeq_{D_1} \subseteq \simeq_{D_2}$
 means there is a unique morphism, and if we pick any $p \in F_1$ then
we want to show $h(p) \in F_2$. 
So, $\delta^{*}_1(q_{0,1}, u) = p \in F_1 \text{ and with } L(D_1) \subseteq 
L(D_2) \text{ it must be that } \delta^{*}_2(q_{0,2}, u) = p \in F_2 
\text{ so the morphism is an F-map. }$

\medskip

PART 3\newline
If there is a B-map $h: D_1 -> D_2$ we know from part 1 that for any morphism between DFAs $h: D_1 -> D_2 \mid \simeq_{D_1} \subseteq \simeq_{D_2}$.  We also know from hwk1 that $L(D_2) \subseteq L(D_1)$ if there is a B-map.\newline
If $\simeq_{D_1} \subseteq \simeq_{D_2}$ and $L(D_2) \subseteq L(D_1)$ . We know from part 1 that $\simeq_{D_1} \subseteq \simeq_{D_2}$ means there is a morphism, and if we pick any $p \in F_2$ then $\delta^{*}_2(q_{0,2}, u) = p \in F_2 \text{ and with } L(D_2) \subseteq L(D_21) \text{ it must be that } \delta^{*}_1(q_{0,1}, u) = p \in F_1 \text{ so the morphism is a B-map. }$\newline

\subsection*{(d)}
Prove that $D_m$ is miminmal for L iff there is a unique proper homomorphism
$h:D \rightarrow D_m$ where D is a trim DFA accepting L.\newline
		First, assume that D' is a trim DFA such that $L(D') = L$ and
		that for every trim DFA D that accepts L there is an $h: D \rightarrow D'$ (UPH).
		Given any minimum DFA $D_m$ accepting L, then by our hypothesis there
		exists $f:D_m \rightarrow D'$ (UPH), then D' has at most as many states 
                as $D_m$ so D' is also minimal.
	We proved that for any trim DFA D there is a unique SPH from D to any
	minimal DFA $D_m$ accepting L. \newline
Converse: If D' is a DFA accepting L, and  if there is a UPH $h: D \rightarrow D'$ from any trim
DFA accepting L, then D' must be minimal.
	From our hypotheses we have $f: D_m \rightarrow D'$ and $h: D' \rightarrow D_m$
	From part (c) and from our hypothesis above, $f:D_m \rightarrow D'$.
	Since we have these two PH then the number of states must be equal because
	of the congruence of the two classes.

\section*{Problem B3}
\subsection*{(a)}
Suppose we have a regular language $L$ and that $\exists
D = \{Q, \Sigma, \delta, q_0, F\}$ such that $L(D) = L$.
We will show by a proof by construction that
$L^{(1/3)} = \{ w \mid www \in L\}$ is regular;
\newline First we will create 3 DFAs:
$$D^1_{pq} = \{Q, \Sigma, \delta, q_0, \{p\}\} \text{ and }
L(D^1_{pq}) = \{u \in \Sigma^* \mid \delta^* (q_0,u) = p\}$$
$$D^2_{pq} = \{Q, \Sigma, \delta, p, \{q\}\} \text{ and }
L(D^2_{pq}) = \{u \in \Sigma^* \mid \delta^* (p,u) = q\}$$
$$D^3_{pq} = \{Q, \Sigma, \delta, q, F\} \text{ and }
L(D^3_{pq}) = \{u \in \Sigma^* \mid \delta^* (q,u) \in F\}$$
No we can say that
$L^{(1/3)} \subseteq L(D^1_{pq}) \cap L(D^2_{pq}) \cap L(D^3_{pq})$ and that
$L^{(1/3)} = \bigcup_{p,q \in Q}
L(D^1_{pq}) \cap L(D^2_{pq}) \cap L(D^3_{pq})$.
Since regular languages are closed under union and intersection
$L^{(1/3)}$ is regular since it is made from 3 languages that are regular,
they have a DFA. \newline
Proof that the $L^{(1/3)} = \{ w \mid www \in L\}$:\newline
By executing  $D^1_{pq} , D^2_{pq} , D^3_{pq}$ on $w \in L^{(1/3)}$
we would have followed a path such that:
$$\delta^*(\delta^*(\delta^*, w), w), w) \in F \implies$$
$$www \in L \text{ therfore }w \in L^{(1/3)}$$
\newline $L^{(3)}$ is not regular. say $L = \{a^nb \mid n \ge 1\}$
now need to get an approrpriate suffix to break this thing from Myhil-Nerode.
\subsection*{(b)}
Suppose we have a regular language $L$ and that $\exists    D = \{Q, \Sigma,
\delta, q_0, F\}$ such that $L(D) = L$.
We will show by a proof by construction that all languages of the form
$L^{(1/k)} = \{ w \mid w^k \in L\}$ are regular;
\newline First we will create $k$ regular languages:
$$L^i_{p_i...p_{k-1}} = \{ w \in \Sigma^* \mid \delta^*(p_{i-1},w) = p_i\},
\forall i \mid 2 \le i \le k-1 $$
$$L^1_{p_i...p_{k-1}} = \{ w \in \Sigma^* \mid \delta^*(q_0,w) = p_1\}$$
$$L^k_{p_i...p_{k-1}} = \{ w \in \Sigma^* \mid \delta^*(p_{k-1},w) \in F\}$$
From this we can see that
$L = \bigcup_{p_i...p_{k-1} \in Q} \bigcap^k_{i=1} L^i_{p_i...p_k}$.
We must now prove that they are equal:
\newline
By executing a string $w$ on the DFAs we have created we would have followed
a path such that:
$\sigma^*(q_0,w) = p_1 , \sigma^*(p_1, w) = p_2 ... \sigma^*{p_{k-1}, w} \in F$.
Therefore we would have a path from $q_0$ to $q_F \in F$ that is for the word
$w^k$. So all workable strings on the right are in the left.
\newline
For all of the strings in in the left to the right,
we will choose a p to get started so it works.
\subsection*{(c)}
Is $L^{1/\infty}$ regular? $L^{1/\infty} = \bigcap_{k\ge1}L^{1/k}$.
From above we know all those languages are regular, and $k$ is finite so
yes $L^{1/\infty}$ is regular.
\newline
Is $\sqrt{L}$ regular? $\sqrt{L} = \bigcup_{k-1}L^{1/k}$.
From above we know all those languages are regular,
and k is finite so yes $\sqrt{L}$ is regular
\newline
Is $L^{\infty}$ regular? Since $L^{\infty}$ is an infininite series it can not be turned into a regular language.
\section*{Problem B4}

\section*{Problem B5}
\subsection*{a}
$\exists \simeq_{D_1}$ on $L_1$ such that $\forall x \in \Sigma^{*}_1 \mid [x]_{D_1}$ something, becuase it is a regular languate. This relation, since it is the Myhill-Nerode, relation is refined by all other relations, $\simeq_{X_1} \subseteq \simeq_{D_1}$.  $\exists \simeq_{D_2}$ on $L_2$ such that $\forall x \in \Sigma^{*}_2 \mid [x]_{D_2}$ something, becuase it is a regular language. This relation, since it is the Myhill-Nerode, relation is refined by all other relations, $\simeq_{X_2} \subseteq \simeq_{D_2}$.
There are 3 cases that can come from $L_1 \cap L_2$ : $L_1 \subseteq L_2$ , $L_2 \subseteq L_1$ , $L_1 \cap L_2 = {0}$\newline
case: $L_1 \subseteq L_2 \implies L_1 \cap L_2 = L_X \subseteq L_1 \subseteq L_2 \implies L_X = \bigcup_{x \in L_1}[x]_{D_1}$ and since each $[x]_{D_1}$ is a regular langauge and the union is finite $L_X$ is a regular language\newline
case: $L_2 \subseteq L_1\implies L_1 \cap L_2 = L_X \subseteq L_2 \subseteq L_1 \implies L_X = \bigcup_{x \in L_2}[x]_{D_2}$ and since each $[x]_{D_2}$ is a regular langauge and the union is finite $L_X$ is a regular language\newline
case: $L_1 \cap L_2 = {0}$ and this is trivially a regular language
\subsection*{b}

\section*{Problem B6}

\subsection*{(i)}
Prove that $R_{i_{0}+1} = R_{i_{0}}$ for some least $i_0$.\newline
Must show: $R_{i_{0} + 1} = (R_{i_{0}} \cup 
\{(\delta(p, a), \delta(q, a)) \mid (p, q)\in R_{i_{0}},\ a\in \Sigma\})_{\approx} 
= R_{i_{0}}$ for some least $i_0$.\newline

\medskip

Prove that $R_{i_{0}}$ is the smallest forward closure
containing $R$.
\subsection*{(ii)}
Prove that $p\equiv q$ iff the forward closure $R^{\dagger}$ of the
relation $R = \{(p, q)\}$ is good.
\end{document}
