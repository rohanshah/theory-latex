\documentclass[12pt]{article}
\usepackage{fullpage}
\usepackage{titlesec}
\usepackage{tikz}
\usepackage{amsfonts,amssymb}
\usepackage{amsmath}
\usepackage{comment}
\usetikzlibrary{automata, positioning}

\input ../libraries/mac.tex
\input ../libraries/mathmac.tex

\begin{document}
\pagestyle{plain}
\titleformat{\subsection}[runin]
  {\normalfont\large\bfseries}{\thesubsection}{1em}{}

\title{Homework 3}
\author{Brooke Fugate, Michael O'Connor, Rohan Shah}
\date{}

\maketitle

\section*{Problem B1}
\subsection*{a}
\subsection*{part 1}
$$x \approx y \implies wx \approx wy \mid \forall w \in \Sigma^*$$
$$\forall p \in Q, \delta^*(p,x) \in F \hbox{ iff } \delta^*(p,y) \in F \implies \forall q \in Q \mid \delta^*(\delta^*(q,w),x) \in F \hbox{ iff }  \delta^*(\delta^*(q,w),y) \in F$$
$$let p \in Q \mid \delta^*(q,w) = p$$
$$\forall p \in Q, \delta^*(p,x) \in F \hbox{ iff } \delta^*(p,y) \in F \implies \forall p \in Q | \delta^*(q,w) = p \mid \delta^*(p,x) \in F \hbox{ iff } \delta^*(p,y) \in F$$
\subsection*{part 2}
$$x \sim y \implies wx \sim wy \mid \forall w \in \Sigma^*$$
$$\forall p \in Q, \delta^*(p,x) = \delta^*(p,y) \implies \forall q \in Q \mid \delta^*(\delta^*(q,w),x) = \delta^*(\delta^*(q,w),y)$$
$$let p \in Q \mid \delta^*(q,w) = p$$
$$\forall p \in Q, \delta^*(p,x) \in F \hbox{ iff } \delta^*(p,y) \in F \implies \forall q \in Q \mid \delta^*(p,x) =\delta^*(p,y)$$

$$x \sim y \implies xw \sim yw \mid \forall w \in \Sigma^*$$
$$\forall p \in Q, \delta^*(p,x) = \delta^*(p,y) \implies \forall q \in Q \mid \delta^*(\delta^*(q,x),w) = \delta^*(\delta^*(q,y),w)$$
$$let p \in Q \mid \delta^*(q,x) = p$$
$$\forall p \in Q, \delta^*(p,w) \delta^*(p,w) \implies \forall p \in Q \mid \delta^*(p,w) =\delta^*(p,w)$$

\subsection*{b}

\subsection*{c}

\section*{Problem B2}
\subsection*{a}
h is a morphism:
$h(\sigma_1(p,a)) = \sigma_2(h(p),a), \forall p \in Q, a \in \Sigma$
$h(q_01) = q_02$
prove surjective proper homomorphism:
$\mapdef{\pi}{D}{D/\equiv}$ \newline
$\mapdef{\pi}(p) = [p]$ \newline
\#1 $\mapdef{\pi}(\sigma(p, a) = [\sigma(p,a) = \sigma/\equiv([p], a) = \sigma / \equiv({\pi}(p,a)$ \newline
$\mapdef{\pi}(\sigma(p, a) = [\sigma(p,a) = \sigma/\equiv([p], a) = \sigma / \equiv({\pi}(p,a)$ \newline
\#2 $\mapdef{\pi}(q_0) = [q_0]$ so 2 ok \newline
F-map : ${\pi}(F) \subseteq F$ \newline
B-map: ${\pi}^{-1}(F/\equiv) \subseteq F$ \newline
$\forall [p] \in F/\equiv, {\pi}^{-1}([p]) =\{ q \in Q p = q\}$ \newline
and $p \in F : \{q \in Q \mid p = q\} \subseteq F \text{ iff } {\pi}^{-1}(F/\equiv) \subseteq F$ using \#2 \newline
if $p=q \text{ and } p\in F \implies q \in F$ from 2 \newline
and its surjective because $\forall \text{ states } q^{'} \in D/\equiv , \exists q \in Q \text{ such that } {\pi}(p) = p^{'}$ \newline

\subsection*{b}
$p \equiv_D q \text{ iff } \forall w \in \Sigma^{*} (\delta^{*}(p,w) \in F \text{ iff } \delta^{*}(q,w) \in F)$

To Prove: $ \equiv \subseteq \equiv_D$
$ \equiv \subseteq \equiv_D \cap_{i \ge 0} \equiv_{Di}$
proof by induction
i = 0 -> get the congruence
inductive step:
$p \equiv_{Di} q and \sigma(p,a) \equiv_{Di} \sigma(q,a) \implies p \equiv_{D-{i+1}} q$
say w = wa, then good

\subsection*{c}
given 2 DFAs D1 and D2 with D1 trim prove that:
 $\exists \text{ DFA morphism } h:D_1 -> D_2 \text{ iff } \simeq_{D1} \subseteq \simeq_{D2}$
Proof :  $h(\delta^{*}_1(q_0, w) = h(\delta^{*}_1(q_0, w)$
if $u \simeq v , \delta^{*}_1(q_0, u) = \delta^{*}_1(q_0, v) , \sigma^{*}_2(q_{02}, u) = h\sigma^{*}_1(q_{01}, u) = h\sigma^{*}_1(q_{01}, v) = \sigma^{*}_1(q_{01}, v)  $ implying that $u \simeq_{D2} v therefore \simeq_{D1} \subseteq \simeq_{D2}$
proof of $\simeq_{D1} \subseteq \simeq_{D2}$
$\forall p \in Q \text{ let } p = \sigma^*(q01, u) \text{ and there must } \exists h(p) h(\sigma_1^*(q01, u)) = \sigma_2^*(q02, u)$
$h: Q1 -> Q2 \forall u \in \Sigma^*\mid \sigma^*_!(q_{01}, u) = p$
//Some other stuff I can not see because two people have gotten here that completely obscure my view
shows that h verifies two pieces of morphism stuff
We are going to pick 2 strings in the equivalence class to do something
$\sigma^*_1(q_{01},u) = $samething on $v = p \text{ iff } u \simeq_{d1} v$
isntead of $u,v$ pick any $v \in [u] \text{ for } \simeq_{d1}$
C2 if $\simeq_{d1} \subseteq \simeq_{d2} and L(D_1) \subseteq L(D_2)$
C3 then $h: D_1 -> D_2$ is an F-map
$\exists$ Fmap h 
$\exists$ B-map h

\subsection*{d}
d is miminmal for L iff there is a unique proper homomorphism H:D->Dm where D is a trim DFA accepting L
	One direction:
		Assume that D' is a trim DFA such that L(D') = L and that for every trim DFA D that accepts L there is an h: D-> D' (UPH).  Given any min DFA Dm accepting L, then by our hypothesis then there exists f:Dm - D' UPH, then D' has at most as states as Dm so D' is also minimal
	We proved that for any trim DFA D there is a unique SPH from D to any minimal DFA Dm accepting L
Converse: if D' is a DFA accepting L if there is a UPH: h: D->D' from any trim DFA accepting L then D' must be minimal.
	From our hypotheses we have f: Dm -> D' and h: D' -> Dm
	h: D' -> Dm from part c and from our hypothesis above f:Dm -> D'.  Since we have these two PH then the number of states must be equal because of the congruence of the two classes

\section*{Problem B3}
\subsection*{a}
Suppose we have a regular language $L$ and that $\exists    D = \{Q, \Sigma, \delta, q_0, F\}$ such that $L(D) = L$. We will show by a proof by construction that $L^{(1/3)} = \{ w \mid www \in L\}$ is regular; \newline First we will create 3 DFAs:
$$D^1_{pq} = \{Q, \Sigma, \delta, q_0, \{p\}\} \text{ and } L(D^1_{pq}) = \{u \in \Sigma^* \mid \delta^* (q_0,u) = p\}$$
$$D^2_{pq} = \{Q, \Sigma, \delta, p, \{q\}\} \text{ and } L(D^2_{pq}) = \{u \in \Sigma^* \mid \delta^* (p,u) = q\}$$
$$D^3_{pq} = \{Q, \Sigma, \delta, q, F\} \text{ and } L(D^3_{pq}) = \{u \in \Sigma^* \mid \delta^* (q,u) \in F\}$$
No we can say that $L^{(1/3)} \subseteq L(D^1_{pq}) \cap L(D^2_{pq}) \cap L(D^3_{pq})$ and that $L^{(1/3)} = \bigcup_{p,q \in Q}  L(D^1_{pq}) \cap L(D^2_{pq}) \cap L(D^3_{pq})$.  Since regular languages are closed under union and intersection $L^{(1/3)}$ is regular since it is made from 3 languages that are regular, they have a DFA. \newline
Proof that the $L^{(1/3)} = \{ w \mid www \in L\}$:\newline
By executing  $D^1_{pq} , D^2_{pq} , D^3_{pq}$ on $w \in L^{(1/3)}$ we would have followed a path such that:
$$\delta^*(\delta^*(\delta^*, w), w), w) \in F \implies$$
$$www \in L \text{ therfore }w \in L^{(1/3)}$$
\newline $L^{(3)}$ is not regular. say $L = \{a^nb \mid n \ge 1\}$ now need to get an approrpriate suffix to break this thing from Myhil-Nerode.
\subsection*{b}
Suppose we have a regular language $L$ and that $\exists    D = \{Q, \Sigma, \delta, q_0, F\}$ such that $L(D) = L$. We will show by a proof by construction that all languages of the form $L^{(1/k)} = \{ w \mid w^k \in L\}$ are regular;
\newline First we will create $k$ regular languages:
$$L^i_{p_i...p_{k-1}} = \{ w \in \Sigma^* \mid \delta^*(p_{i-1},w) = p_i\}, \forall i \mid 2 \le i \le k-1 $$
$$L^1_{p_i...p_{k-1}} = \{ w \in \Sigma^* \mid \delta^*(q_0,w) = p_1\}$$
$$L^k_{p_i...p_{k-1}} = \{ w \in \Sigma^* \mid \delta^*(p_{k-1},w) \in F\}$$
From this we can see that $L = \bigcup_{p_i...p_{k-1} \in Q} \bigcap^k_{i=1} L^i_{p_i...p_k}$. We must now prove that they are equal: \newline
By executing a string $w$ on the DFAs we have created we would have followed a path such that: $\sigma^*(q_0,w) = p_1 , \sigma^*(p_1, w) = p_2 ... \sigma^*{p_{k-1}, w} \in F$.  Therefore we would have a path from $q_0$ to $q_F \in F$ that is for the word $w^k$. So all workable strings on the right are in the left.
\newline For all of the strings in in the left to the right, we will choose a p to get started so it works.
\subsection*{c}
Is $L^{1/\infty}$ regular? $L^{1/\infty} = \bigcap_{k-1}L^{1/k}$.  From above we know all those languages are regular, and $k$ is finite so yes $L^{1/\infty}$ is regular.\newline
Is $\sqrt{L}$ regular? $\sqrt{L} = \bigcup_{k-1}L^{1/k}$.  From above we know all those languages are regular, and k is finite so yes $\sqrt{L}$ is regular

\section*{Problem B4}

\section*{Problem B5}

\section*{Problem B6}

\end{document}
