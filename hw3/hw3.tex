\documentclass[12pt]{article}
\usepackage{fullpage}
\usepackage{titlesec}
\usepackage{tikz}
\usepackage{amsfonts,amssymb}
\usepackage{amsmath}
\usepackage{comment}
\usetikzlibrary{automata, positioning}

\input ../libraries/mac.tex
\input ../libraries/mathmac.tex

\begin{document}
\pagestyle{plain}
\titleformat{\subsection}[runin]
  {\normalfont\large\bfseries}{\thesubsection}{1em}{}

\title{Homework 3}
\author{Brooke Fugate, Michael O'Connor, Rohan Shah}
\date{}

\maketitle

\section*{Problem B1}
\subsection*{(a)} Show that $\approx$ is a left-invariant equivalence relation
and that $\sim$ is an equivalence relation that is both left and right invariant:
\subsubsection*{$\approx$ is a left-invariant equivalence relation:}
$$x \approx y \implies wx \approx wy ,\ \forall w \in \Sigma^*$$
$$\delta^*(p,x) \in F \iff \delta^*(p,y) \in F,\ \forall p \in Q \implies
\delta^*(p,wx) \in F \iff \delta^*(p,wy) \in F,\ \forall w \in \Sigma^*$$
$$\implies \delta^*(\delta^*(p,w),x) \in F \iff
\delta^*(\delta^*(p,w),y) \in F$$
$$\implies \exists \ p' \in Q \ | \ \delta^*(p,w) = p',
\ \forall p \in Q$$
$$\implies \delta^*(p',x) \in F \iff \delta^*(p',y) \in F$$
\subsubsection*{$\sim$ is an equivalence relation that is both left
and right invariant:}
$$x \sim y \implies uxv \sim uyv,\ \forall u, v \in \Sigma^*$$
$$\delta^*(p,x) = \delta^*(p,y),\ \forall p \in Q \implies
\delta^*(p,uxv) = \delta^*(p,uyv),\ \forall u,v \in \Sigma^*$$
$$\implies \delta^*(\delta^*(p,u), xv) = \delta^*(\delta^*(p,u), yv)$$
$$\implies \exists\ p' \in Q\ |\ \delta^*(p,u) = p'$$
$$\implies \delta^*(p',xv) = \delta^*(p', yv)$$
$$\implies \delta^*(\delta^*(p',x),v) = \delta^*(\delta^*(p',x),v)$$
$$\implies \exists\ p'' \in Q\ |\ \delta^*(p',x) = \delta^*(p',y) = p''
\text{ since } x \sim y \text{ is given.}$$
$$\implies \delta^*(p'',v) = \delta^*(p'',v)$$

\newpage
\subsection*{(b)}Let $n$ be the number of states in $Q$. Show that $\approx$
has at most $2^{n}$ equivalence classes and that $\sim$ has at most $n^{n}$
equivalence classes.
\subsubsection*{Show that $\approx$ has at most $2^n$ equivalence classes:}
The equivalence relation $\approx$ is defined as
$$x \approx y \iff \forall p \in Q,
\ \delta^*(p,x) \in F \iff \delta^*(p,y) \in F$$
which is equivalent to
$$x \approx y \iff \forall p \in Q,
\ \delta^*(p,x) \notin F \iff \delta^*(p,y) \notin F$$
Therefore, each equivalence class $[x_i]_{\approx}$, where $x_i \in \Sigma^*$,
is defined by a binary relation for each of the $n$ states in $Q$
i.e. either $\delta^*(p, x_i) \in F \text{ or } \delta^*(p,x_i) \notin F$
for each state $p \in Q$. Thus there are $2^n$ possible combinations of
transitions from each of the $n$ states to a state either in or not in $F$
for all strings in $\Sigma^*$ and thus $\approx$ has at most $2^n$ possible
equivalence classes.
\vspace{5 mm}
\newline
For example, let $n = 3$ and $Q = \{p_1, p_2, p_3\}.$ Then all the
$2^n = 2^3 = 8$ possible equivalence classes are defined as follows:
$$[x_1]_{\approx} : \delta^*(p_1, x_1) \in F,\ \delta^*(p_2, x_1) \in F,\ \delta^*(p_3, x_1) \in F$$
$$[x_2]_{\approx} : \delta^*(p_1, x_2) \in F,\ \delta^*(p_2, x_2) \in F,\ \delta^*(p_3, x_2) \notin F$$
$$[x_3]_{\approx} : \delta^*(p_1, x_3) \in F,\ \delta^*(p_2, x_3) \notin F,\ \delta^*(p_3, x_3) \in F$$
$$[x_4]_{\approx} : \delta^*(p_1, x_4) \in F,\ \delta^*(p_2, x_4) \notin F,\ \delta^*(p_3, x_4) \notin F$$
$$[x_5]_{\approx} : \delta^*(p_1, x_5) \notin F,\ \delta^*(p_2, x_5) \in F,\ \delta^*(p_3, x_5) \in F$$
$$[x_6]_{\approx} : \delta^*(p_1, x_6) \notin F,\ \delta^*(p_2, x_6) \in F,\ \delta^*(p_3, x_6) \notin F$$
$$[x_7]_{\approx} : \delta^*(p_1, x_7) \notin F,\ \delta^*(p_2, x_7) \notin F,\ \delta^*(p_3, x_7) \in F$$
$$[x_8]_{\approx} : \delta^*(p_1, x_8) \notin F,\ \delta^*(p_2, x_8) \notin F,\ \delta^*(p_3, x_8) \notin F$$

\subsubsection*{Show that $\sim$ has at most $n^n$ equivalence classes:}
The equivalence relation $\sim$ is defined as
$$x \sim y \iff \forall p \in Q,\ \delta^*(p,x) = \delta^*(p,y)$$
Therefore, each equivalence class $[x_i]_{\sim}$, where $x_i \in \Sigma^*$,
is defined as
$$\exists q \in Q\ |\ \delta^*(p,x_i) = q,\ \forall p \in Q$$
Informally, this states that an equivalence class $[x_i]_{\sim}$ is a
unique combination of mappings from each state in $Q$ to another state in $Q$
(including itself) defined by the transition function $\delta^*$
and the strings $x_i \in [x_i]_{\sim}$. Since there are $n$ possible states to
which a state can transition to, given an input $x_i$, and each relation is
defined for each of the $n$ states in $Q$, there are $n^n$ possible
combinations of state to state transitions. Therefore, there are $n^n$ possible
equivalence classes.
\newpage
\noindent For example, let $n = 3$ and $Q = \{p_1, p_2, p_3\}.$ Then all the
$n^n = 3^3 = 27$ possible equivalence classes are defined as follows:
$$[x_1]_{\sim} : \delta^*(p_1, x_1) = p_1,\ \delta^*(p_2, x_1) = p_1,\ \delta^*(p_3, x_1) = p_1$$
$$[x_2]_{\sim} : \delta^*(p_1, x_1) = p_1,\ \delta^*(p_2, x_1) = p_1,\ \delta^*(p_3, x_1) = p_2$$
$$[x_3]_{\sim} : \delta^*(p_1, x_1) = p_1,\ \delta^*(p_2, x_1) = p_1,\ \delta^*(p_3, x_1) = p_3$$
$$[x_4]_{\sim} : \delta^*(p_1, x_1) = p_1,\ \delta^*(p_2, x_1) = p_2,\ \delta^*(p_3, x_1) = p_1$$
$$[x_5]_{\sim} : \delta^*(p_1, x_1) = p_1,\ \delta^*(p_2, x_1) = p_2,\ \delta^*(p_3, x_1) = p_2$$
$$[x_6]_{\sim} : \delta^*(p_1, x_1) = p_1,\ \delta^*(p_2, x_1) = p_2,\ \delta^*(p_3, x_1) = p_3$$
$$[x_7]_{\sim} : \delta^*(p_1, x_1) = p_1,\ \delta^*(p_2, x_1) = p_3,\ \delta^*(p_3, x_1) = p_1$$
$$[x_8]_{\sim} : \delta^*(p_1, x_1) = p_1,\ \delta^*(p_2, x_1) = p_3,\ \delta^*(p_3, x_1) = p_2$$
$$[x_9]_{\sim} : \delta^*(p_1, x_1) = p_1,\ \delta^*(p_2, x_1) = p_3,\ \delta^*(p_3, x_1) = p_3$$
$$[x_{10}]_{\sim} : \delta^*(p_1, x_1) = p_2,\ \delta^*(p_2, x_1) = p_1,\ \delta^*(p_3, x_1) = p_1$$
$$[x_{11}]_{\sim} : \delta^*(p_1, x_1) = p_2,\ \delta^*(p_2, x_1) = p_1,\ \delta^*(p_3, x_1) = p_2$$
$$[x_{12}]_{\sim} : \delta^*(p_1, x_1) = p_2,\ \delta^*(p_2, x_1) = p_1,\ \delta^*(p_3, x_1) = p_3$$
$$[x_{13}]_{\sim} : \delta^*(p_1, x_1) = p_2,\ \delta^*(p_2, x_1) = p_2,\ \delta^*(p_3, x_1) = p_1$$
$$[x_{14}]_{\sim} : \delta^*(p_1, x_1) = p_2,\ \delta^*(p_2, x_1) = p_2,\ \delta^*(p_3, x_1) = p_2$$
$$[x_{15}]_{\sim} : \delta^*(p_1, x_1) = p_2,\ \delta^*(p_2, x_1) = p_2,\ \delta^*(p_3, x_1) = p_3$$
$$[x_{16}]_{\sim} : \delta^*(p_1, x_1) = p_2,\ \delta^*(p_2, x_1) = p_3,\ \delta^*(p_3, x_1) = p_1$$
$$[x_{17}]_{\sim} : \delta^*(p_1, x_1) = p_2,\ \delta^*(p_2, x_1) = p_3,\ \delta^*(p_3, x_1) = p_2$$
$$[x_{18}]_{\sim} : \delta^*(p_1, x_1) = p_2,\ \delta^*(p_2, x_1) = p_3,\ \delta^*(p_3, x_1) = p_3$$
$$[x_{19}]_{\sim} : \delta^*(p_1, x_1) = p_3,\ \delta^*(p_2, x_1) = p_1,\ \delta^*(p_3, x_1) = p_1$$
$$[x_{20}]_{\sim} : \delta^*(p_1, x_1) = p_3,\ \delta^*(p_2, x_1) = p_1,\ \delta^*(p_3, x_1) = p_2$$
$$[x_{21}]_{\sim} : \delta^*(p_1, x_1) = p_3,\ \delta^*(p_2, x_1) = p_1,\ \delta^*(p_3, x_1) = p_3$$
$$[x_{22}]_{\sim} : \delta^*(p_1, x_1) = p_3,\ \delta^*(p_2, x_1) = p_2,\ \delta^*(p_3, x_1) = p_1$$
$$[x_{23}]_{\sim} : \delta^*(p_1, x_1) = p_3,\ \delta^*(p_2, x_1) = p_2,\ \delta^*(p_3, x_1) = p_2$$
$$[x_{24}]_{\sim} : \delta^*(p_1, x_1) = p_3,\ \delta^*(p_2, x_1) = p_2,\ \delta^*(p_3, x_1) = p_3$$
$$[x_{25}]_{\sim} : \delta^*(p_1, x_1) = p_3,\ \delta^*(p_2, x_1) = p_3,\ \delta^*(p_3, x_1) = p_1$$
$$[x_{26}]_{\sim} : \delta^*(p_1, x_1) = p_3,\ \delta^*(p_2, x_1) = p_3,\ \delta^*(p_3, x_1) = p_2$$
$$[x_{27}]_{\sim} : \delta^*(p_1, x_1) = p_3,\ \delta^*(p_2, x_1) = p_3,\ \delta^*(p_3, x_1) = p_3$$

\newpage
\subsection*{(c)}

\section*{Problem B2}
\subsection*{(a)}
h is a morphism:
$h(\sigma_1(p,a)) = \sigma_2(h(p),a), \forall p \in Q, a \in \Sigma$
$h(q_01) = q_02$
prove surjective proper homomorphism:
$\mapdef{\pi}{D}{D/\equiv}$ \newline
$\mapdef{\pi}(p) = [p]$ \newline
\#1 $\mapdef{\pi}(\sigma(p, a) = [\sigma(p,a) = \sigma/\equiv([p], a)
= \sigma / \equiv({\pi}(p,a)$ \newline
$\mapdef{\pi}(\sigma(p, a) = [\sigma(p,a) = \sigma/\equiv([p], a)
= \sigma / \equiv({\pi}(p,a)$ \newline
\#2 $\mapdef{\pi}(q_0) = [q_0]$ so 2 ok \newline
F-map : ${\pi}(F) \subseteq F$ \newline
B-map: ${\pi}^{-1}(F/\equiv) \subseteq F$ \newline
$\forall [p] \in F/\equiv, {\pi}^{-1}([p]) =\{ q \in Q p = q\}$ \newline
and $p \in F : \{q \in Q \mid p = q\} \subseteq F \text{ iff }
{\pi}^{-1}(F/\equiv) \subseteq F$ using \#2 \newline
if $p=q \text{ and } p\in F \implies q \in F$ from 2 \newline
and its surjective because $\forall \text{ states } q^{'} \in D/\equiv,
\exists q \in Q \text{ such that } {\pi}(p) = p^{'}$ \newline

\subsection*{(b)}
$p \equiv_D q \text{ iff } \forall w \in \Sigma^{*}
(\delta^{*}(p,w) \in F \text{ iff } \delta^{*}(q,w) \in F)$

To Prove: $ \equiv \subseteq \equiv_D$
$ \equiv \subseteq \equiv_D \cap_{i \ge 0} \equiv_{Di}$
proof by induction
i = 0 -> get the congruence
inductive step:
$p \equiv_{Di} q and \sigma(p,a) \equiv_{Di} \sigma(q,a) \implies
p \equiv_{D-{i+1}} q$
say w = wa, then good

\subsection*{(c)}
given 2 DFAs D1 and D2 with D1 trim prove that:
 $\exists \text{ DFA morphism } h:D_1 -> D_2 \text{ iff }
 \simeq_{D1} \subseteq \simeq_{D2}$
Proof :  $h(\delta^{*}_1(q_0, w) = h(\delta^{*}_1(q_0, w)$
if $u \simeq v , \delta^{*}_1(q_0, u) = \delta^{*}_1(q_0, v),
\sigma^{*}_2(q_{02}, u) = h\sigma^{*}_1(q_{01}, u) = h\sigma^{*}_1(q_{01}, v)
= \sigma^{*}_1(q_{01}, v)$ implying that
$u \simeq_{D2} v therefore \simeq_{D1} \subseteq \simeq_{D2}$
proof of $\simeq_{D1} \subseteq \simeq_{D2}$
$\forall p \in Q \text{ let } p = \sigma^*(q01, u) \text{ and there must }
\exists h(p) h(\sigma_1^*(q01, u)) = \sigma_2^*(q02, u)$
$h: Q1 -> Q2 \forall u \in \Sigma^*\mid \sigma^*_!(q_{01}, u) = p$
//Some other stuff I can not see because two people
have gotten here that completely obscure my view
shows that h verifies two pieces of morphism stuff
We are going to pick 2 strings in the equivalence class to do something
$\sigma^*_1(q_{01},u) = $samething on $v = p \text{ iff } u \simeq_{d1} v$
isntead of $u,v$ pick any $v \in [u] \text{ for } \simeq_{d1}$
C2 if $\simeq_{d1} \subseteq \simeq_{d2} and L(D_1) \subseteq L(D_2)$
C3 then $h: D_1 -> D_2$ is an F-map
$\exists$ Fmap h 
$\exists$ B-map h

\subsection*{(d)}
d is miminmal for L iff there is a unique proper homomorphism
H:D->Dm where D is a trim DFA accepting L
	One direction:
		Assume that D' is a trim DFA such that L(D') = L and
		that for every trim DFA D that accepts L there is an h: D-> D' (UPH).
		Given any min DFA Dm accepting L, then by our hypothesis then there
		exists f:Dm - D' UPH, then D' has at most as states as Dm so D' is also
		minimal.
	We proved that for any trim DFA D there is a unique SPH from D to any
	minimal DFA Dm accepting L
Converse: if D' is a DFA accepting L if there is a UPH: h: D->D' from any trim
DFA accepting L then D' must be minimal.
	From our hypotheses we have f: Dm -> D' and h: D' -> Dm
	h: D' -> Dm from part c and from our hypothesis above f:Dm -> D'.
	Since we have these two PH then the number of states must be equal because
	of the congruence of the two classes

\section*{Problem B3}
\subsection*{(a)}
Suppose we have a regular language $L$ and that $\exists
D = \{Q, \Sigma, \delta, q_0, F\}$ such that $L(D) = L$.
We will show by a proof by construction that
$L^{(1/3)} = \{ w \mid www \in L\}$ is regular;
\newline First we will create 3 DFAs:
$$D^1_{pq} = \{Q, \Sigma, \delta, q_0, \{p\}\} \text{ and }
L(D^1_{pq}) = \{u \in \Sigma^* \mid \delta^* (q_0,u) = p\}$$
$$D^2_{pq} = \{Q, \Sigma, \delta, p, \{q\}\} \text{ and }
L(D^2_{pq}) = \{u \in \Sigma^* \mid \delta^* (p,u) = q\}$$
$$D^3_{pq} = \{Q, \Sigma, \delta, q, F\} \text{ and }
L(D^3_{pq}) = \{u \in \Sigma^* \mid \delta^* (q,u) \in F\}$$
No we can say that
$L^{(1/3)} \subseteq L(D^1_{pq}) \cap L(D^2_{pq}) \cap L(D^3_{pq})$ and that
$L^{(1/3)} = \bigcup_{p,q \in Q}
L(D^1_{pq}) \cap L(D^2_{pq}) \cap L(D^3_{pq})$.
Since regular languages are closed under union and intersection
$L^{(1/3)}$ is regular since it is made from 3 languages that are regular,
they have a DFA. \newline
Proof that the $L^{(1/3)} = \{ w \mid www \in L\}$:\newline
By executing  $D^1_{pq} , D^2_{pq} , D^3_{pq}$ on $w \in L^{(1/3)}$
we would have followed a path such that:
$$\delta^*(\delta^*(\delta^*, w), w), w) \in F \implies$$
$$www \in L \text{ therfore }w \in L^{(1/3)}$$
\newline $L^{(3)}$ is not regular. say $L = \{a^nb \mid n \ge 1\}$
now need to get an approrpriate suffix to break this thing from Myhil-Nerode.
\subsection*{(b)}
Suppose we have a regular language $L$ and that $\exists    D = \{Q, \Sigma,
\delta, q_0, F\}$ such that $L(D) = L$.
We will show by a proof by construction that all languages of the form
$L^{(1/k)} = \{ w \mid w^k \in L\}$ are regular;
\newline First we will create $k$ regular languages:
$$L^i_{p_i...p_{k-1}} = \{ w \in \Sigma^* \mid \delta^*(p_{i-1},w) = p_i\},
\forall i \mid 2 \le i \le k-1 $$
$$L^1_{p_i...p_{k-1}} = \{ w \in \Sigma^* \mid \delta^*(q_0,w) = p_1\}$$
$$L^k_{p_i...p_{k-1}} = \{ w \in \Sigma^* \mid \delta^*(p_{k-1},w) \in F\}$$
From this we can see that
$L = \bigcup_{p_i...p_{k-1} \in Q} \bigcap^k_{i=1} L^i_{p_i...p_k}$.
We must now prove that they are equal:
\newline
By executing a string $w$ on the DFAs we have created we would have followed
a path such that:
$\sigma^*(q_0,w) = p_1 , \sigma^*(p_1, w) = p_2 ... \sigma^*{p_{k-1}, w} \in F$.
Therefore we would have a path from $q_0$ to $q_F \in F$ that is for the word
$w^k$. So all workable strings on the right are in the left.
\newline
For all of the strings in in the left to the right,
we will choose a p to get started so it works.
\subsection*{(c)}
Is $L^{1/\infty}$ regular? $L^{1/\infty} = \bigcap_{k-1}L^{1/k}$.
From above we know all those languages are regular, and $k$ is finite so
yes $L^{1/\infty}$ is regular.
\newline
Is $\sqrt{L}$ regular? $\sqrt{L} = \bigcup_{k-1}L^{1/k}$.
From above we know all those languages are regular,
and k is finite so yes $\sqrt{L}$ is regular

\section*{Problem B4}

\section*{Problem B5}

\section*{Problem B6}

\end{document}
