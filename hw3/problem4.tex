\documentclass[12pt]{article}
\usepackage{fullpage}
\usepackage{titlesec}
\usepackage{tikz}
\usepackage{amsfonts,amssymb}
\usepackage{amsmath}
\usepackage{comment}
\usetikzlibrary{automata, positioning}

\input ../libraries/mac.tex
\input ../libraries/mathmac.tex

\begin{document}
\pagestyle{plain}

\section*{Problem B4}
\subsection*{(a) $L_1 = \{wcw\ |\ w \in \{a,b\}^*\}$
\textnormal {is} not \textnormal{regular.}}
Proof by contradiction using Myhill-Nerode. $L_1$ is infinite since
each $x \in L_1$ is comprised of $w \in \{a,b\}^*$ which is an infinite
language. Proof by contradiction so assume $L_1$ is regular. Therefore, by
Myhill-Nerode there exists an equivalence relation $\simeq$ that is
right-invariant and contains a finite number of equivalence classes. Since
$\{a,b\}^*$ is infinte and $\{a,b\}^* \subseteq \Sigma^*$ there exist
$w,w' \in \{a,b\}^*$ and thus $w,w'\in \Sigma^*$ such that
$w \neq w'$ and $w \simeq w'$. Let $z \in \Sigma^*$ be the string $cw$.
Since $\simeq$ is right-invariant, $wz \simeq w'z$ which is
equal to $wcw \simeq w'cw$. But since $w \neq w'$, $wcw \in L_1$
and $w'cw \notin L_1$ which is a contradiction to the fact that they are
$\simeq$ equivalent strings thus $L_1$ is not regular.

\subsection*{(b) $L_2 = \{xy\ |\ x,y \in \{a,b\}^*
\textnormal{ and } |x| = |y|\}$ is \textnormal{regular.}}
Proof using Myhill-Nerode. Let $\simeq$ be an equivalence relation on
$\Sigma^*$ consisting of two equivalence classes $[x_{even}]$ and $[x_{odd}]$.
The index of $\simeq$ is finite since it consists of exactly 2
equivalence classes. We define $\simeq$ as $\forall x,y \in \Sigma^*,
x \simeq y \iff |x|\ \mod\ 2 = |y|\ \mod\ 2$, i.e. $\forall x \in \Sigma^*$,
$x \in [x_{even}]$ iff $|x|\ \mod\ 2 = 0$ and $x \in [x_{odd}]$ iff
$|x|\ \mod\ 2 = 1$. Now we can prove that $\simeq$ is right-invariant:
$$ x \simeq y \implies xz \simeq yz,\ \forall x,y,x \in \Sigma^*$$
$$ |x|\ \mod\ 2 = |y|\ \mod\ 2 \implies |xz|\ \mod\ 2 = |yz|\ \mod\ 2$$
$$ |x|\ \mod\ 2 = |y|\ \mod\ 2 \implies |x|+|z|\ \mod\ 2 = |y|+|z|\ \mod\ 2$$
$$ |x|\ \mod\ 2 = |y|\ \mod\ 2 \implies |x|\ \mod\ 2 = |y|\ \mod\ 2 \wedge
|z|\ \mod\ 2 = |z|\ \mod\ 2$$
Finally we show that $L_2$ is the union of some equivalence classes of $\simeq$.
$L_2$ is equal to the equivalence class $[x_{even}]$ with the corrollary that
$\{a,b\}^* = \Sigma^*$ when $\Sigma = \{a,b\}$ as is the case in this problem:
$$ z \in L_2 \iff \exists x,y \in \Sigma^* \text{ such that } |x| = |y|
\text{ and } z = xy$$
$$\implies |z| = |xy| = |x| + |y| = |x| + |x| = 2|x|$$
$$\implies |z|\ \mod\ 2 = 2|x|\ \mod\ 2 = 0$$
$$\therefore z \in L_2 \iff z \in [x_{even}]$$
Thus by Myhill-Nerode $L_2$ is regular, since it is the union of some of the
equivalence classes of an equivalence relation $\simeq$ on $\Sigma^*$, which
is right-invariant and has a finite index.

\subsection*{(c) $L_3 = \{a^n\ |\ n \textnormal{ is a prime number}\}$
\textnormal{is} not \textnormal{regular}}
Proof by contradiction using Myhill-Nerode. $L_3$ is infinte since there are
an infinte number of prime numbers $n$. Proof by contradiction so assume $L_3$
is regular. Therefore, by Myhill-Nerode, there exists an equivalence relation
$\simeq$ that is right-invariant and with a finite number of equivalence
classes. Consider the infinite set $\{a^2, a^3, a^5, a^7, ...\} \subseteq
\Sigma^*$ i.e. the infinite subset of strings in $\Sigma^*$ with prime length.
Since it is an infinte subset of $\Sigma^*$ and there are a finite number of
equivalence classes over $\simeq$, there exists $a^i$ and $a^j$ such that $i$
and $j$ are prime, $i < j$, and $a^i \simeq a^j$. Since $i < j$ there exists
some $r > 0$ such that $i + r  = j$. Therefore $a^i \simeq a^{i+r}$.
By the right-invariance of
$\simeq$, $a^i \simeq a^{i+r} \implies a^{i}z \simeq a^{i+r}z,
\ \forall z \in \Sigma^*$. Let $z = a^r \in \Sigma^*$ then
$a^{i}a^{r} \simeq a^{i+r}a^{r}$ which is equivalent to
$a^{i+r} \simeq a^{i+2r}$. Therefore, $a^i \simeq a^{i+r} \simeq a^{i+2r}$.
By induction it follows that, $a^i \simeq a^{i + kr},\ \forall k \ge 0$.
Let $k = i$, then $a^i \simeq a^{i+ir} = a^{i(1+r)}$. Therefore,
$a^i \simeq a^{ix}$ and since $i < ix$, $x \ge 2$, thus $ix$ is not prime. So
$a^i \simeq a^{ix}$ but $a^i \in L_3$ and $a^{ix} \notin L_3$ which is a
contradiction to their $\simeq$ equivalence therefore $L_3$ is not regular.

\subsection*{(d) $L_4 = \{a^{m}n^{n}\ |\ gcd(m,n) = 17\}$
\textnormal{is} not \textnormal{regular}}
Proof by contradiction using Myhill-Nerode. $L_4$ is infinte since there are
an infinte number of multiplies of 17 by prime numbers so maintaints the
greatest common denomiator as 17. Proof by contradiction so assume $L_4$ is
regular. Therefore, by Myhill-Nerode, there exists an equivalence relation
$\simeq$ that is right-invariant and with a finite number of equivalence
classes. Consider the infinte set $\{a^{17i}\}$ where $i$ is a
prime number. Since it is an infinite subset of $\Sigma^*$ and there are a
finite number of equivalence classes, there exists $p$ and $q$ such that
$p < q$, $p$ and $q$ are both prime, and $a^{17p} \simeq a^{17q}$. By the
right-invariance of $\simeq$, $a^{17p} \simeq a^{17q} \implies
a^{17p}z \simeq a^{17q}z,\ \forall z \in \Sigma^*$.
Let $z = b^{17q} \in \Sigma^*$ then $a^{17p}b^{17q} \simeq a^{17q}b^{17q}$.
By the fact that $p$ and $q$ are both prime, $gcd(17p, 17q) = 17$ so
$a^{17p}b^{17q} \in L_4$ but $gcd(17q, 17q) = 17q$ so
$a^{17q}b^{17q} \notin L_4$ which is a contradiction to the fact that the two
are $\simeq$ equivalent. Therefore, $L_4$ is not a regular language.

\end{document}
