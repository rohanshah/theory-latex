\documentclass[12pt]{article}
\usepackage{amsfonts,amssymb}
\usepackage{amsmath}

\setlength{\topmargin}{-.5in}
\setlength{\oddsidemargin}{0 in}
\setlength{\evensidemargin}{0 in}
\setlength{\textwidth}{6.5truein}
\setlength{\textheight}{8.5truein}

\input ../libraries/mac.tex
\input ../libraries/mathmac.tex

\def\fseq#1#2{(#1_{#2})_{#2\geq 1}}
\def\fsseq#1#2#3{(#1_{#3(#2)})_{#2\geq 1}}
\def\qleq{\sqsubseteq}

%
\begin{document}
\begin{center}
\fbox{{\Large\bf Spring, 2014 \hspace*{0.4cm} CIS 511}}\\
\vspace{1cm}
{\Large\bf Introduction to the Theory of Computation\\
Jean Gallier \\
\vspace{0.5cm}
Homework 3}\\[10pt]
February 20, 2014; Due March 6, 2014, {\it beginning of class}\\
\end{center}

``A problems'' are for practice only, and should not
be turned in.


\vspace {0.25cm}
\noindent
{\bf Problem A1.} 
Prove that every finite language is regular.

\vspace {0.25cm}\noindent
{\bf Problem A2.} 
Sketch an algorithm for deciding whether two regular
expressions $R, S$ are equivalent
(i,e, whether $\s{L}[R] = \s{L}[S]$).

\vspace {0.25cm}\noindent
{\bf Problem A3.} 
Given any language $L\subseteq \Sigma^*$, let
\[L^R = \{w^R \mid w \in L\},\]
the {\it reversal language of $L$\/} (where $w^R$
denotes the reversal of the string $w$).
Prove that if $L$ is regular, then $L^R$
is also regular.

\vspace {0.5cm}
``B problems'' must be turned in.

% cis511s01sol2.tex
\vspace {0.25cm}\noindent
{\bf Problem B1 (60 pts).} 
Let $D=(Q,\Sigma,\delta,q_{0},F)$ be a deterministic finite
automaton. Define the relations $\approx$ and $\sim$ on $\Sigma^{*}$
as follows:
\begin{align*}
x \approx y\quad&\hbox{if and only if},\quad\hbox{for all}\quad p\in Q,\\
& \delta^{*}(p,x) \in F\quad\hbox{iff}\quad \delta^{*}(p,y) \in F,
\end{align*}
and
$$x \sim y\quad\hbox{if and only if},\quad\hbox{for all}\quad p\in Q,\quad
 \delta^{*}(p,x) = \delta^{*}(p,y).$$

\medskip
(a) Show that $\approx$ is a left-invariant equivalence relation
and that $\sim$ is an equivalence relation
that is both left and right invariant.
(A relation $R$ on $\Sigma^{*}$ is {\it left invariant\/}
iff $uRv$ implies that $wuRwv$ for all $w \in \Sigma^{*}$,
and $R$ is {\it right invariant\/} iff $uRv$ implies that
$uwRvw$ for all $w \in \Sigma^{*}$.)

\medskip
(b) Let $n$ be the number of states in $Q$ (the set of states of $D$).
Show that $\approx$ has at most $2^{n}$ equivalence classes
and that $\sim$ has at most $n^{n}$ equivalence classes.

\medskip
(c) 
Given any language $L\subseteq \Sigma^*$,
define the relations $\lambda_L$ and $\mu_L$ on $\Sigma^{*}$
as follows:
$$u\, \lambda_L\, v\quad\hbox{iff},\quad\hbox{for all}\quad z\in \Sigma^*
,\quad zu \in L\quad\hbox{iff}\quad zv \in L,$$
and
$$u\, \mu_L\, v\quad\hbox{iff},\quad\hbox{for all}\quad x, y\in\Sigma^*,\quad
 xuy \in L\quad\hbox{iff}\quad xvy \in L.$$

\medskip
Prove that $\lambda_L$ is left-invariant, and that
$\mu_L$ is left and right-invariant. Prove that if $L$ is regular,
then both $\lambda_L$ and $\mu_L$ have a finite number of
equivalence classes.

\medskip\noindent
{\it Hint\/}: Show that the number of classes of $\lambda_L$ is at most
the number of classes of $\approx$, and 
that the number of classes of $\mu_L$ is at most
the number of classes of $\sim$.



\vspace{0.5cm}\noindent
{\bf Problem B2 (80 pts).} 
Recall from class that
given any  DFA 
$D = (Q, \Sigma, \delta, q_{0}, F)$,
a {\it congruence $\equiv$ on $D$\/} is an equivalence
relation $\equiv$ on $Q$ satisfying the following
conditions:
\begin{enumerate}
\item[(1)] 
If $p \equiv q$, then $\delta(p, a)\equiv \delta(q, a)$,
for all $p, q\in Q$ and all $a\in \Sigma$.
\item[(2)] 
If $p \equiv q$ and $p\in F$, then $q\in F$,
for all $p, q\in Q$.
\end{enumerate}

\medskip
(a)
Given a congruence $\equiv$ on a DFA $D$, we define
the {\it quotient DFA $D/\equiv$\/} as follows:
denoting the equivalence class of a state $p\in Q$ as $[p]$,
$$D/\equiv\> = (Q/\equiv, \Sigma, \delta/\equiv, [q_0], F/\equiv),$$
where
$$\delta/\equiv([p], a) = [\delta(p, a)].$$

\medskip
Why is $D/\equiv$ well defined? 
Prove that there is a 
surjective proper
homomorphism $\mapdef{\pi}{D}{D/\equiv}$, and thus,
that $L(D) = L(D/\equiv)$ (you may use results from HW1).

\medskip
(b) Given a DFA $D$, prove that the state equivalence
relation $\equiv_D$ is the coarsest congruence on $D$
(this means that if $\equiv$ is any congruence on $D$,
then $\equiv\>\subseteq\> \equiv_D$).

\medskip
(c)
Given two  DFA's
$D_1 = (Q_1, \Sigma, \delta_1, q_{0, 1}, F_1)$
and 
$D_2 = (Q_2, \Sigma, \delta_2, q_{0, 2}, F_2)$,
with  $D_1$  trim,
prove that the following properties hold.
\begin{enumerate}
\item[(1)]
There is a DFA morphism $\mapdef{h}{D_1}{D_2}$
iff 
\[\simeq_{D_1}\>\subseteq\>\simeq_{D_2}.\]
\item[(2)]
There is an  $F$-map
$\mapdef{h}{D_1}{D_2}$ iff
\[\simeq_{D_1}\>\subseteq\> \simeq_{D_2}
\quad\hbox{and}\quad
L(D_1) \subseteq  L(D_2);\]
\item[(3)]
There is a $B$-map
$\mapdef{h}{D_1}{D_2}$ iff
\[\simeq_{D_1}\>\subseteq\> \simeq_{D_2}
\quad\hbox{and}\quad
L(D_2) \subseteq  L(D_1).\]
\end{enumerate}

Conclude that if  $D_1, D_2$ are trim and
$L(D_1) = L(D_2)$, then
there is a unique surjective proper homomorphism
$\mapdef{h}{D_1}{D_2}$ iff
$$\simeq_{D_1}\>\subseteq\> \simeq_{D_2}.$$
(you may use results from  HW1).

\medskip
Prove that for any trim DFA $D$, there
is a unique surjective proper homomorphism
from $D$ to any minimal DFA
$D_m$ accepting $L = L(D)$

\medskip
(d) 
Given a regular language $L$,
prove that a minimal DFA $D_m$ for $L$ is 
characterized by the property that there is unique surjective
proper homomorphism
$\mapdef{h}{D}{D_m}$ from any trim DFA $D$ accepting $L$ to $D_m$.




\vspace{0.25cm}\noindent
{\bf Problem B3 (70 pts).} 
Let $L$ be any regular language
over some alphabet $\Sigma$.  Define the languages
\begin{eqnarray*}
L^{\infty} & = & \bigcup_{k\geq 1} \{w^k \mid w \in  L\}, \\
L^{1/\infty} & = & \{w \mid  w^k \in  L,\quad \hbox{for all $k\geq 1$}\},
\quad\hbox{and} \\
\sqrt{L} & = &\{w \mid  w^k \in  L,\quad \hbox{for some $k\geq 1$}\}.
\end{eqnarray*}
Also, for any natural number $k \geq 1$, let
\[L^{(k)} = \{w^k \mid  w \in  L\},\] 
and
\[L^{(1/k)} = \{w \mid  w^k \in  L\}.\] 

\medskip
(a)
Prove that $L^{(1/3)}$ is regular. What about $L^{(3)}$?

\medskip
(b)
Let $k\geq 1$ be any natural number.
Prove that there are only finitely many languages
of the form  $L^{(1/k)} = \{w \mid  w^k \in  L\}$
and that they are all regular.
(In fact, if $L$ is accepted by a DFA with $n$ states, there
are at most $2^{n^n}$ languages of the form $L^{(1/k)}$).

\medskip
(c) Is $L^{1/\infty}$ regular or not?
Is $\sqrt{L}$ regular or not?
What about $L^{\infty}$?



\vspace {0.25cm}\noindent
{\bf Problem B4 (60 pts).} 
Which of the following languages are regular? Justify
each answer.

\medskip
(a) $L_{1}=\{wcw \mid w\in \{a,b\}^{*}\}$

\medskip
(b) $L_{2}=\{xy \mid  x,y\in \{a,b\}^{*}\ \hbox{and}\  |x| = |y|\}$

\medskip
(c) $L_{3}=\{a^{n} \mid n\  \hbox{is a prime number}\}$

\medskip
(d) $L_{4} = \{a^mb^n \mid gcd(m, n) = 17\}$.



\vspace {0.25cm}\noindent
{\bf Problem B5 (50 pts).} 
(a) Prove again that the intersection,
$L_1\cap L_2$, of two regular languages, $L_1$ and $L_2$,
is regular, {\bf using the Myhill-Nerode characterization\/}
of regular languages.

\medskip
(b)
Let  $\mapdef{h}{\Sigma^*}{\Delta^*}$ be a homomorhism,
as defined on pages 24-26 of the slides on
DFA's and NFA's.
For any regular language, $L' \subseteq \Delta^*$,
prove that $h^{-1}(L')$ is regular,
{\bf using the Myhill-Nerode characterization\/}
of regular languages. Prove that the number of states
of any minimal DFA for $h^{-1}(L')$ is at most 
the number of states
of any minimal DFA for $L'$. Can it be strictly smaller?




\vspace{0.25cm}\noindent
{\bf Problem B6 (60 pts).} 
The purpose of this problem is to get a fast
algorithm for testing state equivalence in a DFA.
Let $D=(Q,\Sigma,\delta,q_{0},F)$ be a deterministic finite
automaton. Recall that {\it state equivalence\/} is the equivalence
relation $\equiv$ on $Q$, defined  such that,
$$p \equiv q\quad \hbox{iff}\quad \forall z\in\Sigma^* 
(\delta^{*}(p,z) \in F\quad \hbox{iff}\quad \delta^{*}(q,z) \in F),$$
and that {\it $i$-equivalence\/} is the equivalence
relation $\equiv_{i}$ on $Q$, defined such that,
$$p \equiv_{i} q\quad \hbox{iff}\quad \forall z\in\Sigma^*,\ |z| \leq i\ 
(\delta^{*}(p,z) \in F\quad \hbox{iff}\quad \delta^{*}(q,z) \in F).$$

\medskip
A relation $S\subseteq Q\times Q$ is a {\it forward closure\/} iff 
it is an equivalence relation and
whenever $(p, q)\in S$, then $(\delta(p, a), \delta(q, a))\in S$,
for all $a\in\Sigma$.

\medskip
We say that a forward closure $S$ is {\it good\/} iff 
whenever $(p, q)\in S$, then $good(p, q)$, where 
$good(p, q)$ holds iff either both $p, q\in F$, or
both $p, q\notin F$.

\medskip
Given any relation $R\subseteq Q\times Q$,
recall that the smallest equivalence relation $R_\approx$ containing
$R$ is the relation $(R \cup R^{-1})^*$
(where $R^{-1} = \{(q, p)\ \mid (p, q)\in R\}$, and 
$(R \cup R^{-1})^*$ is the reflexive and transitive closure of 
$(R \cup R^{-1})$).
We define the sequence of relations $R_i\subseteq Q\times Q$
as follows:

\begin{align*}
R_0 &= R_{\approx}\\
R_{i + 1} &= (R_{i} \cup 
\{(\delta(p, a), \delta(q, a)) \mid (p, q)\in R_i,\ a\in \Sigma\})_{\approx}.
\end{align*}

\medskip
(i) Prove that $R_{i_{0}+1} = R_{i_{0}}$ for some least $i_0$.
Prove that $R_{i_{0}}$ is the smallest forward closure
containing $R$.
 
\medskip
We denote the smallest forward closure $R_{i_{0}}$ containing $R$ as
$R^{\dagger}$, and call it the
{\it forward closure of $R$\/}.

\medskip
(ii) Prove that $p\equiv q$ iff the forward closure $R^{\dagger}$ of the
relation $R = \{(p, q)\}$ is good.



\vspace{0.5cm}\noindent
{\bf TOTAL: 380 points.}

\end{document}



