\documentclass[12pt]{article}
\usepackage{amsfonts}
\usepackage{amsmath}

\setlength{\topmargin}{-.5in}
\setlength{\oddsidemargin}{0 in}
\setlength{\evensidemargin}{0 in}
\setlength{\textwidth}{6.5truein}
\setlength{\textheight}{8.5truein}
\input ../libraries/mac.tex
\input ../libraries/mathmac.tex

\def\fseq#1#2{(#1_{#2})_{#2\geq 1}}
\def\fsseq#1#2#3{(#1_{#3(#2)})_{#2\geq 1}}
\def\qleq{\sqsubseteq}

%
\begin{document}
\begin{center}
\fbox{{\Large\bf Spring, 2014 \hspace*{0.4cm} CIS 511}}\\
\vspace{1cm}
{\Large\bf Introduction to the Theory of Computation\\
Jean Gallier \\
\vspace{0.5cm}
Homework 4}\\[10pt]
March 6, 2014; Due March 27, 2014\\
\end{center}

\vspace {0.25cm}
``A problems'' are for practice only, and should not
be turned in.

\vspace {0.25cm}
\noindent
{\bf Problem A1.} 
Given any two context-free languages
$L_1$ and $L_2$ over the same alphabet $\Sigma$, prove that
$L_1\cup L_2$ and $L_1L_2$ are also context-free.

\vspace {0.25cm}
\noindent
{\bf Problem A2.} 
Let $\Sigma$ and $\Delta$ be some alphabets, and
let $\mapdef{h}{\Sigma^*}{\Delta^*}$ be a homomorphism.
Given any language $L\subseteq \Sigma^*$,
recall that
\[h(L) = \{h(w)\in \Delta^* \ \mid \ w\in L\}.\]
Prove that if $L$ is context-free, then $h(L)$ is also
context-free.

\vspace {0.25cm}
\noindent
{\bf Problem A3.} 
Given any language $L\subseteq \Sigma^*$, let
\[L^R = \{w^R \mid w \in L\},\]
the {\it reversal language of $L$\/} (where $w^R$
denotes the reversal of the string $w$).
Prove that if $L$ is context-free, then $L^R$
is also context-free.


\vspace {0.5cm}
``B problems'' must be turned in.

% cis511-10hw4
\vspace{0.25cm}\noindent
{\bf Problem B1 (80 pts).} 
(i) Prove that the conclusion of the pumping lemma holds for the following
language $L$ over $\{a, b\}^*$,
and yet, $L$ is {\bf not\/} regular!
\[L = \{w \ \mid \ \exists n \geq 1,\,
\exists x_i\in a^+,\,\exists y_i\in b^+,\, 1\leq i \leq n,\,
\hbox{$n$ is not prime},\,
w = x_1y_1 \cdots x_ny_n\}.\]

\medskip
(ii)
Consider the following version of the pumping lemma.
For any regular language $L$, there is some $m \geq 1$ so that
for every $y\in \Sigma^*$, if $|y| = m$, then there exist
$u, x, v\in \Sigma^*$ so that
\begin{enumerate}
\item[(1)]
$y = uxv$;
\item[(2)]
$x\not= \epsilon$;
\item[(3)]
For all $z\in \Sigma^*$,
\[yz \in L
\quad\hbox{iff}\quad
ux^ivz\in L\]
for all $i\geq 0$.
\end{enumerate}

Prove that this pumping lemma holds.

\medskip
(iii)
Prove that the converse of the pumping lemma in (ii) also holds,
i.e., if a language $L$ satisfies the pumping lemma in (ii),
then it is regular.

\medskip
(iv)
Consider yet another version of the pumping lemma.
For any regular language $L$, there is some $m \geq 1$ so that
for every $y\in \Sigma^*$, if $|y| \geq m$, then there exist
$u, x, v\in \Sigma^*$ so that
\begin{enumerate}
\item[(1)]
$y = uxv$;
\item[(2)]
$x\not= \epsilon$;
\item[(3)]
For all $\alpha, \beta\in \Sigma^*$,
\[\alpha u\beta \in L
\quad\hbox{iff}\quad
\alpha ux^i \beta\in L\]
for all $i\geq 0$.
\end{enumerate}

Prove that this pumping lemma holds.

\medskip
(v)
Prove that the converse of the pumping lemma in (iv) also holds,
i.e., if a language $L$ satisfies the pumping lemma in (iv),
then it is regular.




\vspace {0.25cm}\noindent
{\bf Problem B2 (80 pts).}
This problem is based on the method
proved correct in Problem B6 of Homework $3$.
Also, consult Section 2.6 of the notes.

\medskip
Given a DFA $D = (Q, \Sigma, \delta, q_0, F)$,
for any two states $p, q\in Q$,
a fast algorithm for computing the forward closure of the
relation $R = \{(p, q)\}$, or detecting a bad pair of states,
can be obtained as follows:  An equivalence relation on $Q$ is
represented by a partition $\Pi$. Each equivalence class
$C$ in the partition is represented by a tree structure
consisting of nodes and (parent) pointers, with the
pointers from the sons of a node
to the node itself. The root has a null pointer.
Each node also maintains a counter keeping track
of the number of nodes in the subtree rooted at that node.

\medskip
Two functions $union$ and $find$ are defined as follows.
Given a state $p$, $find(p,\Pi)$ finds the root
of the tree containing $p$ as a node (not necessarily a leaf).
Given two root nodes $p, q$, $union(p, q, \Pi)$ forms
a new partition by merging the two trees with roots $p$ and $q$
as follows: if the counter of $p$ is smaller than
that of $q$, then let the root of $p$ point to $q$,
else let the root of $q$ point to $p$.

\medskip
In order to speed up the algorithm, using an idea due to  Tarjan, 
we can modify $find$ as follows:
during a call $find(p,\Pi)$, as we follow the path from $p$ to
the root $r$ of the tree containing $p$, we redirect the parent pointer
of every node $q$ on the path from $p$ (including $p$ itself) to
$r$.

\medskip
Say that a pair $\lag p, q\rag$ is {\it bad\/} iff either
both $p\in F$ and $q\notin F$, or both $p\notin F$ and $q \in F$.
The function $bad$ is such that $bad(\lag p, q\rag) = true$ if 
$\lag p, q\rag$ is bad, and $bad(\lag p, q\rag) = false$
otherwise.

\medskip
For details of this implementation of partitions, see
{\sl Fundamentals of data structures}, by Horowitz and Sahni,
Computer Science press, pp. 248-256.

\medskip
Then, the algorithm is as follows:

\vfill\eject\noindent

\medskip
{\bf function} $unif[p, q, \Pi, dd]$: $flag$;

\smallskip
\quad 
{\bf begin}

\smallskip
\quad \quad
$trans := left(dd)$; $ff := right(dd)$;  $pq := (p, q)$; 
$st := (pq)$; $flag := 1$;

\smallskip
\quad\quad
 $k:= Length(first(trans))$;

\smallskip
\quad\quad
 {\bf while $st \not= () \land  flag \not= 0$ do}

\smallskip
\quad\quad\quad
       $uv := top(st)$; $uu := left(uv)$; $vv := right(uv)$;

\smallskip
\quad\quad\quad
       $pop(st)$;

\smallskip
\quad\quad\quad
       {\bf if $bad(ff, uv) = 1$ then  $flag := 0$}

\smallskip
\quad\quad\quad
       {\bf else} 

\smallskip
\quad\quad\quad\quad
         $u := find(uu, \Pi)$; $v := find(vv, \Pi)$;

\smallskip
\quad\quad\quad\quad
          {\bf if $u \not= v$ then}

\smallskip
\quad\quad\quad\quad\quad
                 $union(u,v,\Pi)$;

\smallskip
\quad\quad\quad\quad\quad
                {\bf for $i = 1$ to $k$ do}

\smallskip
\quad\quad\quad\quad\quad\quad
      $u1 := delta(trans,uu,k-i+1)$; $v1 := delta(trans,vv,k-i+1)$;

\smallskip
\quad\quad\quad\quad\quad\quad
      $uv := (u1, v1)$; $push(st,uv)$

\smallskip
\quad\quad\quad\quad\quad
                {\bf endfor}

\smallskip
\quad\quad\quad\quad
          {\bf endif}

\smallskip
\quad\quad\quad
     {\bf endif}

\smallskip
\quad\quad
  {\bf endwhile}

\smallskip
\quad
{\bf end}

\medskip
The initial partition $\Pi$ is the identity relation
on $Q$, i.e., it consists of blocks $\{q\}$
for all state $q\in Q$.
The algorithm uses a stack $st$.
We are assuming that the DFA $dd$ is specified
by a list of two sublists, the first list,
denoted $left(dd)$ in the pseudo-code above, being
a representation of the transition function, and the second
one, denoted $right(dd)$,  
the set of final states. The transition function itself
is a list of lists, where the $i$-th list represents the
$i$-th row of the transition table for $dd$.
The function $delta$ is such that $delta(trans,i,j)$ returns
the $j$-th state in the $i$-th row of the transition table of $dd$.
For example, we have a DFA
$$dd = (((2, 3), (2, 4), (2, 3), (2, 5), (2, 3), 
        (7, 6), (7, 8), (7, 9), (7, 6)), (5, 9))$$
consisting of $9$ states labeled $1, \ldots, 9$,
and two final states $5$ and $9$. Also, the alphabet
has two letters, since every row in the transition
table consists of two entries. For example, the two transitions
from state $3$ are given by the pair $(2, 3)$,
which indicates that $\delta(3, a) = 2$ and $\delta(3, b) = 3$. 

\medskip
Implement the above algorithm, and test it at least for the above DFA $dd$
and the pairs of states $(1, 6)$ and $(1, 7)$.
Pay particular attention to the input and output format.
In particular, ouput the current partition at every round through the
{\bf while} loop.
Explain your data structures.

\medskip
{\it 
Please, consult the instructions posted on the web page for
CIS511, Homework section, for instructions on the
format for the input and output for this computer program.
}

\vspace {0.25cm}\noindent
{\bf Extra Credit\/} (up to {\bf 120 pts).\/}
Implement your program in such a way that
it displays the simultaneous parallel forward moves
in the DFA and the updating of the trees
representing the blocks of the partition.  There are programming
languages, such as {\tt Mathematica}, that have primitives to
manipulate and output trees.

% cis511-10hw4
\vspace{0.25cm}\noindent
{\bf Problem B3 (50 pts).} 
Prove that the language
\[
L = \{a^{4n + 3} \mid 4n + 3 \> \mathrm{is\ prime}\}
\]
is not regular.

\medskip
\hint
First, you will have to prove that there are infinitely many primes
of the form $4n + 3$. The list of such primes begins with
\[
3, \> 7, \> 11, \> 19, \> 23,\> 31, \> 43 \cdots
\]
Say we already have  $n + 1$ of these primes, denoted by
\[
3,\> p_1, \> p_2, \cdots, p_n,
\]
where $p_i > 3$. Consider the number
\[
m = 4p_1p_2 \cdots p_n + 3.
\]
If $m = q_1 \cdots q_k$ is a prime factorization of $m$, prove that
$q_j > 3$ for $j = 1, \ldots k$ and that no $q_j$ is equal to any of the 
$p_i$'s. Prove that one of the $q_j$'s must be of the form
$4n + 3$, which shows that there is a prime of the form $4n + 3$
greater than any of the previous primes of the same form.

% cis511-10hw4
\vspace{0.25cm}\noindent
{\bf Problem B4 (60 pts).} 
Let $D = (Q, \Sigma, \delta, q_0, F)$ be a {\it trim\/} DFA.
Consider the following procedure:
\begin{enumerate}
\item[(1)]
Form an NFA, $N^R$,  by reversing all the transitions
of $D$, i.e., there is a transition from $p$ to $q$ on
input $a\in \Sigma$ in $N$ iff $\delta(q, a) = p$ in $D$.
\item[(2)]
Apply the subset construction to the NFA, $N^R$, obtained in (1),
taking the start state to be the set $F$. The final states of the
DFA obtained by applying the subset construction to $N^R$ are all the
subsets containing $q_0$. 
Then, trim the resulting DFA,
to obtain the DFA $D^R$. 
\end{enumerate}

Observe that $L(D^R) = L(D)^R$.

\medskip
Now, apply the above procedure to $D$, getting $D^R$, and
apply this procedure again, to get $D^{RR}$.
Prove that $D^{RR}$ is a minimal DFA for $L = L(D)$.

\medskip\noindent
{\it Hint\/}.
First prove that if $\delta_R$ is the transition function
of $D^R$, then for every $w\in \Sigma^*$ and for every
state, $T\subseteq Q$, of $D^R$,
\[
\delta_R^*(T, w) =  \{q\in Q \mid \delta^*(q, w^R) \in T\}.
\]

% cis511-10hw2.tex
\vspace{0.25cm}\noindent
{\bf Problem B5 (60 pts).} 
An {\it $a$-transducer\/} (or {\it nondeterministic
sequential transducer with accepting states\/})
is a sextuple 
$M = (K, \Sigma, \Delta, \lambda, q_0, F)$,
where $K$ is a finite set of states, $\Sigma$ is a
finite input alphabet, $\Delta$ is a finite output alphabet,
$q_0\in K$ is the start (or initial) state, $F\subseteq K$ is the
set of accepting (of final) states, and
$$\lambda \subseteq K\times \Sigma^*\times \Delta^*\times K$$
is a finite set of quadruples called the {\it transition function\/} 
of $M$.

\medskip
An $a$-transducer defines a binary relation between
$\Sigma^*$ and $\Delta^*$, or equivalently, a function
$\mapdef{M}{\Sigma^*}{2^{\Delta^{*}}}$.
We can explain what this function is by describing
how an $a$-transducer makes a sequence of moves from
configurations to configurations.
The current configuration of an $a$-transducer 
is described by
a triple $(p, u, v)\in K\times \Sigma^*\times \Delta^*$,
where $p$ is the current state, $u$ is the remaining
input, and $v$ is some ouput produced so far.
We define the binary relation $\vdash_M$ on
$K\times \Sigma^*\times \Delta^*$ as follows:
For all $p, q\in K$, $u, \alpha\in\Sigma^*$,
$\beta, v\in \Delta^*$, if
$(p, u, v, q)\in \lambda$, then
$$(p,\, u\alpha,\, \beta) \vdash_M (q,\, \alpha,\, \beta v).$$
Let $\vdash_M^*$ be the transitive and reflexive closure
of $\vdash_M$. 

\medskip
The function
$\mapdef{M}{\Sigma^*}{2^{\Delta^{*}}}$ is defined such that
for every $w\in \Sigma^*$,
$$M(w) = \{y\in \Delta^*\ |\ (q_0,\, w,\, \epsilon)
\vdash_M^* (f,\, \epsilon,\, y),\ f\in F\}.$$
For every language $L \subseteq \Sigma^*$, let
$$M(L) = \bigcup_{w\in L} M(w).$$

\medskip
(a)
Let $\Sigma = \Delta = \{a, b\}$.
Construct an $a$-transducer swapping $a$'s and $b$'s
(for instance, if $w = abbaa$, then $y = baabb$).

\medskip
(b)
Given an $a$-transducer 
$M = (K, \Sigma, \Delta, \lambda, q_0, F)$,
define the new alphabet $T$ as follows:
$$T = \{[p, u, v, q]\ |\ (p, u, v, q)\in \lambda\}.$$
Let $\mapdef{f}{T^*}{\Sigma^*}$ and
$\mapdef{g}{T^*}{\Delta^*}$  be the homomorphisms defined
such that
$$f([p, u, v, q]) = u,\quad\hbox{and}\quad
g([p, u, v, q]) = v.$$

\medskip
Prove that the language
\begin{align*}
R & = \{[q_0, u_1, v_1, q_1][q_1, u_2, v_2, q_2]\cdots 
[q_{n-2}, u_{n-1}, v_{n-1}, q_{n-1}][q_{n-1}, u_n, v_n, q_n]\\
& \qquad \mid  [q_{i-1}, u_i, v_i, q_i]\in T,\, 1\leq i\leq n,\,
q_n\in F,\, n\geq 1\}\cup \{\epsilon\ |\ q_0\in F\}\
\end{align*}
is a regular language.

\medskip
(c) Prove that
\begin{align*}
f^{-1}(L)\cap R & = 
\{[q_0, u_1, v_1, q_1][q_1, u_2, v_2, q_2]\cdots 
[q_{n-2}, u_{n-1}, v_{n-1}, q_{n-1}][q_{n-1}, u_n, v_n, q_n]\\
& \qquad \mid  [q_{i-1}, u_i, v_i, q_i]\in T,\, u_1u_2\cdots u_n\in L,\,
q_n\in F,\, n\geq 1\}\cup \{\epsilon\ |\ q_0\in F,\, \epsilon\in L\}.
\end{align*}

\medskip
(d) Prove that
$$M(L) = g(f^{-1}(L)\cap R).$$

\medskip
If $\s{L}$ is a family of languages closed under
intersection with regular languages, homomorphic images,
and inverse homomorphic images, is $\s{L}$ closed
under $a$-transductions? (Justify your answer).

\medskip
If $L$ is a regular language, is $M(L)$ regular?
(Justify your answer).

\medskip
(e)
If $M$ is an $a$-transducer from $\Sigma^*$ to $\Delta^*$ prove that
for any regular language, $L'\subseteq \Delta^*$, the language
$M^{-1}(L')$ is also regular (see the definition of $M^{-1}(L')$
in the class slides).

\vspace{0.25cm}\noindent
{\bf Problem B6 (40 pts).} 
Consider the language
\[
L = \{w\in \{a b\}^* \mid \hbox{$w$ has an odd number of $a$'s or an
  odd number of $b$'s}\}.
\]

\medskip
(1)
Give a DFA for $L$, with four states.

\medskip
(2)
Use node-elimination  to obtain a regular expression denoting $L$.


% cis51104hw4.tex
\vspace{0.25cm}\noindent
{\bf Problem B7 (60 pts).} 
Give context-free grammars for the following languages:

\medskip
(a)
$L_{5} = \{wcw^{R}\mid w \in \{a,b\}^{*}\}$\ 
($w^{R}$ denotes the reversal of $w$)

\medskip
(b)
$L_{6} = \{a^{m}b^{n}\mid 1 \leq m \leq n \leq 2m\}$

\medskip
For any fixed integer $K\geq 2$, 

\medskip
$L_{7} = \{a^{m}b^{n}\mid 1 \leq m \leq n \leq Km\}$

\medskip
(c)
$L_{8} = \{a^{n}b^{n}\mid n \geq 1\}\cup \{a^{n}b^{2n}\mid n \geq 1\}$

\medskip
(d)
$L_{9} = \{a^{m}b^{n}a^{m}b^{p}\mid m, n, p \geq 1\}
\cup \{a^{m}b^{4n}a^{p}b^{4n}\mid m, n, p \geq 1\}$

\medskip
(e)
$L_{10} = \{xcy \mid |x| = 2|y|,\, x, y\in \{a, b\}^*\}$

\medskip
In each case, give a justification of the fact that
your grammar generates the desired language.

\vspace{0.5cm}\noindent
{\bf TOTAL:  430 $+$   120 points.}

\end{document}



