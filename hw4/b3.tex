\documentclass[12pt]{article}
\usepackage{fullpage}
\usepackage{titlesec}
\usepackage{tikz}
\usepackage{amsfonts,amssymb}
\usepackage{amsmath}
\usepackage{comment}
\usetikzlibrary{automata, positioning}

\input ../libraries/mac.tex
\input ../libraries/mathmac.tex

\begin{document}
\pagestyle{plain}
\titleformat{\subsection}[runin]
  {\normalfont\large\bfseries}{\thesubsection}{1em}{}

\title{Homework 4}
\author{Brooke Fugate, Michael O'Connor, Rohan Shah}
\date{}

\maketitle

\section*{Problem B3}
\textbf{Claim:} There are infinitely many primes of the form $4n+3$.
\newline
\textbf{Proof:} Say we already have $n+1$ of these primes, denoted by
$$3, p_1, p_2, \cdots, p_n,$$
where $p_i > 3$. Consider the number
$$m = 4p_1p_2 \cdots p_n + 3$$
and its prime factorization
$$m=q_1 \cdots q_k.$$
First we can prove that $q_j > 3$ for $j = 1,...,k$. Since any number of the
form $4n+3$ is odd $2$ can not be a prime factor of it therefore $q_j \neq 2$.
We can finish the proof by
contradiction, so assume there exists some $q_j = 3$. Since $q_j$ is a prime
factor of $m$ we know that
$$m\ \mod\ q_j = 0 \implies (4p_1p_2 \cdots p_n + 3)\ \mod\ 3 = 0
\implies (4p_1p_2 \cdots p_n)\ \mod\ 3 = 0$$
which is a contradiction because
$$4\ \mod\ 3 \neq 0,p_1\ \mod\ 3 \neq 0,...,p_n\ \mod\ 3 \neq 0.$$
Therefore $q_j > 3$. Now we can prove that for all $p_i$ and $ q_j$,
$p_i \neq q_j$ by contradiction so assume there exists some $p_i$ and $q_j$ such
that $p_i = q_j$. Since $q_j$ is a prime factor of m and $q_j = p_i$ it must be
true that
$$(4p_1p_2 \cdots p_n + 3)\ \mod\ p_i = 0.$$
And it is true that
$$(4p_1p_2 \cdots p_n)\ \mod\ p_i = 0.$$
Therefore
$$(4p_1p_2 \cdots p_n + 3)\ \mod\ p_i = 0 \iff 3\ \mod\ p_i = 0$$
which is equivalent to
$$3\ \mod\ q_j = 0$$
which is a contradiction since $q_j > 3$, therefore $p_i \neq q_j$ for all $p_i$
and $q_j$. Finally we can prove that there exists a $q_j$ of the form $4n+3$.
Observe that all odd numbers are either of the form $4n+1$ or $4n+3$. Also
observe that given two numbers of the form $4n+1$, say $x = 4m+1$ and
$y = 4n+1$ then $xy = 4(4mn + m + n) + 1$ which is again of the form $4n+1$.
We can proceed by contradiction so assume there are no $q_j$ of the form
$4n+3$ therefore all $q_j$ must be of the form $4n+1$ which implies that
$q_1q_2\cdots q_n$ is of the form $4n+1$ which is a contradiction since $m$ is
of the form $4n+3$ therefore there must exists a $q_j$ of the form $4n+3$.
Since, there exists a $q_j$ of the form $4n+3$ that is greater than $3$ and not
equal to any of the first $(n+1)$ $p_i$ of the form $4n+3$ and since $q_j$ is a
prime factor of $m$, there exists a prime number of the form $4n+3$ greater than
all previous primes of the form $4n+3$ so by induction there are an infinite
number of primes of the form $4n+3$.
\end{document}
