\documentclass[12pt]{article}
\usepackage{fullpage}
\usepackage{titlesec}
\usepackage{tikz}
\usepackage{amsfonts,amssymb}
\usepackage{amsmath}
\usepackage{comment}
\usetikzlibrary{automata, positioning}

\input ../libraries/mac.tex
\input ../libraries/mathmac.tex

\begin{document}
\pagestyle{plain}
\titleformat{\subsection}[runin]
  {\normalfont\large\bfseries}{\thesubsection}{1em}{}

\title{Homework 4}
\author{Brooke Fugate, Michael O'Connor, Rohan Shah}
\date{}

\maketitle

\section*{Problem B1}
\subsection*{i}
We can prove the language $L$ is not regular by contradiction, so assume that
$L$ is regular. Thus, its complement $\overline{L}$ must also be regular.
Further, given the trivially regular language $X = \{a^+b^+\}^+$, the language
$L' = \overline{L} \cap X$ must also be regular, where
$L' = \{(a^+b^+)^n\ |\ n \text{ is prime}\}$. But we have previously
shown that a language consisting of exactly a prime number of concatenation
of words in $\Sigma^*$ is not regular which is a contradiction thus, $L$ is not
regular.
\medskip
\newline
To prove that the pumping lemma holds for the language $L$ we must show that
there exists some $m \ge 1$ such that for all $w \in \Sigma^*$, if $w \in L$
and $|w| \ge m$, there exists $u,x,v \in \Sigma^*$ such that $w = uxv$,
$x \neq \epsilon$, $|ux| \le m$, and $ux^iv \in L$ for all $i \ge 0$. Let
$m = 9$ and consider a $w \in L$ where $|w| \ge m$. Then there exists a
prefix of $w$ of length $m$, i.e. there exists some $w_p,w_s \in \Sigma^*$
such that $w = w_pw_s$ and $|w_p| = m$. If we do case analysis on
$w_p$ we can see that it can take one of two different forms. The first is the
case where there exists a substring $s \in \Sigma^*$ of $w_p$ such that
$s = a^i$ or $s = b^i$ where $i \ge 2$. The second case is the alternate where
there does not exist such a substring $s$ in $w_p$, i.e. $w_p$ consists solely
of substrings of the form $ab$. Both these cases are due to the fact that
$w \in L$. We will show that the pumping lemma holds in both these cases.
\medskip
\newline
Consider the first case. Let $w_p = uxv'$ such that $u$ is the prefix of
$w_p$ up till the occurence of the first substring of the form $a^i$ or $b^i$
where $i \ge 2$. Then let $x = a$ or $x = b$ respectively. Finally let $v$ be
the remaining suffix of our original word $w$ (by our previous definitions this
means $v = (a | b)^jv'w_s)$ but this is not important to the proof).
By definition $w = uxv$ and $x \neq \epsilon$. Further $|ux| \le m$ since both
$ux$ is a prefix of $w_p$ which we defined such that $|w_p| = m$. Finally since
$x$ is the single character $a$ or $b$ of a substring of $w$ where there are
more than one $a$'s or $b$'s respecitvely, pumping down once or pumping up
infinitely does not change the value of $n$ in the definition of $L$ meaning
that it remains non prime, nor does it change the form of $w$ meaning $w$
maintains the form $w = x_1y_1 \cdots x_ny_n$, therefore $ux^iv \in L$ for all
$i \ge 0$ and thus the pumping lemma holds for words $w$ in this case.
\medskip
\newline
Now consider the second case. We know that $w = ababababaz$ for some
$z \in \Sigma^*$ and thus that $w_p = ababababa$. Since all we know about $n$ is
that it's not prime, it could be either odd or even. So consider first when $n$
is odd. Let $u = a$, $x = b$, and $v$ be the rest of $w$ i.e. $v = abababaz$. By
construction $w = uxv$, $x \neq \epsilon$, and $|ux| = 2 \le m = 9$. Pumping
down, meaning removing $x$, maintains the form of $w$ but decreases $n$ by 1.
But since $n$ is odd this makes $n$ even which is not prime. Pumping up does
not change $n$ and thus $n$ remains non prime and of the required form
thus the pumping lemma holds when $n$ is odd. Now consider the case when $n$ is
even. Let $u = \epsilon$, $x = abab$, and $v$ be the rest of $w$ i.e.
$v = ababaz$. By construction $w = uxv$, $x \neq \epsilon$, and
$|ux| = 4 \le m = 9$. Pumping down, meaning removing $x$, does not change the
form of $w$ but does decrease $n$ by 2. But since $n$ is even, $n$ remains even
and thus $n$ is not prime. Pumping up also lets $n$ remain even and of the
required form. Thus, the pumping lemma also holds when $n$ is even.
\medskip
\newline
Therefore the pumping lemma holds in all cases of $w \in L$ where $|w| \ge m = 9$,
thus the pumping lemma holds for the language $L$ even though $L$ is not regular.

\subsection*{ii}
To prove that this pumping lemma holds for any regular language $L$ we must show
that there exists some $m \ge 1$ such that for all $y \in \Sigma^*$,
if $y \in L$ and $|y| = m$, there exists $u,x,v \in \Sigma^*$ such that
$y = uxv$, $x \neq \epsilon$, and for all $z \in \Sigma^*$,
$yz \in L \iff ux^ivz \in L$ for all $i \ge 0$. Since $L$ is a regular language
it has a finite, positive, non-zero number of Myhill-Nerode equivalence classes.
Let $m$ be equal to this number. Now consider any word $y \in L$ where $|y| = m$.
There are $m+1$ prefixes of the word $y$ (including $\epsilon$ and $y$). Since
there are only $m$ equivalence classes for the language $L$ there must exists
two prefixes of $y$, namely $p_1,p_2 \in \Sigma^*$ such that $p_1 \simeq_D p_2$
and $|p_1| > |p_2|$. Therefore it follows that $p_2$ is also a prefix of $p_1$
so $p_1 = p_2x'$ for some $x' \in \Sigma^*$ where $|x'| \neq 0$ which implies
$x' \neq \epsilon$. Let $x=x'$, $u=p_2$, and $v$ be the suffix of $y$
such that $y=p_1v$. Thus we have $y = uxv$ and $x \neq \epsilon$ by construction.
We also have $u \simeq_D ux$ by substitution. Now we just need to prove that
$yz \in L \iff ux^ivz \in L$ for all $i \ge 0$.
Since $y = uxv$ this is equivalent to proving $uxvz \in L \iff ux_ivz \in L$.
But we can show something stronger than this, namely that $uxvz \simeq_D ux^ivz$
which directly implies that $uxvz \in L \iff ux^ivz \in L$ by the way in which
Myhill-Nerode string equivalence is defined. Further, we can prove
$uxvz \simeq_D ux^ivz$ for all $z \in \Sigma^*$ by proving $uxv \simeq_D ux^iv$
for all $i \ge 0$ and applying the right-invariance property of Myhill-Nerode
equivalence. Given $u \simeq ux$ from above and using right-invariance we have
$ux \simeq_D ux^2$. By transitivity and induction we have $ux \simeq_D ux^i$
for all $x \ge 0$. By right-invariance again we have $uxv \simeq_D ux^iv$ which
is what we wanted to prove therefore the pumping lemma holds for all regular
languages $L$.

\subsection*{iii}
Show $\rho_L$ has a finite number of y equivalence classes. If this is true, then L is regular. First, show that for any $w \in \Sigma^*$, if $|w| \ge m$, then $\exists w' \in \Sigma^*$ such that $|w'| < m$ and $w \rho_L w'$. We know that $y \rho_L ux^iv$ for all $i \ge 0$. Pick any w such that $|w| \ge m$. Write $w=yy_2$ such that $|y|=m$. We have $y=uxv$ with $x \neq \epsilon$. Then, $w'=uvy_1$ by (3) with $z_1=y_1$ we have $yz=yy_1$  

\subsection*{iv}
\subsection*{v}

\end{document}


















