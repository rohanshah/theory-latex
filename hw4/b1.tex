\documentclass[12pt]{article}
\usepackage{fullpage}
\usepackage{titlesec}
\usepackage{tikz}
\usepackage{amsfonts,amssymb}
\usepackage{amsmath}
\usepackage{comment}
\usetikzlibrary{automata, positioning}

\input ../libraries/mac.tex
\input ../libraries/mathmac.tex

\begin{document}
\pagestyle{plain}
\titleformat{\subsection}[runin]
  {\normalfont\large\bfseries}{\thesubsection}{1em}{}

\title{Homework 4}
\author{Brooke Fugate, Michael O'Connor, Rohan Shah}
\date{}

\maketitle

\section*{Problem B1}
First, prove that L is not regular in a proof by contradiction. Assume L is regular. 
If L is regular, then \={L} must be regular. Further, \={L} $\cap (ab)^+ = \{(ab)^n |$ n is prime number \}
is regular. In the previous homework assignment we proved that the language \{$a^n |$ n is a prime number \}
is not regular. It follows that \={L} $\cap (ab)^+ = \{(ab)^n |$ n is prime number \} must not be regular, 
and therefore L is not regular.

\medskip

Then, choose m to be 9. We will show that $m=9$ satisfies the requirements of the pumping lemma.
We can represent w (with $|w| \ge m$) as $w=a^{i_1}b^{j_1}a^{i_2}b^{j_2}a^{i_3}b^{j_3}a^{i_4}b^{j_4}z \in L$
and $z \neq \epsilon$. 
There are two cases to consider. The first case is if either of the first four blocks 
($a^{i_1}b^{j_1}a^{i_2}b^{j_2}a^{i_3}b^{j_3}a^{i_4}b^{j_4}$) has $i_k,j_k \ge 2$. In this case,
we can choose as our x one of a's where there are multiple or one of the b's where there are multiple.
If we take out this letter, the new word is still in the language, and if we add any number of this letter
the new word is also 
still in the language. The second case is such that $w=ababababz$. There are two subcases to 
consider. Either n is even or n is odd. If n is even, then $n \ge 6$. We can choose $x=abab$. 
So if we pump down, n is even and $n \ge 4$ and the the word is still in the language, 
and similarly if we pump up, n will be even, and the new word will still be in the language.
If n is odd, then $n \ge 9$. We can choose the first b that appears in w such that if we pump down,
n becomes even and the new word is still in the language. If we pump up, n remains constant 
and the new word is still in the language. So for all cases, $m=9$ satisfies the pumping lemma
for the language L.


\end{document}
