\documentclass[12pt]{article}
\usepackage{fullpage}
\usepackage{titlesec}
\usepackage{tikz}
\usepackage{amsfonts,amssymb}
\usepackage{amsmath}
\usepackage{comment}
\usetikzlibrary{automata, positioning}

\input ../libraries/mac.tex
\input ../libraries/mathmac.tex

\begin{document}
\pagestyle{plain}
\titleformat{\subsection}[runin]
  {\normalfont\large\bfseries}{\thesubsection}{1em}{}

\title{Homework 5}
\author{Brooke Fugate, Michael O'Connor, Rohan Shah}
\date{}

\maketitle

\section*{Problem B4}
\subsection*{i}
We may assume that L in infinite. We take $k$, the constant of the pumping lemma and consider $r = k!$. For all $w \in L$, if $|w| \ge k$, then $\{wa^{rn} | n \ge 0 \} \subseteq L$. \newline 
We can consider $w=uvxyz$ with $u=a^{i_1}$, $v=a^{i_2}$, $x=a^{i_3}$ , $y=a^{i_4}$ , and $z=a^{i_5}$. We know from the pumping lemma that $i_3 \ge 1$, $i_2 + i_4 \ge 1$ and $i_2 + i_3 + i_4 \le k$. Then we consider $uv^mxy^mz \in L$ for all $m \ge 0$.  So, $uv^mxy^mz = a^{i_1+i_3+i_5 + m(i_2 + i_4)}$. Let $k_1 = i_1 + i_3 + i_5$ and $k_2 = i_2 + i_4$. Then, $a^{k_1 + mk_2} \in L$ for all $m \ge 0$. It is clear that $1 \le k_2 \le k$, and $p=k_1+k_2$. Pick any $n \ge 0$. We can find an $m$ such that $mk_2=k_2 +nk$ by setting $m=1+ \dfrac{nr}{k_2}$. So, $k_1+mk_2=k_1+k_2+nr=p+nr$. This shows that $a^{k_1+mk_2}=a^{p+nr}=wa^{nr} \in L$, which is what we set out to show. 
\subsection*{ii}
We have $0 \le i < r$ and $L_i = \{a^n |a^n \in L, n \ge k , n \equiv i \bmod r \}$ and 
$$L = \{ a^n | a^n \in L , n < k\} \cup \bigcup_{0 \le i < r} L_i$$ where $\{ a^n | a^n \in L , n < k\}$ is finite. If $L_i \neq \phi$, then let $z_i$ be the shortest string in $L_i$ such that $(|z_i| \ge k)$. We claim that $L_i = \{z_i a^{rm} | m \ge 0\}$. By (i) we know that $\{z_i a^{rm} | m \ge 0\} \subseteq L_i$. To prove the reverse induction, pick any $a^n \in L_i$. Write $z_i = a^{n_i}$. We have $n \ge n_i$, $n_i = m_ir + i$ and $n=mr +i$. We also know that $m \ge m_i$ which shows that $n=(m-m_i)k + m_ik + i$, so $n=(m-m_i)k + n_i$. But $m-m_i \ge 0$, so $an=a^{n_i}a^{(m-m_i)r}=z_ia^{(m-m_i)r}$.  

\subsection*{iii}

We can show that $L_i=L(z_i(a^r)^*)$, so $L_i$ is regular. Let $L_i'=\{a^n|n \ge k, n \equiv i \bmod r\}$. We observe that $L_i=L_i' \cap L$.  $L_i'$ is regular, and a DFA can be constructed. By B3, since $L_i'$ is regular and L is context free, we can construct a grammar $G_i$ in $L_i=L_i' \cap L$. Use $T(G_i)$ to test and check if the start symbol is in it. 

\subsection*{iv}
Assume $L \subseteq \{a,b\}^*$ and $L$ is a context free language.  There may be an algorithm to decide $\{a\}^* \subseteq L$ iff $\{a\}^* \cap L=\{a\}^*$. Then, let $L_1=\{a\}^* \cap L$ and notice that $L$ is a context free language by B3 and $L_1 \subseteq \{a\}^*$. By (ii), $L$ is regular. Now we need to consider the DFA equivalence of $L_1$. 

\end{document}
