\documentclass[12pt]{article}
\usepackage{fullpage}
\usepackage{titlesec}
\usepackage{tikz}
\usepackage{amsfonts,amssymb}
\usepackage{amsmath}
\usepackage{comment}
\usetikzlibrary{automata, positioning}

\input ../libraries/mac.tex
\input ../libraries/mathmac.tex

\begin{document}
\pagestyle{plain}
\titleformat{\subsection}[runin]
  {\normalfont\large\bfseries}{\thesubsection}{1em}{}

\title{Homework 5}
\author{Brooke Fugate, Michael O'Connor, Rohan Shah}
\date{}

\maketitle

\section*{Problem B4}

\subsection*{(i)} Consider $r = k!$ where $k$ is the constant of the pumping
lemma for the context-free language $G$. We prove that for all $w \in L$,
if $|w|\ge k$, then $\{wa^{rn} | n \ge 0 \} \subseteq L$. Let $w=uvxyz$, where
$u=a^{i_1}$, $v=a^{i_2}$, $x=a^{i_3}$ , $y=a^{i_4}$ , and $z=a^{i_5}$.
From the pumping lemma we have that $i_3 \ge 1$, $i_2 + i_4 \ge 1$ and
$i_2 + i_3 + i_4 \le k$. Then consider $uv^mxy^mz \in L$ for all $m \ge 0$.
So, $uv^mxy^mz = a^{i_1+i_3+i_5 + m(i_2 + i_4)}$. Let $k_1 = i_1 + i_3 + i_5$
and $k_2 = i_2 + i_4$. Then, $a^{k_1 + mk_2} \in L$ for all $m \ge 0$.
We have that $1 \le k_2 \le k$ and $p=k_1+k_2$. For any $n \ge 0$ we can find
an $m$ such that $mk_2=k_2 +nk$ by letting $m=1+ \dfrac{nr}{k_2}$. So,
$k_1+mk_2=k_1+k_2+nr=p+nr$. This shows that
$a^{k_1+mk_2}=a^{p+nr}=wa^{nr} \in L$, which is what we wanted to prove.

\subsection*{(ii)}
For any $i$ such that $0 \le i < r$ let
$$L_i = \{a^n |a^n \in L, n \ge k , n \equiv i \bmod r \}$$
Then clearly,
$$L = \{ a^n | a^n \in L , n < k\} \cup \bigcup_{0 \le i < r} L_i$$
where $\{ a^n | a^n \in L , n < k\}$ is finite.
If $L_i \neq \emptyset$, then let $z_i$ be the shortest string in $L_i$
such that $|z_i| \ge k$. We claim that
$$L_i = \{z_i a^{rm} | m \ge 0\}$$
By part (i) we know that $\{z_i a^{rm} | m \ge 0\} \subseteq L_i$.
To prove the reverse induction, we can choose any $a^n \in L_i$.
Let $z_i = a^{n_i}$. Then we have $n \ge n_i$, $n_i = m_ir + i$ and $n=mr +i$.
We also know that $m \ge m_i$ which gives $n=(m-m_i)k + m_ik + i$.
So we have that $n=(m-m_i)k + n_i$. But $m-m_i \ge 0$,
so $a_n=a^{n_i}a^{(m-m_i)r}=z_ia^{(m-m_i)r}$.
Since each $L_i$ can be written as a regular expression of the form
$\{z_ia^r\}^*$ each $L_i$ is regular. Since $L$ is the union of a finite set of
$L_i$'s together with the union of another trivially regular language,
$L$ is regular.

\subsection*{(iii)}
We just showed that $L_i$ is regular. Since it is regular, it is also
context-free so we can construct a grammar $G_i$ for $L_i$ i.e. a grammar such
that $L(G_i) = L_i$. It is decidable whether or not the start symbol of the
grammar $G_i$ is in $T(G_i)$. If it isn't then it must be the case that
$L_i = \emptyset$. Therefore it is decidable whether $L_i = \emptyset$.

\subsection*{(iv)} Given that $L \subseteq \{a,b\}^*$ and $L$ is a context free
language, it is the case that $\{a\}^* \subseteq L$ if and only if
$\{a\}^* \cap L=\{a\}^*$. Note that the language $\{a\}^*$ is trivially regular.
Let $L_1=\{a\}^* \cap L$. Since we have previously proven that context-free
languages are closed under intersection with regular languages, $L_1$ is a
context free language. Further, it is clear that $L_1 \subseteq \{a\}^*$.
By part (ii) of this problem, $L_1$ is regular. Since $L_1$ and $\{a\}^*$ are
both regular we can create DFAs for both languages, Since it is decidable
whether two DFAs are equivalent (by comparing their minimal DFAs, where minimal
DFAs are unique for each regular language) and therefore whether two regular
languages are equivalent, it is decidable whether $\{a\}^* \subseteq L$
by testing whether $L_1$ and $\{a\}^*$ are equivalent.

\end{document}
