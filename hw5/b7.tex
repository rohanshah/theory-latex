\documentclass[12pt]{article}
\usepackage{fullpage}
\usepackage{titlesec}
\usepackage{tikz}
\usepackage{amsfonts,amssymb}
\usepackage{amsmath}
\usepackage{comment}
\usetikzlibrary{automata, positioning}

\input ../libraries/mac.tex
\input ../libraries/mathmac.tex

\begin{document}
\pagestyle{plain}
\titleformat{\subsection}[runin]
  {\normalfont\large\bfseries}{\thesubsection}{1em}{}

\title{Homework 5}
\author{Brooke Fugate, Michael O'Connor, Rohan Shah}
\date{}

\maketitle

\section*{Problem B7}

\subsection*{(i)} From the defintion of $\tau_1$ we have that:
$$\tau_1(b)= v_1bv_2 \text{ such that } v_1,v_2 \in (\bar\Omega \cup \{E\}^*)
\text{ for all } b\in \Delta$$
So we can extend $\tau_1$ for languages $L \subseteq \Delta^*$ such that
$\epsilon \not\in L$ to:
$$\tau_1(L) = \{v_1b_1v_2\cdots v_nb_nv_{n+1} \mid
v_j \in (\bar\Omega \cup \{E\}^*),\ b_1\cdots b_n \in L\}$$
And from the definition of $R$ we have that
$$ R = \{u_1\bar a_1h(a_1)\cdots u_n\bar a_nh(a_n)
\mid a\in \Omega,\ u_i\in \{E\}^*\}$$
So we can define $L_2 = \tau_1(L) \cap R$ as
$$L_2 = \{u_1\bar a_1h(a_1)\cdots u_n\bar a_nh(a_n)
\mid u_i \in \{E\}^*.\ h(a_1)\cdots h(a_n) \in L,\ a_i\in \Omega\}$$
Finally we can see that $L_2 \cap (\bar\Omega\cup\{h(a) \mid a\in \Omega\})^*$
contains any word from $L_2$ without any instances of $E$ i.e. that all
$u_i = \epsilon$ so we have
$$L_2\cap \left(\ptb{\Omega}\cup \{h(a) \mid a\in\Omega\}\right)^*
=\{\ptb{a_1}h(a_1) \ptb{a_2}h(a_2)\cdots \ptb{a_n}h(a_n)\ |\ 
h(a_1a_2\cdots a_n)\in L,\ a_i\in \Omega\}$$


\subsection*{(ii)} If $\epsilon \not\in L$ then we have from above that:
$$\tau_2(g(\tau_1(L)\cap R)) = \tau_2(g(L_2))$$
And we can extended the definition of $g$ to languages to give:
$$L_3 = g(L_2) = \{u_1a_1u_2a_2\cdots u_na_nu_{n+1} \mid u_i \in \{E\}^*
,\ h(a_1a_2\cdots a_n) \in L,\ a_i\in \Omega
,\ a_1a_2\cdots a_n \in h^{-1}(L)\}$$
where all the instances of $h(a_i)$ in strings in $L_2$ where replaced with
$\epsilon$ by $g$ and all instances of $\bar a_i$ where replaced by
instances of $a_i$ by $g$. By applying the extension of $\tau_2$ to languages
we get:
$$\tau_2(L_3) = \{v_1a_1v_2a_2\cdots v_na_nv_{n_1}
\mid h(a_1a_2\cdots a_n) \in L,\ v_1\in \Gamma^*\}$$
where
$$h(v_1a_1v_2a_2\cdots v_na_nv_{n+1}) = h(a_1a_2\cdots a_n)$$
since applying the homomorphism $h$ to a an element of $\Gamma$ is replaced by
$\epsilon$ from the definition of $\Gamma$.
So since
$$h(a_1a_2\cdots a_n) \in L$$
then
$$h(v_1a_1v_2a_2\cdots v_na_nv_{n+1}) \in L$$
which implies that
$$v_1a_1v_2a_2\cdots v_na_nv_{n+1} \in h^{-1}(L)$$
thus
$$\tau_2(L_3) = \tau_2(g(\tau_1(L)\cap R)) = h^{-1}(L) \text{ when }
\epsilon \not\in L$$
If $\epsilon \in L$, then it is also possible for words from $\Gamma^*$ to be in
$\tau_2(g(\tau_1(L)\cap R))$ since we need to condider the case when
$g(a) = \epsilon$ when $a\in \Delta$. So we can modify our above equality to be:
$$h^{-1}(L) = \tau_2(g(\tau_1(L-\{\epsilon\}) \cap R) \cup \Gamma^*
\text{ when } \epsilon\in L$$
Since $L\in \s{L}$ it is closed under substituion, intersection and union with
regular languages. Since homomorphisms are a special kind of regular substituion
$L$ is also closed under homomorphism. Therefore
$$\tau_2(g(\tau_1(L)\cap R)) \in \s{L} \text{ when } \epsilon \not\in L$$
and
$$\tau_2(g(\tau_1(L-\{\epsilon\}) \cap R) \cup \Gamma^* \in \s{L} \text{ when }
\epsilon\in L$$
which means
$$h^{-1}(L) \in \s{L}$$
since we proved its equality to those two languages.
Therefore $L$ is closed under inverse homomorphism.

\subsection*{(iii)} We proved in the previous problem that context-free languages
are closed under substituion by regular languages. Further, in problem 3, we
proved that context-free languages are closed under intersection with regular
languages. It is given that context-free languages are closed under union
with regular languages. Therefore, if $L$ is a context-free language,
$L \in \s{L}$. Since we just proved that $\s{L}$ is closed under inverse
homomorphism, then it is immediate that $L$ is closed under inverse
homomorphism. Therefore $h^{-1}(L)$ is context-free if $L$ is context-free.

\end{document}
