\documentclass[12pt]{article}
\usepackage{fullpage}
\usepackage{titlesec}
\usepackage{tikz}
\usepackage{amsfonts,amssymb}
\usepackage{amsmath}
\usepackage{comment}
\usetikzlibrary{automata, positioning}

\input ../libraries/mac.tex
\input ../libraries/mathmac.tex

\begin{document}
\pagestyle{plain}
\titleformat{\subsection}[runin]
  {\normalfont\large\bfseries}{\thesubsection}{1em}{}
\titleformat{\subsubsection}[runin]
  {\bfseries}{}{}{}

\title{Homework 5}
\author{Brooke Fugate, Michael O'Connor, Rohan Shah}
\date{}

\maketitle

\section*{Problem B1}
\subsection*{(i)} A context-free grammar for the language
$L_1 = \{xcy \mid x\neq y,\ x,y\in \{a,b\}^*\}$
\begin{center}
\parbox{7cm}{
$S \rightarrow S_1|S_2|S_3|S_4$ \newline
$S_1 \rightarrow aS_1a|bS_1b|aS_1b|bS_1a$ \newline
$S_1 \rightarrow aDc|bDc$ \newline
$D \rightarrow aD|bD| \epsilon$ \newline
$S_2 \rightarrow aS_2a|bS_2b|aS_2b|bS_2a$ \newline
$S_2 \rightarrow cEa|cEb$ \newline
$E \rightarrow Ea|Eb| \epsilon$ \newline
$S_3 \rightarrow aFabX|bFbbX|aFbbX|bFabX|FbX$ \newline
$X \rightarrow Xa|Xb|\epsilon$ \newline
$F \rightarrow aFa|bFb|aFb|bFa$ \newline
$F \rightarrow aXc$ \newline
$S_4 \rightarrow aGaaH|bGbaH|aGbaH|bGaaH|GaH$ \newline
$H \rightarrow Ha|Hb|\epsilon$ \newline
$G \rightarrow aGa|bGb|aGb|bGa$ \newline
$G \rightarrow bHc$
}\end{center}
\subsubsection*{Intuition: }Consider the cases such that the lengths of the
strings x and y are different. Then consider the cases in which the strings are
of the same length.
\newline
\textbf{Case 1: }$|x| > |y|$ Write your string such that $u_1v_1cu_2$ is the
form of the string with $|u_1|=|u_2|$ and $v_1 \neq \epsilon$.
\newline
\textbf{Case 2: }$|y| > |x|$ Proceed similarly to Case 1.
\newline
\textbf{Case 3(a): }$u_1av_1cu_2bv_2$ with $|u_1|=|u_2|$.
\newline
\textbf{Case 3(b): }$u_1bv_1cu_2av_2$ with $|u_1|=|u_2|$.

\subsection*{(ii)} A context-free grammar for the language
$L_2 = \{xcy \mid x\neq y^R,\ x,y\in\{a,b\}^*\}$
\begin{center}
\parbox{7cm}{
$S \rightarrow S_1|S_2|S_3|S_4$ \newline
$S_1 \rightarrow aS_1a|bS_1b|aS_1b|bS_1a$ \newline
$S_1 \rightarrow aDc|bDc$ \newline
$D \rightarrow aD|bD| \epsilon$ \newline
$S_2 \rightarrow aS_2a|bS_2b|aS_2b|bS_2a$ \newline
$S_2 \rightarrow cEa|cEb$ \newline
$E \rightarrow Ea|Eb| \epsilon$ \newline
$S_3 \rightarrow aFa|bFb|aFb|bFa|aXcXb$ \newline
$X \rightarrow Xa|Xb|\epsilon$ \newline
$F \rightarrow aFa|bFb|aFb|bFa$ \newline
$F \rightarrow aXcXb$ \newline
$S_4 \rightarrow aFa|bFb|aFb|bFa|bXcXa$ \newline
$X \rightarrow Xa|Xb|\epsilon$ \newline
$F \rightarrow aFa|bFb|aFb|bFa$ \newline
$F \rightarrow bXcXa$
}\end{center}

\subsubsection*{Intuition: }Consider the cases such that the lengths of
the strings x and y are different. Then consider the cases in which the strings
are of the same length.
\newline
\textbf{Case 1: }$|x| > |y|$ Write your string such that $u_1v_1cu_2$ is the
form of the string with $|u_1|=|u_2|$ and $v_1 \neq \epsilon$.
\newline
\textbf{Case 2: }$|y| > |x|$ Proceed similarly to Case 1.
\newline
\textbf{Case 3(a): }$u_1av_1cv_2bu_2^R$ (it is optional that $|v_1|=|v_2|$)
\newline
\textbf{Case 3(b): }$u_1bv_1cv_2au_2^R$

\section*{Problem B2}
\subsection*{(i)} Prove that the language $L_1 = \{a^mb^nc^p \mid 1\le m<n<p\}$
is not context-free
\medskip \newline
Proof by contradiction, so assume $L_1$ is context-free, then it must also
satisfy Ogden's lemma. Let $m$ be the constant $K > 1$ for Ogden's lemma.
Let $w = a^mb^{m+1}c^{m+2}$ such that $w \in L_1$. Mark every $a$ in $w$,
no $b$'s or $c$'s are marked. The condition that $w$ contains at least $m$
marked occurences holds since it contains exactly $m$ marked occurences.
By Ogden's lemma there exists some decomposition of $w = uvxyz$. By applying
each of the properties of Ogden's lemma (except the pumping property) we can
narrow down the decomposition of $w$ with the following restrictions:
\begin{itemize}
\item $u=a^i$ such that $i\ge 0$
\item $v=a^i$ such that $i\ge 0$
\item $x=a^ib^jc^k$ such that $i\ge 1$ and $j,k\ge 0$
\item $y=a^ib^jc^k$ such that $i,j,k\ge 0$
\item $z=a^ib^jc^k$ such that $i,j,k\ge 0$
\item $u=a^i,\ v=a^j$ such that $i,j\ge 1$ \textbf{or}
$y=a^i,\ z=a^jb^{m+1}c^{m+2}$ such that $i,j\ge 1$
\item $vxy = a^ib^jc^k$ such that $i<m$ and $j,k\ge 0$
\end{itemize}
Now we can further narrow down the decomposition of $w$ by applying the pumping
property, that is, since $w = uvxyz \in L(G)$ then $uv^nxy^nz \in L(G)$ for all
$n \ge 0$. By this property we can further restrict $y$ to the form
\begin{itemize}
\item $y=c^k$ such that $k\ge 0$
\end{itemize}
since pumping any other $y \in \Sigma^*$ would break the pumping property. We
can easily prove this by case analysis. If $y$ contained more than one symbol
from $\Sigma$, i.e. $y=a^ib^jc^k$, $y=a^ib^j$ or $y=b^jc^k$ such that $i,j,k\ge
1$, then pumping it would cause the new string to contain a substring of the
form $c^ka^i$, $b^ja^i$, or $c^kb^j$ such that $i,j,k\ge 1$ respectively,
which means that the pumped string is not in $L_1$. Thus $y$ must only contain
a single symbol from $\Sigma$. But if $y=a^i$ or $y=b^j$ such that $i,j\ge 1$
then it would again break the pumping property since it implies that $z$ must
contain all the $p$ $c$'s in $w$ and thus pumping $y$ would eventually cause
the pumped string to contain more $a$'s or $b$'s then $c$'s respectively,
which again means the pumped string is not in $L_1$. Thus we have the above
restriction to $y$ i.e. that $y=c^k$ such that $k\ge 0$. With this restriction
and we can again use the properties of Ogden's lemma to restrict $u$, $v$, and
$x$ to the forms:
\begin{itemize}
\item $x=a^ib^nc^k$ such that $i\ge 1$ and $k\ge 0$
\item $u=a^i,\ v=a^j$ such that $i,j\ge 1$
\end{itemize}
since $x$ must contain a marked occurence and therefore must also contain all
$b$'s in $w$ since $y$ contains only $c$'s. Further, $u$ and $v$ must contain
marked occurences since it is impossible for both $y$ and $z$ to. This final
form of the decomposition of $w=uvxyz$ also breaks the pumping property since
pumping $v$ will cause the number of $a$'s to be greater than the number of
$b$'s (which don't get pumped since they are all in $x$) and thus the pumped
string is not in $L_1$. Therefore, Ogden's lemma does not hold for the language
$L_1$, which is a contradiction, so $L_1$ is not context-free.

\subsection*{(ii)}Prove that the language
$L_2 = \{a^nb^nc^p \mid n,p\ge 1, p\neq n\}$ is not context-free.
\medskip \newline
Proof by contradiction, so assume $L_2$ is context-free, then it must also
satisfy Ogden's lemma. Let $m$ be the constant $K > 1$ for Ogden's lemma.
Let $p=m!$. Let $w = a^{2p}b^{2p}c^p$ such that $w \in L_2$. Mark every $c$ in
$w$, no $a$'s or $b$'s are marked. The condition that $w$ contains at least
$m$ marked occurences holds since it contains $m!$ marked occurences.
By Ogden's lemma there exists some decomposition of $w = uvxyz$ of the following
form:
\begin{itemize}
\item $u=a^ib^jc^k$ such that $i,j,k\ge 0$
\item $v=a^ib^jc^k$ such that $i,j,k\ge 0$
\item $x=a^ib^jc^k$ such that $i,j\ge 0$ and $k\ge 1$
\item $y=c^i$ such that $i\ge 0$
\item $z=c^i$ such that $i\ge 0$
\item $u=a^{2p}b^{2p}c^i,\ v=c^j$ such that $i,j\ge 1$ \textbf{or}
$y=c^i,\ z=c^j$ such that $i,j\ge 1$
\item $vxy = a^ib^{2p}c^k$ such that $0\le k<m$
\end{itemize}
Now we can further narrow down the decomposition of $w$ by applying the pumping
property, that is, since $w = uvxyz \in L(G)$ then $uv^nxy^nz \in L(G)$ for all
$n \ge 0$. By this property we can further restrict $v$ to the form
\begin{itemize}
\item $v=c^k$ such that $k\ge 0$
\end{itemize}
since pumping any other $v \in \Sigma^*$ would break the pumping property. We
can easily prove this by case analysis. If $v$ contained more than one symbol
from $\Sigma$, i.e. $v=a^ib^jc^k$, $v=a^ib^j$ or $v=b^jc^k$ such that $i,j,k\ge
1$, then pumping it would cause the new string to contain a substring of the
form $c^ka^i$, $b^ja^i$, or $c^kb^j$ such that $i,j,k\ge 1$ respectively,
which means that the pumped string is not in $L_2$. Thus $y$ must only contain
a single symbol from $\Sigma$. But if $y=a^i$ such that $i\ge 1$ then it would
break the pumping property since it implies that $x$ would contain all the $2p$
$b$'s in $w$ in which case pumping $v$ would cause the pumped string to contain
more $a$'s than $b$'s which means the pumped string would not be in the language.
And if $y=b^j$ such that $j\ge 1$ then it would again break the pumping property
since it implies that $u$ would contain all the $2p$ $a$'s in $w$ in which case
pumping $v$ would cause the pumped string to contain more $b$'s than $a$'s which
means the pumped string would not be in the language. Thus we have the above
restriction to $v$ i.e. that $v=c^k$ such that $k\ge 0$. With this we can
further restrict the decomposition of $w$ to the following form:
$$u=a^{2p}b^{2p}c^f, v=c^h, x=c^j, y=c^k,\text{ and } z=c^l
\text{ such that } f,h,k,l\ge 0 \text{ and } j\ge1$$
And so we have that:
$$w = a^{2p}b^{2p}c^{f+h+j+k+l} \text{ where } f,h,k,l\ge 0, j\ge 1,
\text{ and } f+h+j+k+l=p$$
which by the pumping property of Ogden's lemma implies that
$$w = a^{2p}b^{2p}c^{i(h+k) + (f+j+l)} \text { where } i(h+k)+(f+j+l)\neq 2p
\text{ for all } i\ge 0$$
where either $f,h\ge 1$ or $k,l\ge 1$ by the property of Ogden's lemma that
either both $u$ and $v$ or both $y$ and $z$ must contain some marked occurences.
So, for Ogden's lemma to hold it must be the case that:
$$i \neq \frac{2p-(f+j+l)}{(h+k)} \text{ for all } i\ge 0$$
where $f,h,k,l \ge 0, j\ge 1,$ either $f,h\ge 1$ or $k,l\ge 1$ and
$f+j+l+h+l = p$ i.e. $(f+j+l) \le p \le 2p$ and $(h+k) \le p \le 2p$. And we
know that
$$p = m!$$
where $h+j+k < m$ by the property of Ogden's lemma that $v,x,y$ must contain
less than $m$ marked occurences. Which means that
$$\frac{2p}{(h+k)}-\frac{(f+j+l)}{(h+k)}$$
is a positive integer which is a contradiction since $i\ge 0$ is any positive
integer therefore Ogden's lemma does not hold, so $L_2$ is not context-free.

\section*{Problem B3}
Given a context-free language $L$ and a regular language $R$, prove that
$L\cap R$ is context-free. Proof by construction of a CFG $G_2$ such that
$L(G_2) = L\cap R$. Let $G = (V,\Sigma,P,S)$ be a CFG in Chomsky normal form
such that $L(G) = L$ and $N = V - \Sigma$ be the set of nonterminals in $G$.
Let $D = (Q,\Sigma, \delta, q_0, F)$ be a DFA such that $L(D) = R$.
Let $G_2 = (V',\Sigma,P',S_0)$ be a CFG where $N' = V' - \Sigma$ be the set of
nonterminals in $G_2$ where $N' = Q\times N\times Q \cup \{S_0\}$ and
productions in $P'$ are defined by the following properties:
\begin{itemize}
\item $(S_0 \rightarrow (q_0,S,f)) \in P'$ for every $f \in F$
\item $(S_0 \rightarrow \epsilon) \in P' \iff (S\rightarrow \epsilon) \in P$ and
$q_0 \in F$
\item $((p,A,\delta(p,a)) \rightarrow a) \in P' \iff (A \rightarrow a) \in P$
for all $a \in \Sigma$, all $A \in N$, and all $p \in Q$.
\item $((p,A,s) \rightarrow (p,B,q)(q,C,s)) \in P' \iff (A \rightarrow BC)\in P$
for all $p,q,s \in Q$ and all $A,B,C \in N$
\end{itemize}
\textbf{Claim: } For all $p,q\in Q$, all $A\in N$, all $w\in \Sigma^+$,
and all $n\ge 1$
$$(p,A,q) \underset{lm}{\overset{n}{\Longrightarrow}}_{G_2} w
\iff A \underset{lm}{\overset{n}{\Longrightarrow}}_{G} w
\text{ and } \delta^*(p,w) = q$$
\textbf{Case: } $(p,A,q) \underset{lm}{\overset{n}{\Longrightarrow}}_{G_2} w
\implies A \underset{lm}{\overset{n}{\Longrightarrow}}_{G} w
\text{ and } \delta^*(p,w) = q$
\newline \textbf{Proof: } By induction on $n$
\newline \textbf{Base Case: } $n = 1$.
\medskip \newline
So given that
$$((p,A,q) \rightarrow a) \in P' \text{ for } a\in \Sigma$$
We need to prove that
$$ (A \rightarrow a) \in P \text{ and } \delta(p,a) = q$$
Which we have immediately from our above construction of $P'$ where
$$((p,A,\delta(p,a)) \rightarrow a) \in P' \implies
((p,A,q) \rightarrow a) \in P' \text{ where } q=\delta(p,a) \text{ and }
(A\rightarrow a) \in P$$
\newline \textbf{Inductive Case: } $n = n+1$.
\newline \textbf{Induction Hypothesis: }
$(p,A,q) \underset{lm}{\overset{n}{\Longrightarrow}}_{G_2} w
\implies A \underset{lm}{\overset{n}{\Longrightarrow}}_{G} w
\text{ and } \delta^*(p,w) = q$.
\medskip \newline
So given that
$$(p,A,q) \underset{lm}{\overset{n+1}{\Longrightarrow}}_{G_2} w$$
We need to prove that
$$A \underset{lm}{\overset{n+1}{\Longrightarrow}}_{G} w
\text{ and } \delta^*(p,w) = q$$
From our construction of $G_2$ above we have that
$$(p,A,q) \rightarrow (p,B,s)(s,C,q)
\underset{lm}{\overset{n}{\Longrightarrow}}_{G} w$$
Let $w=uv$ where $u,v\in \Sigma^*$ then we have
$$ (p,A,q) \underset{lm}{\Longrightarrow}_{G} (p,B,s)(s,C,q)
\underset{lm}{\overset{n_1}{\Longrightarrow}}_{G} u(s,C,q)
\underset{lm}{\overset{n_2}{\Longrightarrow}}_{G} uv = w
\text{ where } n_1,n_2 \le n $$
Then by the induction hypothesis we have that
$$B \underset{lm}{\overset{n_1}{\Longrightarrow}}_{G} u
\text{ and } \delta^*(p,u) = s$$
$$C \underset{lm}{\overset{n_2}{\Longrightarrow}}_{G} v
\text{ and } \delta^*(s,v) = q$$
And by construction we have
$$A\rightarrow BC \in P$$
So we have that
$$A \underset{lm}{\Longrightarrow}_{G} BC
\underset{lm}{\overset{n_1}{\Longrightarrow}}_{G} uC
\underset{lm}{\overset{n_1}{\Longrightarrow}}_{G} uv
\text{ and } \delta^*(\delta^*(p,u),v) = q$$
which is equivalent to
$$A \underset{lm}{\overset{n+1}{\Longrightarrow}}_{G} w
\text{ and } \delta^*(p,w) = q$$
which is what we are trying to prove.
\medskip \newline
\textbf{Case: } $A \underset{lm}{\overset{n}{\Longrightarrow}}_{G} w
\text{ and } \delta^*(p,w) = q \implies
(p,A,q) \underset{lm}{\overset{n}{\Longrightarrow}}_{G_2} w$
\newline \textbf{Proof: } By induction on $n$
\newline \textbf{Base Case: } $n = 1$.
\medskip \newline
So given that
$$ (A \rightarrow a) \in P \text{ and } \delta(p,a) = q$$
We need to prove that
$$((p,A,q) \rightarrow a) \in P' \text{ for } a\in \Sigma$$
By construction we have that
$$(A \rightarrow a) \in P \implies ((p,A,\delta(p,a)) \rightarrow a) \in P'$$
and by substituion we have
$$((p,A,q) \rightarrow a) \in P'$$
so the base case is proven trivially.
\medskip \newline
\textbf{Inductive Case: } $n = n+1$.
\newline \textbf{Induction Hypothesis: }
$A \underset{lm}{\overset{n}{\Longrightarrow}}_{G} w
\text{ and } \delta^*(p,w) = q \implies
(p,A,q) \underset{lm}{\overset{n}{\Longrightarrow}}_{G_2} w$
\medskip \newline
So given that
$$A \underset{lm}{\overset{n+1}{\Longrightarrow}}_{G} w
\text{ and } \delta^*(p,w) = q$$
We need to prove that
$$(p,A,q) \underset{lm}{\overset{n+1}{\Longrightarrow}}_{G_2} w$$
Since $G$ is in Chomskey normal form we have that
$$A \rightarrow BC \underset{lm}{\overset{n}{\Longrightarrow}}_{G} w$$
Let $w =uv$ for $u,v\in \Sigma^*$, then we have
$$A \underset{lm}{\Longrightarrow}_{G} BC
\underset{lm}{\overset{n_1}{\Longrightarrow}}_{G} uC
\underset{lm}{\overset{n_2}{\Longrightarrow}}_{G} uv
\text{ where } n_1,n_2\le n$$
So we have
$$B \underset{lm}{\overset{n_1}{\Longrightarrow}}_{G} u
\text{ and } \exists s\in Q,\ \delta^*(p,u) = s$$
$$C \underset{lm}{\overset{n_2}{\Longrightarrow}}_{G} v
\text{ and } \exists q\in Q,\ \delta^*(s,u) = q$$
So by the induction hypothesis we have
$$(p,B,s) \underset{lm}{\overset{n}{\Longrightarrow}}_{G_2} u$$
$$(s,C,q) \underset{lm}{\overset{n}{\Longrightarrow}}_{G_2} v$$
and by construction given that we have (from above) that
$$A \rightarrow BC \in P$$
we have
$$(p,A,q) \rightarrow (p,B,s)(s,C,q) \in P'$$
so by substitution we have
$$(p,A,q) \underset{lm}{\Longrightarrow}_{G} (p,B,s)(s,C,q)
\underset{lm}{\overset{n_1}{\Longrightarrow}}_{G_2} u(s,C,q)
\underset{lm}{\overset{n_2}{\Longrightarrow}}_{G_2} uv$$
which is equivalent to
$$(p,A,q) \underset{lm}{\overset{n+1}{\Longrightarrow}}_{G_2} w$$
which is what we are trying to prove. So our claim is fully proven.
We can now show that $L(G_2) = L\cap R$. By definition
$L(G_2) = \{w \in \Sigma^* \mid S_0 \overset{+}{\Longrightarrow} w\}$,
$L = L(G) = \{w \in \Sigma^* \mid S \overset{+}{\Longrightarrow} w\}$, and
$R = L(D) = \{w \in \Sigma^* \mid \delta^*(q_0,w) \in F\}$. By our
construction of $G_2$ we have that
$$(S_0 \rightarrow (q_0,S,f)) \in P' \text{ for every } f \in F$$
and by our above claim we know that
$$(q_0,S,f) \underset{lm}{\overset{n}{\Longrightarrow}}_{G_2} w \iff
S \underset{lm}{\overset{n}{\Longrightarrow}}_{G} w
\text{ and } \delta^*(q_0,w) = f \text{ where } f\in F$$
Therefore it is immediate that
$$ S_0 \overset{+}{\Longrightarrow}_{G_2} w \iff
S \overset{+}{\Longrightarrow}_{G} w \text{ and } \delta^*(q_0,w) \in F$$
thus $L(G_2) = L\cap R$ which implies that the language $L\cap R$ is
context-free.

\section*{Problem B4}
\subsection*{(i)} Consider $r = k!$ where $k$ is the constant of the pumping
lemma for the context-free language $G$. We prove that for all $w \in L$,
if $|w|\ge k$, then $\{wa^{rn} | n \ge 0 \} \subseteq L$. Let $w=uvxyz$, where
$u=a^{i_1}$, $v=a^{i_2}$, $x=a^{i_3}$ , $y=a^{i_4}$ , and $z=a^{i_5}$.
From the pumping lemma we have that $i_3 \ge 1$, $i_2 + i_4 \ge 1$ and
$i_2 + i_3 + i_4 \le k$. Then consider $uv^mxy^mz \in L$ for all $m \ge 0$.
So, $uv^mxy^mz = a^{i_1+i_3+i_5 + m(i_2 + i_4)}$. Let $k_1 = i_1 + i_3 + i_5$
and $k_2 = i_2 + i_4$. Then, $a^{k_1 + mk_2} \in L$ for all $m \ge 0$.
We have that $1 \le k_2 \le k$ and $p=k_1+k_2$. For any $n \ge 0$ we can find
an $m$ such that $mk_2=k_2 +nk$ by letting $m=1+ \dfrac{nr}{k_2}$. So,
$k_1+mk_2=k_1+k_2+nr=p+nr$. This shows that
$a^{k_1+mk_2}=a^{p+nr}=wa^{nr} \in L$, which is what we wanted to prove.

\subsection*{(ii)}
For any $i$ such that $0 \le i < r$ let
$$L_i = \{a^n |a^n \in L, n \ge k , n \equiv i \bmod r \}$$
Then clearly,
$$L = \{ a^n | a^n \in L , n < k\} \cup \bigcup_{0 \le i < r} L_i$$
where $\{ a^n | a^n \in L , n < k\}$ is finite.
If $L_i \neq \emptyset$, then let $z_i$ be the shortest string in $L_i$
such that $|z_i| \ge k$. We claim that
$$L_i = \{z_i a^{rm} | m \ge 0\}$$
By part (i) we know that $\{z_i a^{rm} | m \ge 0\} \subseteq L_i$.
To prove the reverse induction, we can choose any $a^n \in L_i$.
Let $z_i = a^{n_i}$. Then we have $n \ge n_i$, $n_i = m_ir + i$ and $n=mr +i$.
We also know that $m \ge m_i$ which gives $n=(m-m_i)k + m_ik + i$.
So we have that $n=(m-m_i)k + n_i$. But $m-m_i \ge 0$,
so $a_n=a^{n_i}a^{(m-m_i)r}=z_ia^{(m-m_i)r}$.
Since each $L_i$ can be written as a regular expression of the form
$\{z_ia^r\}^*$ each $L_i$ is regular. Since $L$ is the union of a finite set of
$L_i$'s together with the union of another trivially regular language,
$L$ is regular.

\subsection*{(iii)}
We just showed that $L_i$ is regular. Since it is regular, it is also
context-free so we can construct a grammar $G_i$ for $L_i$ i.e. a grammar such
that $L(G_i) = L_i$. It is decidable whether or not the start symbol of the
grammar $G_i$ is in $T(G_i)$. If it isn't then it must be the case that
$L_i = \emptyset$. Therefore it is decidable whether $L_i = \emptyset$.

\subsection*{(iv)} Given that $L \subseteq \{a,b\}^*$ and $L$ is a context free
language, it is the case that $\{a\}^* \subseteq L$ if and only if
$\{a\}^* \cap L=\{a\}^*$. Note that the language $\{a\}^*$ is trivially regular.
Let $L_1=\{a\}^* \cap L$. Since we have previously proven that context-free
languages are closed under intersection with regular languages, $L_1$ is a
context free language. Further, it is clear that $L_1 \subseteq \{a\}^*$.
By part (ii) of this problem, $L_1$ is regular. Since $L_1$ and $\{a\}^*$ are
both regular we can create DFAs for both languages, Since it is decidable
whether two DFAs are equivalent (by comparing their minimal DFAs, where minimal
DFAs are unique for each regular language) and therefore whether two regular
languages are equivalent, it is decidable whether $\{a\}^* \subseteq L$
by testing whether $L_1$ and $\{a\}^*$ are equivalent.

\section*{Problem B5}
\subsection*{(1)} Let $G_2 = (\Sigma\cup \{S\},\Sigma,P,S)$ be a context-free
grammer for the language $D_2$ where the productions in P are defined as:
$$ S \rightarrow SS \mid \epsilon \mid aS\bar a \mid bS\bar b$$
This is a grammar for the language $D_2$ for the following reasons.
The smallest string possible is $\epsilon$ which is in the language.
Llarger words are constructed by pairing an $a$ and an $\bar a$ or a $b$ and a
$ \bar b$. If we remove these pairings from the word following the rules for
$\simeq$ we will always get the word $\epsilon$ therefore the word is in $D_2$.

\subsection*{(2)} Let $G_m = (\Sigma\cup \{S\},\Sigma,P,S)$ be a context-free
grammer for the language $D_m$ where the productions in P are defined as:
$$S \rightarrow SS\mid \epsilon \mid a_iS\bar a_i \text{ where } 1\le i\le m$$
Observation (OBS1): if $u \simeq^* \epsilon$ then $wuz \simeq^* wz$\newline
Proof: if $u \simeq^* \epsilon$ then $u$ is of the type $a_i\ptb{a_i} \rightarrow wuz = wa_i\ptb{a_i}z \rightarrow wa_i\ptb{a_i}z \simeq^* wz$\newline
We first will prove that $L(G_m) \subseteq D_m$:\newline
$\forall w \in L(G_m) \rightarrow w \in D_m$ : Proof by Induction: on k $\mid S \rightarrow^k w$\newline
$k = 0 : S \rightarrow^0 \epsilon$ and $\epsilon \simeq^* \epsilon$\newline
Inductive step \newline
$S \rightarrow^{k+1} w$ we have two cases $1) S \rightarrow a_iS\ptb{a_i} \rightarrow^{k} w \text{ or } 2)S \rightarrow SS \rightarrow^{k} w$\newline
case 1: $w = a_iu\ptb{a_i} \text{ and } S \rightarrow^k u$ from our IH we have $u \simeq^* \epsilon$ and from OBS1 we can remove $u$ so we have $a_i\ptb{a_i} \simeq^* \epsilon$ so good.\newline
case 2: $w = uv , S\rightarrow^{k_1} u , S\rightarrow^{k_2} v$ with $k_1 + k_2 = k + 1$ and $k1, k2 \le k$.\newline
from IH we have $u \simeq^* \epsilon$ and $v \simeq^* \epsilon \rightarrow uv  \simeq^* \epsilon$ so $uv \in D_m$
So we have $w in D_m$ if $w \in L(G_m)$.\newline
We next prove $D_m \subseteq L(G_m)$
$\forall w \in D_m \rightarrow w \in L(G_m)$ : Proof by Induction on $|w|$\newline
$|w| = 0 \rightarrow \epsilon \in D_m$ and $\epsilon \in L(G_m)$ so good.\newline
Inductive Step:\newline
$|w| \ge 1 -> w \simeq^* \epsilon$ and we have 2 cases: \newline
1) $w = a_iu\ptb{a_i} \rightarrow |u| \ge |w| \text{ and from IH } u \simeq^* w \text{ from OBS1 we can remove u and } a_i\ptb{a_i} \simeq^* \epsilon$ so good \newline
2) $w=uv \rightarrow |u| \le |w| \rightarrow u \simeq^* \epsilon , |v| \le |w| \rightarrow v \simeq^* \epsilon \rightarrow uv \simeq^* \epsilon$ so good\newline
So now we have $L(G_m) \subseteq D_m \text{ and } D_m \subseteq L(G_m) \text{ therefore } D_m = L(G_m)$

\section*{Problem B6}
\subsection*{(i)} For the example where
$\tau$: $\Sigma \rightarrow 2^{\Sigma^*}$ is the subsitution defined such
that $\tau(a) = \{\epsilon, a\}$, $\tau(L)$ is the language consisting of all
words $w \in L$ and every modified version of $w$ where any character(s) in $w$
can be replaced by $\epsilon$ characters. For example if
$w=a_1a_2\cdots a_n \in L$ then
$\tau(w) = \{\epsilon, a_1\}\{\epsilon, a_2\}\cdots \{\epsilon, a_n\}
\subseteq \tau(L)$.

\subsection*{(ii)}
For every production $A \rightarrow \beta$ in $G$, we want to replace every
terminal $a$ by $S_a$, $S_a = \tau(a)$, and leave every nonterminal alone.
We can achieve this by defining a substitution $\tau_1$ such that
$\tau_1(a)=V_a$ with $a \in \Sigma$ and $\tau_1(\beta)=\beta$ with
$\beta \in N$ where $V \cap V_a = \emptyset$ and $V_a \cap V_b = \emptyset$
for all $a \neq b$. We define a grammar,
$$G_{\tau} = (V \cup \bigcup_{a \in \Sigma} V_a, \Delta,
P \cup \bigcup_{a \in \Sigma} P_a, S)$$
where $P_a$ is the set of productions that contain all the
$V_a$ productions for all $a \in \Sigma$. Since we have constructed a grammar
that defines the language $L$, it is context-free.

\section*{Problem B7}
\subsection*{(i)} From the defintion of $\tau_1$ we have that:
$$\tau_1(b)= v_1bv_2 \text{ such that } v_1,v_2 \in (\bar\Omega \cup \{E\}^*)
\text{ for all } b\in \Delta$$
So we can extend $\tau_1$ for languages $L \subseteq \Delta^*$ such that
$\epsilon \not\in L$ to:
$$\tau_1(L) = \{v_1b_1v_2\cdots v_nb_nv_{n+1} \mid
v_j \in (\bar\Omega \cup \{E\}^*),\ b_1\cdots b_n \in L\}$$
And from the definition of $R$ we have that
$$ R = \{u_1\bar a_1h(a_1)\cdots u_n\bar a_nh(a_n)
\mid a\in \Omega,\ u_i\in \{E\}^*\}$$
So we can define $L_2 = \tau_1(L) \cap R$ as
$$L_2 = \{u_1\bar a_1h(a_1)\cdots u_n\bar a_nh(a_n)
\mid u_i \in \{E\}^*.\ h(a_1)\cdots h(a_n) \in L,\ a_i\in \Omega\}$$
Finally we can see that $L_2 \cap (\bar\Omega\cup\{h(a) \mid a\in \Omega\})^*$
contains any word from $L_2$ without any instances of $E$ i.e. that all
$u_i = \epsilon$ so we have
$$L_2\cap \left(\ptb{\Omega}\cup \{h(a) \mid a\in\Omega\}\right)^*
=\{\ptb{a_1}h(a_1) \ptb{a_2}h(a_2)\cdots \ptb{a_n}h(a_n)\ |\
h(a_1a_2\cdots a_n)\in L,\ a_i\in \Omega\}$$

\subsection*{(ii)} If $\epsilon \not\in L$ then we have from above that:
$$\tau_2(g(\tau_1(L)\cap R)) = \tau_2(g(L_2))$$
And we can extended the definition of $g$ to languages to give:
$$L_3 = g(L_2) = \{u_1a_1u_2a_2\cdots u_na_nu_{n+1} \mid u_i \in \{E\}^*
,\ h(a_1a_2\cdots a_n) \in L,\ a_i\in \Omega
,\ a_1a_2\cdots a_n \in h^{-1}(L)\}$$
where all the instances of $h(a_i)$ in strings in $L_2$ where replaced with
$\epsilon$ by $g$ and all instances of $\bar a_i$ where replaced by
instances of $a_i$ by $g$. By applying the extension of $\tau_2$ to languages
we get:
$$\tau_2(L_3) = \{v_1a_1v_2a_2\cdots v_na_nv_{n_1}
\mid h(a_1a_2\cdots a_n) \in L,\ v_1\in \Gamma^*\}$$
where
$$h(v_1a_1v_2a_2\cdots v_na_nv_{n+1}) = h(a_1a_2\cdots a_n)$$
since applying the homomorphism $h$ to a an element of $\Gamma$ is replaced by
$\epsilon$ from the definition of $\Gamma$.
So since
$$h(a_1a_2\cdots a_n) \in L$$
then
$$h(v_1a_1v_2a_2\cdots v_na_nv_{n+1}) \in L$$
which implies that
$$v_1a_1v_2a_2\cdots v_na_nv_{n+1} \in h^{-1}(L)$$
thus
$$\tau_2(L_3) = \tau_2(g(\tau_1(L)\cap R)) = h^{-1}(L) \text{ when }
\epsilon \not\in L$$
If $\epsilon \in L$, then it is also possible for words from $\Gamma^*$ to be in
$\tau_2(g(\tau_1(L)\cap R))$ since we need to condider the case when
$g(a) = \epsilon$ when $a\in \Delta$. So we can modify our above equality to be:
$$h^{-1}(L) = \tau_2(g(\tau_1(L-\{\epsilon\}) \cap R) \cup \Gamma^*
\text{ when } \epsilon\in L$$
Since $L\in \s{L}$ it is closed under substituion, intersection and union with
regular languages. Since homomorphisms are a special kind of regular substituion
$L$ is also closed under homomorphism. Therefore
$$\tau_2(g(\tau_1(L)\cap R)) \in \s{L} \text{ when } \epsilon \not\in L$$
and
$$\tau_2(g(\tau_1(L-\{\epsilon\}) \cap R) \cup \Gamma^* \in \s{L} \text{ when }
\epsilon\in L$$
which means
$$h^{-1}(L) \in \s{L}$$
since we proved its equality to those two languages.
Therefore $L$ is closed under inverse homomorphism.

\subsection*{(iii)} We proved in the previous problem that context-free languages
are closed under substituion by regular languages. Further, in problem 3, we
proved that context-free languages are closed under intersection with regular
languages. It is given that context-free languages are closed under union
with regular languages. Therefore, if $L$ is a context-free language,
$L \in \s{L}$. Since we just proved that $\s{L}$ is closed under inverse
homomorphism, then it is immediate that $L$ is closed under inverse
homomorphism. Therefore $h^{-1}(L)$ is context-free if $L$ is context-free.


\end{document}
