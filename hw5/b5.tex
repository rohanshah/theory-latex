\documentclass[12pt]{article}
\usepackage{fullpage}
\usepackage{titlesec}
\usepackage{tikz}
\usepackage{amsfonts,amssymb}
\usepackage{amsmath}
\usepackage{comment}
\usetikzlibrary{automata, positioning}

\input ../libraries/mac.tex
\input ../libraries/mathmac.tex

\begin{document}
\pagestyle{plain}
\titleformat{\subsection}[runin]
  {\normalfont\large\bfseries}{\thesubsection}{1em}{}

\title{Homework 5}
\author{Brooke Fugate, Michael O'Connor, Rohan Shah}
\date{}

\maketitle

\section*{Problem B5}
\subsection*{(1)} Let $G_2 = (\Sigma\cup \{S\},\Sigma,P,S)$ be a context-free
grammer for the language $D_2$ where the productions in P are defined as:
$$ S \rightarrow SS \mid \epsilon \mid aS\bar a \mid bS\bar b$$
This grammar is correct because the smallest string possible is $\epsilon$ which is in the language. Then, to make larger words, both an $a \bar a$ or a $b \bar b$ must be added. There will always be a pair $a \bar a$ or $b \bar b$ in the word that we can remove, following the rules for $\simeq$, until we get the word $\epsilon$.

\subsection*{(2)} Let $G_m = (\Sigma\cup \{S\},\Sigma,P,S)$ be a context-free
grammer for the language $D_m$ where the productions in P are defined as:
$$S \rightarrow SS\mid \epsilon \mid a_iS\bar a_i \text{ where } 1\le i\le m$$


Observation (OBS1): if $u \simeq^* \epsilon$ then $wuz \simeq^* wz$\newline
Proof: if $u \simeq^* \epsilon$ then $u$ is of the type $a_i\ptb{a_i} \rightarrow wuz = wa_i\ptb{a_i}z \rightarrow wa_i\ptb{a_i}z \simeq^* wz$\newline

We first will prove that $L(G_m) \subseteq D_m$:\newline
$\forall w \in L(G_m) \rightarrow w \in D_m$ : Proof by Induction: on k $\mid S \rightarrow^k w$\newline
$k = 0 : S \rightarrow^0 \epsilon$ and $\epsilon \simeq^* \epsilon$\newline
Inductive step \newline
$S \rightarrow^{k+1} w$ we have two cases $1) S \rightarrow a_iS\ptb{a_i} \rightarrow^{k} w \text{ or } 2)S \rightarrow SS \rightarrow^{k} w$\newline
case 1: $w = a_iu\ptb{a_i} \text{ and } S \rightarrow^k u$ from our IH we have $u \simeq^* \epsilon$ and from OBS1 we can remove $u$ so we have $a_i\ptb{a_i} \simeq^* \epsilon$ so good.\newline
case 2: $w = uv , S\rightarrow^{k_1} u , S\rightarrow^{k_2} v$ with $k_1 + k_2 = k + 1$ and $k1, k2 \le k$.\newline
from IH we have $u \simeq^* \epsilon$ and $v \simeq^* \epsilon \rightarrow uv  \simeq^* \epsilon$ so $uv \in D_m$
So we have $w in D_m$ if $w \in L(G_m)$.\newline

We next prove $D_m \subseteq L(G_m)$
$\forall w \in D_m \rightarrow w \in L(G_m)$ : Proof by Induction on $|w|$\newline
$|w| = 0 \rightarrow \epsilon \in D_m$ and $\epsilon \in L(G_m)$ so good.\newline
Inductive Step:\newline
$|w| \ge 1 -> w \simeq^* \epsilon$ and we have 2 cases: \newline
1) $w = a_iu\ptb{a_i} \rightarrow |u| \ge |w| \text{ and from IH } u \simeq^* w \text{ from OBS1 we can remove u and } a_i\ptb{a_i} \simeq^* \epsilon$ so good \newline
2) $w=uv \rightarrow |u| \le |w| \rightarrow u \simeq^* \epsilon , |v| \le |w| \rightarrow v \simeq^* \epsilon \rightarrow uv \simeq^* \epsilon$ so good\newline

So now we have $L(G_m) \subseteq D_m \text{ and } D_m \subseteq L(G_m) \text{ therefore } D_m = L(G_m)$

\end{document}
