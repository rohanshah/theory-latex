\documentclass[12pt]{article}
\usepackage{fullpage}
\usepackage{titlesec}
\usepackage{tikz}
\usepackage{amsfonts,amssymb}
\usepackage{amsmath}
\usepackage{comment}
\usetikzlibrary{automata, positioning}

\input ../libraries/mac.tex
\input ../libraries/mathmac.tex

\begin{document}
\pagestyle{plain}
\titleformat{\subsection}[runin]
  {\normalfont\large\bfseries}{\thesubsection}{1em}{}

\title{Homework 5}
\author{Brooke Fugate, Michael O'Connor, Rohan Shah}
\date{}

\maketitle

\section*{Problem B5}
\subsection*{(1)} Let $G_2 = (\Sigma\cup \{S\},\Sigma,P,S)$ be a context-free
grammer for the language $D_2$ where the productions in P are defined as:
$$ S \rightarrow SS \mid \epsilon \mid aS\bar a \mid bS\bar b$$
This grammar is correct because the smallest string possible is $\epsilon$ which is in the language. Then, to make larger words, both an $a \bar a$ or a $b \bar b$ must be added. There will always be a pair $a \bar a$ or $b \bar b$ in the word that we can remove, following the rules for $\simeq$, until we get the word $\epsilon$.

\subsection*{(2)} Let $G_m = (\Sigma\cup \{S\},\Sigma,P,S)$ be a context-free
grammer for the language $D_m$ where the productions in P are defined as:
$$S \rightarrow SS\mid \epsilon \mid a_iS\bar a_i \text{ where } 1\le i\le m$$

To prove this is the correct CFG, assume $w=a_iu \bar a_i \in D_m(w \simeq^{*} \epsilon)$. We then consider three cases: 
\begin{enumerate}
\item $a_iu \bar a_i = a_i \bar a_i u' \bar a_i \simeq u' \bar a_i \simeq^{*} \epsilon$
\item $a_iu \bar a_i = a_i u'a_i \bar a_i \simeq a_iu' \simeq^{*} \epsilon$ 
\item $a_iu \bar a_i \simeq a_i u' \bar a_i \simeq^{*} \epsilon$ 
\end{enumerate}
The first two cases are obviously correct, so we consider the third case.
Let $u \simeq u' and a_iu' \bar a_i \simeq^* \epsilon$ with $|u| < |u|$. We now have two new cases to consider. In the first case, $u' \simeq^{*} \epsilon$ and in the second case, $a_iu' \bar a_i = a_iu''v'' \bar a_i \simeq^{*} \epsilon$ and $a_iu'' \simeq^{*} \epsilon , v'' \bar a_i \simeq^{*} \epsilon$. So $u'=u''v''$ and we can insert some $a_j\bar a_j$ into $u''$ to get $u_1$. So $a_iu_1 \simeq a_iu'' \simeq^* \epsilon$. 
Next we consider $w=a_iu_iv'' \bar a_i \simeq \epsilon \epsilon$.
Insert $a_j \bar a_j$ into $v''$to get $v_1$, then $a_1 \bar a_j \simeq v'' \bar a_j \simeq^* \epsilon$.Then, 
$w=a_iu''v_1 \bar a_i \simeq^* \epsilon \epsilon$. 

Next, we have $w=a_iu \bar a_j \simeq^* \epsilon$ with $i \neq j$. There are three subcases. First, $a_i \bar a_i u' \bar a_j \simeq u' \bar a_j \simeq^* \epsilon$. This is the same case as before. Then, $a_iu'a_j \bar a_j \simeq a_iu' \simeq* \epsilon$ which is also the same as case 2. Note, the case where $u \simeq^* \epsilon$ is impossible. Finally we have the third case which is similar to a previous case. $a_iu''v'' \bar a_j$. 


















\end{document}
