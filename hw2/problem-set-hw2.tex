\documentclass[12pt]{article}
\usepackage{amsfonts,amssymb}
\usepackage{amsmath}

\setlength{\topmargin}{-.5in}
\setlength{\oddsidemargin}{0 in}
\setlength{\evensidemargin}{0 in}
\setlength{\textwidth}{6.5truein}
\setlength{\textheight}{8.5truein}
%
\input ../libraries/mac.tex
\input ../libraries/mathmac.tex
%
\input xy
\xyoption{all}
\def\fseq#1#2{(#1_{#2})_{#2\geq 1}}
\def\fsseq#1#2#3{(#1_{#3(#2)})_{#2\geq 1}}
\def\qleq{\sqsubseteq}
\def\Hom#1#2#3{\mathrm{Hom}_{#1}(#2, #3)}

%
\begin{document}
\begin{center}
\fbox{{\Large\bf Spring, 2014 \hspace*{0.4cm} CIS 511}}\\
\vspace{1cm}
{\Large\bf Introduction to the Theory of Computation\\
Jean Gallier \\
\vspace{0.5cm}
Homework 2}\\[10pt]
February 4,  2014; Due February 18, 2014, {\it beginning of class}\\
\end{center}


``A problems'' are for practice only, and should not
be turned in.

\medskip\noindent
{\bf Problem A1.} 
Recall that two regular expressions $R$ and $S$ are equivalent, denoted
as $R \cong S$, iff they denote the same regular language 
${\cal L}[R] = {\cal L}[S]$.
Show that the following identities hold for regular expressions:
\begin{align*}
R^{**} &\cong  R^{*}\\
(R + S)^{*} &\cong  (R^{*} + S^{*})^{*}\\
(R + S)^{*} &\cong  (R^{*}S^{*})^{*}\\
(R + S)^{*} &\cong  (R^{*}S)^{*}R^{*}
\end{align*}


\vspace {0.25cm}\noindent
{\bf Problem A2.} 
Recall that a homomorphism $\mapdef{h}{\Sigma^*}{\Delta^*}$
is a function such that $h(uv) = h(u)h(v)$
for all $u, v\in\Sigma^*$.
Given any language $L\subseteq \Sigma^*$, we define $h(L)$ as
$$h(L) = \{h(w)\ |\ w\in L\}.$$
Given any language $L'\subseteq \Delta^*$, we define $h^{-1}(L')$ as
$$h^{-1}(L') = \{w\in \Sigma^*\ | \ h(w) \in L'\}.$$

\medskip
Prove that if  $L\subseteq \Sigma^*$ and
$L'\subseteq \Delta^*$ are regular languages, then
so are $h(L)$ and $h^{-1}(L')$.


\medskip\noindent
{\bf Problem A3.} 
Construct an NFA accepting the language
$L = \{aa, aaa\}^*$. Apply the subset construction
to get a DFA accepting $L$.

\vspace {0.25cm}
``B3  problems'' must be turned in.

% cis51104hw2.tex

\vspace{0.25cm}\noindent
{\bf Problem B1 (40 pts).}  
Let $\Sigma = \{a_1,\ldots, a_n\}$ 
be an alphabet of $n$ symbols.

\medskip
(a) Construct an NFA with $2n+1$ (or $2n$) states accepting the
set $L_n$ of strings over $\Sigma$ such that,
every string in $L_n$ has an odd number of $a_i$, for some $a_i\in\Sigma$.
Equivalently, if $L_{n}^{i}$ is the set of all strings
over $\Sigma$ with an odd number of $a_i$, then
$L_n = L_{n}^{1}\cup \cdots \cup L_{n}^{n}$.

\medskip
(b) Prove that there is a DFA with $2^n$ states accepting the language
$L_n$.

\medskip
(c)
Prove that every DFA accepting $L_n$ has at least $2^n$ states.

\medskip\noindent
{\it Hint\/}: If a DFA $D$ with $k < 2^n$ states accepts $L_n$, show
that there are two strings $u, v$ with the property
that, for some $a_i\in\Sigma$, 
$u$ contains an odd number of $a_i$'s, 
$v$ contains an even number of $a_i$'s, 
and $D$ ends in the same state after processing
$u$ and $v$. From this, conclude that $D$ accepts 
incorrect strings.

\vspace{0.25cm}\noindent
{\bf Problem B2 (30 pts).}  
(a)
Let $T = \{0, 1, 2\}$, let $C$ be the set of $20$ strings
of length three over the alphabet $T$,
\[
C = \{u\in T^3 \mid u \notin \{110, 111, 112, 101, 121, 011, 211\}\},
\]
let $\Sigma = \{0, 1, 2, c\}$ and consider the language
\[
L_M = \{w \in \Sigma^* \mid
w = u_1cu_2c \cdots cu_n,\, n \geq 1, u_i \in C\}.
\]
Prove that $L$ is regular.

\medskip
(b)
The language $L_M$ has a geometric interpretation as a certain
subset of $\reals^3$ (actually, $\rats^3$), as follows:
Given any string,
$w = u_1cu_2c \cdots cu_n \in L_M$, denoting the $j$th character
in $u_i$ by $u_i^j$, where $j\in \{1, 2, 3\}$, we obtain three
strings
\begin{eqnarray*}
w^1 & = & u_1^1u_2^1 \cdots u_n^1 \\
w^2 & = & u_1^2u_2^2 \cdots u_n^2 \\
w^3 & = & u_1^3u_2^3 \cdots u_n^3.
\end{eqnarray*}
For example, if $w = 012c001c222c122$ we have
$w^1 = 0021$, $w^2 = 1022$, and $w^3 = 2122$.
Now, a string $v\in T^+$ can be interpreted as a decimal real number
written in base three! Indeed, if
\[
v = b_1b_2\cdots b_k,
\quad\hbox{where}\quad
b_i\in \{0, 1, 2\} = T\>\> (1\leq i \leq k),
\]
we interpret $v$ as $n(v) = 0.b_1b_2\cdots b_k$, i.e.,
\[
n(v) = b_1 3^{-1} + b_2 3^{-2} + \cdots + b_k 3^{-k}.
\]
Finally, a string, $w = u_1cu_2c \cdots cu_n \in L_M$, is interpreted
as the point, $(x_w, y_w, z_w)\in \reals^3$, where
\[
x_w  = n(w^1),\> y_w = n(w^2),\> z_w = n(w^3).
\]
Therefore, the language, $L_M$, is the encoding of a set of
rational points in $\reals^3$, call it $M$.
This turns out to be the rational part of a fractal known as
the {\it Menger sponge\/}. 

\medskip
Explain the best you can what are the recursive rules to create
the Menger sponge, starting from a unit cube in $\reals^3$.
Draw some pictures illustrating this process and showing
approximations of the Menger sponge.

\medskip\noindent
{\bf Extra Credit (20 points).}
Write a computer program to draw  the Menger sponge (based on the ideas
above).


\vspace {0.25cm}
\noindent
{\bf Problem B3 (60 pts).} 
Let $R$ be any regular language
over some alphabet $\Sigma$. 

\medskip
(1)
Prove that the language
$$L_1 = \{u \mid \exists v\in\Sigma^*,\, uv\in R,\, |v| = 2|u|\}$$
is regular.

\medskip
(2) Let $k\geq 1$ be any integer.
Prove that the language
$$L_1^k = \{u \mid \exists v\in\Sigma^*,\, uv\in R,\, |v| = k|u|\}$$
is regular.



\vspace{0.25cm}
\noindent
{\bf Problem B4 (30 pts).}
Let $L$ be a regular language. Are the following languages regular,
and if so, give a proof (or construction).

\medskip
(a) $\mathrm{Pre}(L) =\{u \mid \hbox{$u$ is a prefix of some $w\in L$}\}$

\medskip
(b) $\mathrm{Suf}(L) =\{u \mid \hbox{$u$ is a suffix of some $w\in L$}\}$

\medskip
(c) $\mathrm{Sub}(L) =\{u \mid \hbox{$u$ is a substring of some $w\in L$}\}$


\vspace{0.25cm}
\noindent
{\bf Problem B5 (20 pts).}
Let $L$ be any language over some alphabet $\Sigma$.

\medskip
(a) Prove that $L = L^+$ iff $LL \subseteq L$.

\medskip
(b) Prove that $(L =\emptyset$ or $L = L^*)$ iff $LL = L$.


\vspace {0.5cm}\noindent
{\bf Problem B6 (90 pts).} ({\it wqo's\/})
We let $\natnums$ denote the set $\{0,1,2,\ldots\}$ of natural numbers,
and $\natnums_{+}$ denote the set 
$\{1,2,\ldots\}$ of positive natural numbers. Given a set $S$,
an {\it infinite sequence\/} is a function $s : {\bf N}_{+} \rightarrow S$.
An infinite sequence $s$ is also denoted by $\fseq{s}{i}$, or
by $\langle s_{1},s_{2},\ldots,s_{i},\ldots\rangle$.
Given an infinite sequence $s=\fseq{s}{i}$, an {\it infinite subsequence\/}
of $s$ is any infinite sequence
$s'=\fseq{s'}{j}$ such that there is a strictly monotonic function
$\mapdef{f}{\natnums_+}{\natnums_+}$ 
and
$s'_{i}=s_{f(i)}$ for  all $i>0$
(recall that
a function $\mapdef{f}{\natnums_+}{\natnums_+}$ 
is {\it strictly monotonic\/} (or {\it increasing\/}) 
iff for all $i, j>0$, $i<j$ implies that $f(i)<f(j)$).
An infinite subsequence $s'$ of $s$ associated with
the function $f$ is also denoted as
$s'=\fsseq{s}{i}{f}$.

\medskip
We now review preorders and well-foundedness. 

\medskip
Given a set $A$, a binary relation $\preceq\ \subseteq A\times A$ 
on the set $A$ is
a {\it preorder\/} (or {\it quasi-order\/}) iff it is reflexive
and transitive. A preorder that is also antisymmetric is called
a {\it partial order\/}. A preorder is {\it total\/} iff
for every $x, y\in A$, either $x\preceq y$ or $y\preceq x$.
The relation $\succeq$ is defined such that
$x\succeq y$ iff $y\preceq x$, the relation $\prec$ such that 
$$x \prec y\quad\hbox{iff}\quad x \preceq y\quad\hbox{and}\quad
y \not\preceq x,$$
and the relation $\succ$ such that
$x \succ y$ iff $y\prec x$.
We say that $x$ and $y$ are {\it incomparable\/} iff
$x\not\preceq y$ and $y\not\preceq x$, and this is also denoted
by $x\mid  y$. 

\medskip
Given a preorder $\preceq$ over a set $A$, an infinite sequence
$\fseq{x}{i}$ is an {\it infinite decreasing chain\/} iff
$x_{i} \succ x_{i+1}$ for all $i\geq 1$.
An infinite sequence
$\fseq{x}{i}$ is an {\it infinite antichain\/} iff
$x_{i}\mid   x_{j}$ for all $i, j$, $1\leq i<j$.
We say that $\preceq$ is {\it well-founded\/} and that
$\succ$ is {\it Noetherian\/} iff there are no infinite decreasing chains
w.r.t. $\succ$.

\medskip
We now turn to the fundamental concept of a well quasi-order (wqo).

\medskip
Given a preorder $\preceq$ over a set $A$, 
an infinite sequence $\fseq{a}{i}$ of elements in $A$ 
is termed {\it good\/} iff there exist positive integers
$i$, $j$ such that $i < j$ and $a_{i}\preceq a_{j}$, and otherwise, 
it is termed a {\it bad\/} sequence.
A preorder $\preceq$ is a {\it well quasi-order\/},
abbreviated as {\it wqo\/}, iff  every infinite
sequence of elements of $A$ is good.

\medskip
Prove that the standard total ordering $\leq$ on $\natnums$ is a wqo.
If  $\preceq$ is a wqo on a set $A$, a {\it finite\/}
sequence is not  necessarily good (why?).

\medskip
(a)
Prove the following characterizations of {\it wqo\/}'s.
Given a preorder $\preceq$ on a set $A$, the following conditions
are equivalent:
\begin{enumerate}
\item[(1)]
Every infinite sequence is good (w.r.t. $\preceq$).
\medskip
\item[(2)]
There are no infinite decreasing chains 
and no infinite antichains (w.r.t. $\preceq$).
\end{enumerate}

\medskip
Given a preorder $\preceq$ on a set $A$, 
say that a member $s_{i}$ of an infinite sequence $s$ is {\it terminal\/} iff
there is no $j>i$ such that $s_{i}\preceq s_{j}$. 

\medskip
(b) 
Prove that the following statements 
are equivalent:
\begin{enumerate}
\item[(1)]
$\preceq$ is a {\it wqo\/} on $A$.
\medskip
\item[(2)]
Every infinite sequence $s=\fseq{s}{i}$ over $A$ 
contains some infinite subsequence
$s'=\fsseq{s}{i}{f}$ such that $s_{f(i)}\preceq s_{f(i+1)}$ for all $i>0$.
\end{enumerate}

\medskip\noindent
{\it Hint\/}. First, prove that if $\preceq$ is a wqo, then
the number of terminal elements in any infinite  sequence $s$ is finite. 

\medskip
Given two preorders $\langle \preceq_{1},A_{1}\rangle$
and $\langle \preceq_{2},A_{2}\rangle$, the cartesian product
$A_{1}\times A_{2}$ is equipped with the preorder $\preceq$ defined
such that $(a_{1},a_{2})\preceq (a_{1}',a_{2}')$ iff
$a_{1} \preceq_{1} a_{1}'$ and $a_{2} \preceq_{2} a_{2}'$.

\medskip
(c) 
Prove that
if  $\preceq_{1}$ and $\preceq_{2}$ are {\it wqo\/}, 
then  $\preceq$ is a {\it wqo\/} on $A_{1}\times A_{2}$.

\remark
This is due to Nash-Williams.

\medskip
(d) 
Prove the following result.

\medskip
Let $n$ be any  integer such that $n>1$.
Given any infinite sequence $\fseq{s}{i}$ of $n$-tuples of
natural numbers, there exist positive integers $i, j$ such that
$i<j$ and $s_{i}\preceq_{n} s_{j}$, where $\preceq_{n}$ is the
partial order on $n$-tuples of natural numbers
induced by the natural ordering $\leq$ on $\natnums$

\remark
This is due to 
Dickson, 1913!

\medskip
Let $\qleq$ be a preorder on a set $A$. We define the 
preorder $\ll$ ({\it string embedding\/})
on $A^{*}$ as follows:

\medskip\noindent
$\epsilon \ll u$ for each $u\in A^{*}$, and,
for any two strings $u=u_{1}u_{2}\ldots u_{m}$ and
$v=v_{1}u_{2}\ldots v_{n}$, $1\leq m\leq n$, 
$$u_{1}u_{2}\ldots u_{m} \ll v_{1}v_{2}\ldots v_{n}$$
iff there exist  integers $j_{1},\ldots,j_{m}$ such that
$1\leq j_{1} < j_{2} < \ldots < j_{m-1} < j_{m} \leq n$ and
$$u_{1} \qleq v_{j_{1}},\ \ldots,\ u_{m} \qleq v_{j_{m}}.$$

\medskip
(e)
Prove that $\ll$ is a preorder. Prove that
$\ll$ is a partial order if $\qleq$ is a
partial order.
Prove  that $\ll$ is the least preorder on
$A^{*}$ satisfying the following two properties:

\begin{enumerate}
\item[(1)] (deletion property)
$uv \ll uav$, for all $u, v\in A^{*}$ and $a\in A$;
\medskip
\item[(2)] (monotonicity)
$uav\ll ubv$ whenever $a\qleq b$, for all $u, v\in A^{*}$ and $a, b\in A$.
\end{enumerate}

\remark
The following theorem due to  Higman can be proved, but the proof
is quite tricky.

\medskip\noindent
{\bf Theorem\/}
If $\qleq$ is a {\it wqo\/} on $A$, then $\ll$ is a {\it wqo\/} on
$A^{*}$.




\vspace{0.5cm}\noindent
{\bf TOTAL:   270 $+$ 20 points.}

\end{document}
