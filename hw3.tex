\documentclass[12pt]{article}
\usepackage{fullpage}
\usepackage{titlesec}
\usepackage{tikz}
\usepackage{amsfonts,amssymb}
\usepackage{amsmath}
\usepackage{comment}
\usetikzlibrary{automata, positioning}

\input ../libraries/mac.tex
\input ../libraries/mathmac.tex

\begin{document}
\pagestyle{plain}
\titleformat{\subsection}[runin]
  {\normalfont\large\bfseries}{\thesubsection}{1em}{}

\title{Homework 3}
\author{Brooke Fugate, Michael O'Connor, Rohan Shah}
\date{}

\maketitle

\section*{Problem B1}

\section*{Problem B2}

\section*{Problem B3}
\subsection*{a}
Suppose we have a regular language $L$ and that $\exists    D = \{Q, \Sigma, \delta, q_0, F\}$ such that $L(D) = L$. We will show by a proof by construction that $L^{(1/3)} = \{ w \mid www \in L\}$ is regular; \newline First we will create 3 DFAs:
$$D^1_{pq} = \{Q, \Sigma, \delta, q_0, \{p\}\} \text{ and } L(D^1_{pq}) = \{u \in \Sigma^* \mid \delta^* (q_0,u) = p\}$$
$$D^2_{pq} = \{Q, \Sigma, \delta, p, \{q\}\} \text{ and } L(D^2_{pq}) = \{u \in \Sigma^* \mid \delta^* (p,u) = q\}$$
$$D^3_{pq} = \{Q, \Sigma, \delta, q, F\} \text{ and } L(D^3_{pq}) = \{u \in \Sigma^* \mid \delta^* (q,u) \in F\}$$
No we can say that $L^{(1/3)} \subseteq L(D^1_{pq}) \cap L(D^2_{pq}) \cap L(D^3_{pq})$ and that $L^{(1/3)} = \bigcup_{p,q \in Q}  L(D^1_{pq}) \cap L(D^2_{pq}) \cap L(D^3_{pq})$.  Since regular languages are closed under union and intersection $L^{(1/3)}$ is regular since it is made from 3 languages that are regular, they have a DFA. \newline
Proof that the $L^{(1/3)} = \{ w \mid www \in L\}$:\newline
By executing  $D^1_{pq} , D^2_{pq} , D^3_{pq}$ on $w \in L^{(1/3)}$ we would have followed a path such that:
$$\delta^*(\delta^*(\delta^*, w), w), w) \in F \implies$$
$$www \in L \text{ therfore }w \in L^{(1/3)}$$
\newline $L^{(3)}$ is not regular. say $L = \{a^nb \mid n \ge 1\}$ now need to get an approrpriate suffix to break this thing from Myhil-Nerode.
\subsection*{b}
Suppose we have a regular language $L$ and that $\exists    D = \{Q, \Sigma, \delta, q_0, F\}$ such that $L(D) = L$. We will show by a proof by construction that all languages of the form $L^{(1/k)} = \{ w \mid w^k \in L\}$ are regular;
\newline First we will create $k$ regular languages:
$$L^i_{p_i...p_{k-1}} = \{ w \in \Sigma^* \mid \delta^*(p_{i-1},w) = p_i\}, \forall i \mid 2 \le i \le k-1 $$
$$L^1_{p_i...p_{k-1}} = \{ w \in \Sigma^* \mid \delta^*(q_0,w) = p_1\}$$
$$L^k_{p_i...p_{k-1}} = \{ w \in \Sigma^* \mid \delta^*(p_{k-1},w) \in F\}$$
From this we can see that $L = \bigcup_{p_i...p_{k-1} \in Q} \bigcap^k_{i=1} L^i_{p_i...p_k}$. We must now prove that they are equal: \newline
By executing a string $w$ on the DFAs we have created we would have followed a path such that: $\sigma^*(q_0,w) = p_1 , \sigma^*(p_1, w) = p_2 ... \sigma^*{p_{k-1}, w} \in F$.  Therefore we would have a path from $q_0$ to $q_F \in F$ that is for the word $w^k$. So all workable strings on the right are in the left.
\newline For all of the strings in in the left to the right, we will choose a p to get started so it works.
\subsection*{c}
Is $L^{1/\infty}$ regular? $L^{1/\infty} = \bigcap_{k-1}L^{1/k}$.  From above we know all those languages are regular, and $k$ is finite so yes $L^{1/\infty}$ is regular.\newline
Is $\sqrt{L}$ regular? $\sqrt{L} = \bigcup_{k-1}L^{1/k}$.  From above we know all those languages are regular, and k is finite so yes $\sqrt{L}$ is regular

\section*{Problem B4}

\section*{Problem B5}

\section*{Problem B6}

\end{document}
